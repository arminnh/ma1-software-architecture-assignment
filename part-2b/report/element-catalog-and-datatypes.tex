% \chapter{Element Catalog and Datatypes}\label{ch:elements-datatypes}

% Delete the command below to remove the hints and instructions
\showcatalognotes{}

\section{Element catalog}\label{app:catalog}
    \todoinline{
    List all components and describe their responsibilities and provided
    interfaces.
    Per interface, list all methods using a Java-like syntax and describe their
    effect and exceptions if any.
    List all elements and interfaces alphabetically for ease of navigation.
    }

    \componentItem{ComponentZ}{
    	\begin{itemize}[noitemsep,nolistsep]
    		\item \textbf{Responsibility:} Responsibilities of the component.
    		\item \textbf{Super-component:} The direct super-component, if any.
    		\item \textbf{Sub-components:} the direct sub-components, if any.
    	\end{itemize}
    	\subsubsection*{Provided interfaces}
    	\begin{itemize}[noitemsep,nolistsep]
    		\item InterfaceA
    		\begin{itemize}
    			\item \texttt{returntType1 operation1(ParamType param) throws SomeException}
    			\begin{itemize}
    				\item Effect: Describe the effect of the operation
    			\end{itemize}
    			%
    			\item \texttt{void operation2(ParamType2 param)}
    			\begin{itemize}
    				\item Effect: Describe the effect of the operation
    				\item Exceptions: None
    			\end{itemize}
    		\end{itemize}
    		%
    		\item InterfaceB
    		\begin{itemize}
    			\item \texttt{returntType2 operation3()}
    			\begin{itemize}
    				\item Effect: Describe the effect of the operation
    			\end{itemize}
    		\end{itemize}
    	\end{itemize}
    	}

    \componentItem{ComponentA}{
    	\begin{itemize}[noitemsep,nolistsep]
    		\item \textbf{Responsibility:} Responsibilities of the component.
    		\item \textbf{Super-component:} The direct super-component, if any.
    		\item \textbf{Sub-components:} the direct sub-components, if any.
    	\end{itemize}
    	\subsubsection*{Provided interfaces}
    	\begin{itemize}[noitemsep,nolistsep]
    		\item InterfaceC
    		\begin{itemize}
    			\item \texttt{returntType1 operation1(ParamType param) throws SomeException}
    			\begin{itemize}
    				\item Effect: Describe the effect of the operation
    			\end{itemize}
    			%
    			\item \texttt{void operation2(ParamType2 param)}
    			\begin{itemize}
    				\item Effect: Describe the effect of the operation
    			\end{itemize}
    		\end{itemize}
    		%
    		\item InterfaceD
    		\begin{itemize}[noitemsep,nolistsep]
    			\item \texttt{returntType2 operation3()}
    			\begin{itemize}
    				\item Effect: Describe the effect of the operation
    			\end{itemize}
    		\end{itemize}
    	\end{itemize}
    }

    % This will alphabetically print the list of components.
    \printComponents


\section{Common interfaces}
    \todoinline{If you have any common interfaces used by multiple components you may define them here and refer to them.}

\section{Defined Exceptions}
    \todoinline{Instead of describing the exceptions with each operation, you may define common exceptions here and refer to them from the operation definition.}

\section{Defined data types}\label{app:datatypes}
    \todoinline{
    List and describe all data types defined in your interface specifications. List
    them alphabetically for ease of navigation.
    }

    \begin{itemize}
    	\item \texttt{Paramtype1}: Description of data type.
    	\item \texttt{Paramtype2}: Description of data type.
    	\item \texttt{returnType1}: Description of data type.
    \end{itemize}
