\section{P2: Requests to the pluggable data database}

    \subsection*{Key Decisions}

        \begin{itemize}
            \item \texttt{PluggableDeviceDataDB} separates the requests concerning pluggable data
                  so that those requests have no impact on requests concerning other data.
        	\item \texttt{DeviceDataScheduler} handles all requests for the \texttt{PluggableDeviceDataDB},
                  recognizes when the data processing subsystem needs to run in normal or overload mode,
                  and prevents starvation of requests.
        \end{itemize}
        \emph{Employed tactics and patterns:} \ldots

    \subsection*{Rationale}
        \paragraph{P2: Scheduling}
            The pluggable data processing subsystem needs to be able to run in normal
            or overload mode, depending on whether or not the system can process
            requests within the deadlines given in the quality requirement. Also,
            a mechanism should be in place to avoid starvation of any type of request. \\
            The \texttt{PluggableDeviceDataScheduler} is used to deal with this problem.
            It is responsible for scheduling requests that wish to interact with
            the \texttt{PluggableDeviceDB}. In normal mode, the system processes
            incoming requests in a FIFO order. In overload mode, the requests are
            given a priority based on what the request is for and what the source
            of the request is. The requests are then not simply processed in an
            order based on their priorities, but an aging technique is to be used
            such that starvation will be avoided. Thus, in overload mode,
            requests are processed in an order based on a combination of the
            priorities of the requests and the age of the requests.

        \paragraph{P2: Pluggable data separation}
            The processing of (large amounts of) requests concerning pluggable
            data has no impact on requests concerning other data, e.g. available applications. \\
            In order to statisfy this constraint, all data directly related to
            pluggable data has been separated into the \texttt{PluggableDeviceDB}.
            All requests concerning pluggable data will be handled by this new
            component. \texttt{PluggableDeviceDB} will run on a node different
            from the node that the \texttt{Datbase} component runs on. This way
            requests concerning pluggable will have no impact on
            requests concerning other data.

    \subsection*{Considered Alternatives}
        \paragraph{Alternative(s) for choice 1} Explain what alternative(s) you
        considered for this design choice and why they where not selected.

    \subsection*{Deployment Decisions}
        \ldots

    \subsection*{Considered Deployment Alternatives}
        \ldots
