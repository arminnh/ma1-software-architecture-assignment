\section{P1: Large number of users}

    \subsection*{Rationale}
        The use cases mentioned in P1 are mostly related to Infrastructure Owner requests and Customer Organisation requests.
        Our split of the databases into \texttt{PluggableDeviceDB}, \texttt{DeviceDB}, \texttt{OtherDataDB}, goes well with this statment,
        as most of the requests of Infrastructure Owners lead to requests in the \texttt{DeviceDB} and most of the requests of Customer Organisation
        lead to requests in the \texttt{OtherDataDB}. This means that as both of these groups of users grow, the growth of one group will not
        influence the performance noticed by the other group. \\

        P1 also states that scaling up to service an increasing amount of infrastructure owners, customers organisations
        and applications should (in worst case) be linear.
        This is certainly possible to an extent, as all the components related to requests from infrastructure
        owners and customers organisations are stateless. They could be replicated and load balancing could be applied. \\
        The story for applications is slightly more complex, as at a certain point more \texttt{ApplicationContainerManager} nodes
        would be necessary to keep track of all \texttt{ApplicationContainer}s. This would then also create a need for some
        kind of coordinator so that all \texttt{ApplicationContainerManager}s can work together correctly to keep track of and maintain
        all \texttt{ApplicationContainer}s. \\

        A problem for linear scaling we did not handle is the performance of the databases. In case replication and load
        balancing is applied on other components to handle a growing number of users,
        the databases will become the bottleneck of the system. To handle this, replication for performance should be used on the databases. \\

        Another problem that we didn't handle is the scalability of the databases themselves. Ideally, a database management system
        that supports linear scalability should be used.
