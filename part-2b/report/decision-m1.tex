\section{M1: Integrate new sensor or actuator manufacturer}

    \subsection*{Key Decisions}

    \begin{itemize}
    	\item Modifiability is maintained by splitting up functionality that would need to be updated
              when new pluggable devices are introduced to the system into different components.
        \item In interfaces, pluggable devices are only referred to by their unique PluggableDeviceID,
              types and other device info is left out of parameter lists.
        \item \texttt{PluggableDeviceDataDB} and \texttt{DeviceDB} store data about devices in
              a generic way.
    \end{itemize}
    \emph{Employed tactics and patterns:} generalize modules

    \subsection*{Rationale}
        M1 states that available application instances in the system can be updated to use any new pluggable device.
        This is made possible by not adding specific device type information in any method parameters.
        All available applications can request data (be it sensor data or just specific data about a device)
        from new types of devices simply by referring to them by their ID, if they have the rights
        to use that device. The new type of info will then be encapsulated by the \texttt{DeviceData} data type.
        This means that new information would have to be added in the \texttt{DeviceDB} about the
        new pluggable device (and possible its new type). \\
        Thus, the new types of sensor or actuator data can be transmitted,
        processed and stored, and be made available to applications.\\

        For the problem of data conversion, we have added the \texttt{DeviceDataConverter}.
        This component can be used by \texttt{ApplicationInstance}s to convert data into
        a desired format. This component simply contains methods to convert data.
        If a new type of data were to be added to the system, then some new methods
        could be added to this component to support converting to/from that new
        type of data. The component would then have to be re-deployed.\\

        The \texttt{DeviceCommandConstructor} would need to be able to verify the correctness of actuation
        and configuration commands and generate the actual commands for the devices. This is handled
        by letting the \texttt{DeviceCommandConstructor} fetch the necessary data from the \texttt{DeviceDB}.
        Thus, this kind of information should also be added to the \texttt{DeviceDB} when adding a
        new device to the system.\\

        M1 also states that infrastructure managers must be able to initialize the new type
        of pluggable device, configure access rights for these devices, and
        view detailed information about the new type of pluggable device.
        As all of the necessary data for these scenarios about the pluggable device comes from the \texttt{DeviceDB},
        nothing else needs to be done for this point.
