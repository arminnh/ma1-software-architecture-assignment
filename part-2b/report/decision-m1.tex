\section{M1: Integrate new sensor or actuator manufacturer}

    \subsection*{Key Decisions}

    \begin{itemize}
    	\item Modifiability is maintained by splitting up functionality that would need to be updated
              when new pluggable devices are introduced to the system into different components.
        \item In interfaces, pluggable devices are only referred to by their unique PluggableDeviceID,
              types and other device info is left out of parameter lists.
        \item \texttt{PluggableDeviceDataDB} and \texttt{DeviceDB} store data about devices in
              a generic way.

    \end{itemize}
    \emph{Employed tactics and patterns:} None

    \subsection*{Rationale}
        \paragraph{M1: Data conversion}
            With new types of devices, the pluggable data processing subsystem
            should be extended with relevant data conversions,
            e.g. converting temperature in degrees Fahrenheit to degrees Celsius. \\
            The \texttt{DeviceDataConverter} is put in place to handle the
            task of converting pluggable device data to data of a different type in the system.
            This component can easily be modified for new types of data simply by
            adding a new conversion method for the new.

        \paragraph{M1: Usage of new data by applications}
            The available applications in the system can be updated to use any
            new pluggable devices. \\
            This is made possible by the RequestData
            interface provided by \texttt{DeviceDataScheduler}.
            Data of the new type of device can be requested in the same way
            as for older devices: by using the device's unique id.
            The application manager can get pluggable device data from the
            \texttt{PluggableDeviceDataDB} and return this data to applications in
            the DeviceData datatype. This datatype can easily be
            updated for new types of pluggable devices.

        \paragraph{P2: Scheduling}
            The pluggable data processing subsystem needs to be able to run in normal
            or overload mode, depending on whether or not the system can process
            requests within the deadlines given in the quality requirement. Also,
            a mechanism should be in place to avoid starvation of any type of request. \\
            The \texttt{DeviceDataScheduler} is used to deal with this problem.
            It is responsible for scheduling requests that wish to interact with
            the \texttt{PluggableDeviceDataDB}. In normal mode, the system processes
            incoming requests in a FIFO order. In overload mode, the requests are
            given a priority based on what the request is for and what the source
            of the request is. The requests are then not simply processed in an
            order based on their priorities, but an aging technique is to be used
            such that starvation will be avoided. Thus, in overload mode,
            requests are processed in an order based on a combination of the
            priorities of the requests and the age of the requests.

        \paragraph{P2: Pluggable data separation}
            The processing of (large amounts of) requests concerning pluggable
            data has no impact on requests concerning other data, e.g. available applications. \\
            In order to statisfy this constraint, all data directly related to
            pluggable data has been separated into the \texttt{PluggableDeviceDataDB}.
            All requests concerning pluggable data will be handled by this new
            component. \texttt{PluggableDeviceDataDB} will run on a node different
            from the node that the \texttt{Datbase} component runs on. This way
            requests concerning pluggable will have no impact on
            requests concerning other data.

        \paragraph{M1: Handling new types of pluggable devices}
            The new types of sensor or actuator data should be transmitted,
            processed and stored, and should be made available to applications.
            The infrastructure managers must be able to initialize the new type
            of pluggable device, configure access rights for these devices, and
            view detailed information about the new type of pluggable device. \\
            The components created thus far have been created with high cohesion in
            mind so that updating them for new devices would be relatively straightforward.
            In order for this constraint to be satisfied, changes have to be made to
            the following elements:
            \begin{itemize}
                \item \emph{PluggableDevice}: This component needs to be updated
                      so that the new type of device can be initialised and configured,
                      and thus so that the device's data can be sent to the system.
                \item \emph{DeviceData}: Depending on how this data type
                      is implemented, it might need an update in order for it
                      to represent possible new data types (for example
                      Temperature Filipcikova) and for the new data types to be
                      serialized.
                \item \emph{PluggableDeviceDataDB}: The database needs to be updated
                      so that information can be retrieved about the new types
                      of sensors and the new types of data. Data related to the
                      displaying of sensor data will also need to be updated.
                \item \emph{PluggableDeviceConverter}: see above.
            \end{itemize}

    \subsection*{Considered Alternatives}
        \paragraph{Alternative(s) for choice 1} Explain what alternative(s) you
        considered for this design choice and why they where not selected.

    \subsection*{Deployment Decisions}
        \ldots

    \subsection*{Considered Deployment Alternatives}
        \ldots
