\documentclass[english]{sareport}
% use the option peerreview for creating an anonymized version of your report
% E.g., \documentclass[english,peerreview]{sareport}

\usepackage[colorlinks, linkcolor=black, citecolor=black, urlcolor=black]{hyperref}
\usepackage{fontawesome}
\usepackage[normalem]{ulem}

\usepackage{graphicx}
\usepackage{rotating}
\usepackage{float}
\usepackage{enumitem}
\setlist{noitemsep} % \setlist{nosep}

% Set all authors, if your group counts 2, set third author empty \authorthree{}
% Set the groupname as well
\authorone{Monika Filipcikova (r0683254)}
\authortwo{Armin Halilovic(r0679689)}
\authorthree{}
\groupname{Filipcikova-Halilovic}
\academicyear{2016--2017}

\casename{Shared Internet Of Things Infrastructure Platform}
\phasenumber{2b}
\phasename{The Complete Architecture}

\begin{document}
\maketitle

\tableofcontents
% the following two command are necessary for obtaining the mini list of figures in the cs-view, decomposition view, deployment view and scenarios chapters
\dominilof
\fakelistoffigures

\chapter{Architectural Decisions}\label{sec:overview}
%\section{Decomposition 1: Av3, UC14, UC15, UC18 (SIoTIP System)}


\subsection*{Selected architectural drivers}
    The non-functional drivers for this decomposition are:
    \begin{itemize}
    	\item \emph{Av3}: Pluggable device or mote failure
    \end{itemize}

    The related functional drivers are:
    \begin{itemize}
    	\item \emph{UC14}: Send heartbeat (Av3) \\
              This use case checks whether or not motes and pluggable devices
              are still operational.
    	\item \emph{UC15}: Send notification (Av3) \\
              This use case sends a notification to a registered user.
    	\item \emph{UC18}: Check and deactivate applications (Av3) \\
              This use case deactivates any application that requires deactivation,
              because of unavailability of essential pluggable devices
              or unassigned mandatory roles.
    \end{itemize}

    \paragraph{Rationale}
        Av3 was chosen first since it has high priority and it is more relevant to
        the core of the system than the other quality requirements with high
        priority (M1 and U2).
        We believe that handling pluggable device failure/connectivity is
        more important to the whole of the system than M1 and U2, and that
        handling this first would give a stronger starting point for later ADD iterations
        than M1 or U2.


\subsection*{Architectural design}\label{sec:architectural-design}
    This section describes what needs to be done to satisfy the requirements for
    this decomposition and how involved problems/obstacles are solved.

    \paragraph{Av3: Failure detection}
        Gateway need to be able to autonomously detect failure of one of its
        connected motes and pluggable devices. This is achieved by making motes
        send heartbeats to their connected gateways. The gateways can
        then monitor their connected devices. The heartbeats contain a list
        of devices that are connected/operational at the moment the mote sends
        the heartbeat. Each gateway makes use of a \texttt{DeviceManager}
        component to monitor the devices. This component uses timers to keep track
        of how long it has been since a device has sent a heartbeat or occured in
        a list of connected devices. Once a timer expires, this is treated as
        a failure. \\

        A mote has failed when 3 consecutive heartbeats do not arrive within 1
        second of their expected arrival time. \\
        A pluggable device has failed when it does not occur in a heartbeat of the
        mote in which it is expected to be in. This is is detected within 2
        seconds after the arrival of the heartbeat.

    \paragraph{Av3: Automatic application deactivation and redundancy settings}
        Applications should be automatically suspended when they can no longer
        operate due to failure of a pluggable device or mote and reactivated
        once the failure is resolved. Application providers can design their
        applications such that they explicitly require redundancy in
        the available pluggable devices. \\
        This problem is tackled by the \texttt{DeviceManager}. It
        stores the requirements for pluggable devices set by applications for all
        applications that use the gateway that the the \texttt{DeviceManager}
        runs on. When it detects that an application can no longer operate
        due to failures, it will send a command to the \texttt{ApplicationManager}
        (via the \texttt{GatewayFacade})
        to suspend that application. When the required devices are operational
        again, the \texttt{DeviceManager} detects this and sends a
        command to reactivate the application. \\

        Applications are suspended within 1 minute after detecting
        the failure of an essential pluggable device. \\
        Application are reactivated within 1 minute after the failure is resolved.

    \paragraph{Av3: Notifications}
        The infrastructure owner should be notified of any persistent
        pluggable device or mote failures. Customer organisations should be
        notified if one or more of their applications is suspended or
        reactivated. Applications using a failed pluggable device or any device
        on a failed mote should be notified. \\
        The \texttt{NotificationHandler} was put in place to deal with
        notifications. Other components can use it to generate notifications for
        certain users in the system. The \texttt{NotificationHandler} will then
        insert information relevant to the notification in the database (message,
        status, date and time, source, ...), and use an external delivery
        service to deliver the notification to users. The used delivery medium
        is based on the user's preferences. \\
        Since they are stored in the database, users can always view
        their notifications via their dashboard. However, this funcionality is not
        expanded on in this decomposition yet. \\

        Infrastructure owners are notified within 1 minute after detecting a mote outage lasting at
        least 10 seconds. \\
        Infrastructure owners are notified within 1 minute after the detection of the unavailability of
        a pluggable device for 30 seconds. \\
        Applications are notified of the failure of relevant pluggable devices within 10 seconds.

    \subsubsection{Alternatives considered}
        \paragraph{Av3: Failure detection}
            An alternative would have been to move the \texttt{DeviceManager}
            component from gateways to the Online Service. This solution would make the
            gateways do less work, but would be very unscalable. The reason is
            that as the customer base (and thus the amount of devices) increases,
            the Online Service would need to keep track of huge amounts of devices.
            This would also flood the network to the Online Service with heartbeats.

        \paragraph{Av3: Failure detection}
            Another alternative for failure detection could have been the use of
            a Ping/Echo mechanism instead of Heartbeats. Pings could then be used
            to check if a device is currently operational. However, as a device could
            not be operational for a moment because of e.g. interference, timers
            would still be necessary to keep track of operational devices. We opted
            to use heartbeats, as this would reduce the amount of data sent over
            the network used by the motes, and as motes would have to do slightly
            more work to process each Ping request in order to generate a reply.

        \paragraph{Av3: Notifications}
            Reliable and quick delivery of notifications is crucial to the
            system in order to solve problems should things go wrong. Currently,
            the solution is to use a third party service for delivery of
            notifications. In the case that no external services are found
            satisfactory, or if this dependency on an external service is
            unwanted, it is possible to build an internal solution for this.
            For example, a \texttt{NotificationSender} component could make use
            of the \texttt{Factory pattern} for different message channels for
            different delivery methods (each with their own sendNotification method).
            This solution allows us to easily add new message channels in the
            future with little effort. The disadvantage of this is that an
            internal solution takes a lot more time to implement.


\subsection*{Instantiation and allocation of functionality}
    This section lists the new components which instantiate our solutions
    described in the section above. For each component we note the quality
    attribute or use case that prompted us to create it. Descriptions about
    the components can be found under chapter \ref{ch:elements-datatypes}. \\

    \begin{itemize}
        \item ApplicationManager: Av3
        \item Database: /
        \item DeviceManager: Av3
        \item GatewayFacade: /
        \item Mote: UC14
        \item NotificationHandler: UC15
    \end{itemize}

    \paragraph{Decomposition}
        Figure \ref{fig:it1-cc_main} shows the components resulting from the
        decomposition in this run.

        \begin{figure}[!htp]
        	\centering
        	\includegraphics[width=1.00\textwidth]{images/component-diagram-1}
        	\caption{Component-and-connector diagram of this decomposition.}
            \label{fig:it1-cc_main}
        \end{figure}

    \paragraph{Deployment}
        Figure \ref{fig:it1-depl_main} shows the allocation of components
        to physical nodes.

        \begin{figure}[!htp]
        	\centering
        	\includegraphics[width=0.55\textwidth]{images/deployment-diagram-1}
        	\caption{Deployment diagram of this decomposition.}\label{fig:it1-depl_main}
        \end{figure}


\subsection*{Interfaces for child modules}\label{add1-interfaces}
    This section lists new interfaces assigned to the components defined
    in the section above. Detailed information about each interface and
    its methods can be found under chapter \ref{ch:elements-datatypes}.

    \subsubsection{ApplicationManager}
        \begin{itemize}
            \item GWAppInstanceMgmt
        \end{itemize}

    \subsubsection{Database}
        \begin{itemize}
            \item NotificationMgmt
            \item AppMgmt
        \end{itemize}

    \subsubsection{GatewayFacade}
        \begin{itemize}
            \item Heartbeat
            \item DeviceData
            \item DeviceMgmt
            \item AppDeviceMgmt
        \end{itemize}

    \subsubsection{Mote}
        \begin{itemize}
            \item DeviceMgmt
        \end{itemize}

    \subsubsection{NotificationHandler}
        \begin{itemize}
            \item Notify
            \item DeliveryMgmt
        \end{itemize}

    \subsubsection{External notification delivery service}
        \begin{itemize}
            \item NotificationDeliveryMgmt
        \end{itemize}

    \subsubsection{DeviceManager}
        \begin{itemize}
        	\item DeviceMgmt
        \end{itemize}


\subsection*{New data types}
    This section lists the data types introduced in this decomposition.

    \begin{itemize}
        \item{PluggableDeviceInfo}
        \item{Notification}
        \item{ApplicationInstance}
        \item{Subscription}
        \item{PluggableDeviceID}
        \item{PluggableDeviceType}
        \item{DeviceData}
        \item{Map<String,String>}
    \end{itemize}

\subsection*{Verify and refine}
    The selected architectural drivers have been handled completely
    in this decomposition.
    This section describes per component which (parts of) the remaining
    requirements it is responsible for. If requirements are split in
    multiple parts, this is indicated by the addition of a letter
    (or number, depending on the structure of the requirement) after their title.

    \paragraph{ApplicationManager}
        \begin{itemize}
            \item \emph{Av2}: Application failure \\
                   Prevention: a, b \\
                   Detection: a, b, c \\
                   Resolution: a, b, c
           \item \emph{P1}: Large number of users: c
           \item \emph{M1}: Integrate new sensor or actuator manufacturer: 1.c, 2.a
           \item \emph{M2}: Big data analytics on pluggable data and/or application usage data: d, e
           \item \emph{U1}: Application updates: a, b, c, d
           \item \emph{U2}: Easy Installation: e
           \item \emph{UC12}: Perform actuation command
           \item \emph{UC17}: Activate an application: 3, 4
        \end{itemize}

    \paragraph{Database}
        \begin{itemize}
          	\item None
        \end{itemize}

    \paragraph{GatewayFacade}
        \begin{itemize}
            \item \emph{Av1}: Communication between SIoTIP gateway and Online Service \\
                              Resolution: b, c, d
            \item \emph{M1}: Integrate new sensor or actuator manufacturer: 1.a, 2.b
            \item \emph{U2}: Easy Installation: a, c, d
            \item \emph{UC11}: Send pluggable device data: 1
        \end{itemize}

    \paragraph{Mote}
        \begin{itemize}
            \item \emph{M1}: Integrate new sensor or actuator manufacturer: 1.a, 2.b
            \item \emph{U2}: Easy Installation: b, c, d
            \item \emph{UC04}: Install mote: 1, 2
            \item \emph{UC05}: Uninstall mote: 1
            \item \emph{UC06}: Insert a pluggable device into a mote: 2
            \item \emph{UC07}: Remove a pluggable device from its mote: 2
            \item \emph{UC11}: Send pluggable device data: 1
        \end{itemize}

    \paragraph{NotificationHandler}
        \begin{itemize}
            \item \emph{UC16}: Consult notification message: 5
            \item \emph{UC17}: Activate an application: 5, 6
        \end{itemize}

    \paragraph{OtherFunctionality1}
        \begin{itemize}
            \item \emph{Av1}: Communication between SIoTIP gateway and Online Service \\
                               Detection: a, b, c, d
                               Resolution: a
           	\item \emph{P1}: Large number of users: a
            \item \emph{P2}: Requests to the pluggable data database
            \item \emph{M1}: Integrate new sensor or actuator manufacturer: 1.d
            \item \emph{M2}: Big data analytics on pluggable data and/or application usage data: a
            \item \emph{U2}: Easy Installation: e
            \item \emph{UC01}: Register a customer organisation
            \item \emph{UC02}: Register an end-user
            \item \emph{UC03}: Unregister an end user
            \item \emph{UC04}: Install mote: 3
            \item \emph{UC05}: Uninstall mote: 2.b
            \item \emph{UC06}: Insert a pluggable device into a mote: 3: topology part; alternative 3a.1.b
            \item \emph{UC07}: Remove a pluggable device from its mote: 3.b
            \item \emph{UC08}: Initialise a pluggable device: 1, 2, 4
            \item \emph{UC09}: Configure pluggable device access rights
            \item \emph{UC10}: Consult and configure the topology
            \item \emph{UC11}: Send pluggable device data: 3
            \item \emph{UC13}: Configure pluggable device
            \item \emph{UC16}: Consult notification message: 1, 2, 3, 4
            \item \emph{UC17}: Activate an application: 1, 2
            \item \emph{UC19}: Subscribe to application
            \item \emph{UC20}: Unsubscribe from application
            \item \emph{UC21}: Send invoice
            \item \emph{UC22}: Upload an application
            \item \emph{UC23}: Consult application statistics
            \item \emph{UC24}: Consult historical data
            \item \emph{UC25}: Access topology and available devices
            \item \emph{UC26}: Send application command or message to external front-end
            \item \emph{UC27}: Receive application command or message to external front-end
            \item \emph{UC28}: Log in
            \item \emph{UC29}: Log out
        \end{itemize}

    \paragraph{DeviceManager}
        \begin{itemize}
            \item \emph{U2}: Easy Installation: c, d
            \item \emph{UC04}: Install mote: 4
            \item \emph{UC05}: Uninstall mote: 2
            \item \emph{UC06}: Insert a pluggable device into a mote: 3: uninitialised part; alternative 3a.1 3a.2 3a.4; 4
            \item \emph{UC07}: Remove a pluggable device from its mote: 3.a, 3.c
            \item \emph{UC08}: Initialise a pluggable device: 3
            \item \emph{UC11}: Send pluggable device data: 2, 3a
        \end{itemize}

%\newpage
%\section{Decomposition 2: OtherFunctionality (M1, P2, UC11)}

\subsection{Module to decompose}
    In this run we decompose OtherFunctionality.


\subsection{Selected architectural drivers}
    The non-functional drivers for this decomposition are:
    \begin{itemize}
    	\item \emph{M1}: Integrate new sensor or actuator manufacturer
        \item \emph{P2}: Requests to the pluggable data database
    \end{itemize}

    \noindent The related functional drivers are:
    \begin{itemize}
        \item \emph{UC11}: Send pluggable device data (P2) \\
              This use case stores pluggable device data in the pluggable device data storage.
              This could be a sensor reading, or an actuator status.
    \end{itemize}

    \paragraph{Rationale}
    We choose M1 because it belogs to the quality attributes with hight priority. M1 is about
    integration of new sensor or actuator. And it is very important to easily add new devices, because market grows very fast
    and new applications are developing. So we want to focus on this quality attribute
    in the early stages and then based on that create other functionality and components.
    And we also choose P2 because it is related to M1. M1 required minimal changes to data processing and storage,
    so we have to deal with good solution for this topic.  
        
     %Why we do dis???? One was high priority and P2 is related. THey are family and family belongs together.


\subsection{Architectural design}
    % Tactics:
    %     Limit event response? reply within response measure deadlines
    %     Prioritize events
    %     Introduce concurrency
    %     Schedule resources

    \paragraph{Handling new types of pluggable devices for M1}
        The developers have to make changes to: component1, component2, datatype X.
        The new type of sensor needs to be able to be initialised so that it can send data.
        Thus, the PluggableDeviceFacade code that initialises devices should be updated for
        each new type of sensor. The PluggableDeviceData datatype should be updated to
        represent the new type of data. In this case, the new type will have to be added
        to the database that contains all different types of sensor data.

    \paragraph{Data conversions for M1}
        \texttt{The PluggableDeviceDataConverter} is resposible for converting data 
         in system, for instance converting temperature in degrees Fahrenheit 
         to degrees Celsius. System has to work with relevant data, 
         otherwise problem may arise. 
         

    \paragraph{Usage of new data by applications for M1}
        This is possible through the RequestData interface provided by PluggableDeviceDataScheduler.
        The application manager can get device data from the PluggableDeviceDB and return this
        data to applications in the PluggableDeviceData datatype. This datatype can easily be
        updated for new types of pluggable devices.

    \paragraph{Configuration of new device by infrastructure owners for M1}
        Initialisation: IO triggers the initialise() function which has been
        updated for the new pluggable device -> OK\\
        Configure access rights: has absolutely fucking nothing to do with the
        new sensor type -> OK \\
        Consult and configure topology: same as configure access rights

    \paragraph{Scheduling for P2}
        dynamic priority scheduling \\
        tactics: schedule resource, prioritize events, also limit event response?\\
        starvation avoidance

    \paragraph{Pluggable data separation for P2}
        "pluggable data has no impact on other data"
        two databases

    \subsubsection{Alternatives considered}
        \paragraph{Alternatives for solution}
            A discussion of the alternative solutions and why that were not selected.


\subsection{Instantiation and allocation of functionality}
    \paragraph{Decomposition}
        Main aspects of the resulting decomposition.

        \begin{figure}[!htp]
        	\centering
            \includegraphics[width=1.00\textwidth]{component-diagram-2}
        	\caption{Component-and-connector diagram of this decomposition.}
            \label{fig:it1-cc_main}
        \end{figure}

    \subparagraph{PluggableDeviceDB}
        store data related to pluggable devices

    \subparagraph{PluggableDeviceDataScheduler}
        scheduling, detect overload mode, store data, forward data

    \subparagraph{PluggableDeviceDataConverter}
        M1: conversion of new type of data of new type of device

    % \paragraph{Behaviour}
        % A SEQUENCE DIAGRAM FOR UC11 WOULD BE ACTUALLY VERY USEFUL (shows how the gateway checks if devices are initialised)
        % REMOVE THIS PART BECAUSE MONEYKA IS LAAAAZZZZYYYYYY BAD STUDENT "IT IS NOT NECESSARY"

    \paragraph{Deployment}
        Rationale of the allocation of components to physical nodes.

        \begin{figure}[!htp]
        	\centering
        	\includegraphics[width=1.00\textwidth]{deployment-diagram-2}
        	\caption{Deployment diagram of this decomposition.
        	}\label{fig:it1-depl_main}
        \end{figure}


\subsection{Interfaces for child modules}

    \subsubsection{GatewayFacade}
        See "\ref{ADD1-int-gatewayfacade}: GatewayFacade" for the rest of the interfaces provided by this component.
        \begin{itemize}
            \item MoteDataMgmt
            \begin{itemize}
                \item \texttt{void sendData(PluggableDeviceData data)}
                \begin{itemize}
                    \item Effect: Sends pluggable device data to the connected mote.
                    \item Exceptions: None
                \end{itemize}
            \end{itemize}

            \item DeviceMgmt
            \begin{itemize}
                \item \texttt{void initialiseDevice(int deviceID, PluggableDeviceSettings settings)}
                \begin{itemize}
                    \item Effect: Initialises a pluggable device for use with the system.
                    \item Exceptions: None
                \end{itemize}
            \end{itemize}

            \item AppDeviceMgmt
            \begin{itemize}
                \item \texttt{void configurePluggableDevice(int deviceID, PluggableDeviceSettings settings)}
                \begin{itemize}
                    \item For: Use case 11 step 3.b
                    \item Effect: Causes certain settings to be set on a pluggable
                          device that the gateway is connected to.
                    \item Exceptions: None
                \end{itemize}
            \end{itemize}
        \end{itemize}

    \subsubsection{MoteFacade}
        See "\ref{add1-int-motefacade}: MoteFacade" for the rest of the interfaces provided by this component.
        \begin{itemize}
            \item PluggableDeviceDataMgmt
            \begin{itemize}
                \item \texttt{void sendData(PluggableDeviceData data)}
                \begin{itemize}
                    \item Effect: Sends pluggable device data to the connected mote.
                    \item Exceptions: None
                \end{itemize}
            \end{itemize}

            \item PluggableDeviceMgmt
            \begin{itemize}
                \item \texttt{void initialise(int deviceID, PluggableDeviceSettings settings)}
                \begin{itemize}
                    \item Effect: Initialises a connected pluggable device according to some settings
                    \item Exceptions: None
                \end{itemize}
            \end{itemize}
        \end{itemize}

    \subsubsection{PluggableDeviceFacade}
        \begin{itemize}
        	\item PluggableDeviceMgmt
        	\begin{itemize}
                \item \texttt{void initialise(PluggableDeviceSettings settings)}
                \begin{itemize}
                    \item Effect: Initialises the pluggable device according to some settings
                    \item Exceptions: None
                \end{itemize}
        	\end{itemize}
        \end{itemize}

    \subsubsection{PluggableDeviceManager}
        \begin{itemize}
        	\item DeviceListMgmt
        	\begin{itemize}
        		\item \texttt{bool isDeviceInitialised(int deviceID)}
        		\begin{itemize}
        			\item Effect: Returns true if the device with id "deviceID" has been initialized.
        			\item Exceptions: None
        		\end{itemize}
        	\end{itemize}
        \end{itemize}

    \subsubsection{PluggableDeviceDataScheduler}
        \begin{itemize}
            \item RequestData
            \begin{itemize}
                \item \texttt{List<PluggableDeviceData> requestData(int applicationID, int deviceID, DateTime from, DateTime to)}
                \begin{itemize}
                    \item Effect: Request data from a specific device in a certain time period
                    \item Exceptions: None
                \end{itemize}
            \end{itemize}

            \item PluggableDeviceDataMgmt
            \begin{itemize}
                \item \texttt{void sendData(PluggableDeviceData data)}
                \begin{itemize}
                    \item Effect: Sends pluggable device data to the scheduler to be processed.
                    \item Exceptions: None
                \end{itemize}
            \end{itemize}
        \end{itemize}

    \subsubsection{PluggableDeviceDB}
        \begin{itemize}
            \item PluggableDeviceDataMgmt
            \begin{itemize}
                \item \texttt{void sendData(PluggableDeviceData data)}
                \begin{itemize}
                    \item Effect: Sends pluggable device data to the DB to be stored.
                    \item Exceptions: None
                \end{itemize}
                \item \texttt{List<PluggableDeviceData> getData(int deviceID, DateTime from, DateTime to)}
                \begin{itemize}
                    \item Effect: Returns data from a specific device in a certain time period.
                    \item Exceptions: None
                \end{itemize}
                \item \texttt{List<int> getApplicationsForDevice(int deviceID)}
                \begin{itemize}
                    \item Effect: Returns a list of applications that can use the device with id "deviceID."
                    \item Exceptions: None
                \end{itemize}
            \end{itemize}
        \end{itemize}


\subsection{Data type definitions}
    \paragraph{DateTime} Represents an instant in time, typically expressed as a date and time of day.


\subsection{Verify and refine}
    Completely handled: M1, P2, UC11 \\

    \noindent This section describes per component which (parts of) the remaining
    requirements it is responsible for.

    \paragraph{ApplicationManager}
        \begin{itemize}
            \item  \emph{Av2}: Application failure \\
                   Prevention: a, b \\
                   Detection: a, b, c \\
                   Resolution: a, b, c
           \item \emph{P1}: Large number of users: c
           \item \emph{M2}: Big data analytics on pluggable data and/or application usage data: d, e
           \item \emph{U1}: Application updates: a, b, c, d
           \item \emph{U2}: Easy Installation: e
           \item \emph{U12}: Perform actuation command
           \item \emph{UC17}: Activate an application: 3, 4
        \end{itemize}

    \paragraph{Database}
        \begin{itemize}
          	\item None
        \end{itemize}

    \paragraph{GatewayFacade}
        \begin{itemize}
            \item \emph{Av1}: Communication between SIoTIP gateway and Online Service \\
                               Resolution: b, c, d
            \item \emph{U2}: Easy Installation: a, c, d
        \end{itemize}

    \paragraph{MoteFacade}
        \begin{itemize}
            \item \emph{U2}: Easy Installation: b, c, d
            \item \emph{UC4}: Install mote: 1, 2
            \item \emph{UC5}: Uninstall mote: 1
            \item \emph{UC6}: Insert a pluggable device into a mote: 2
            \item \emph{UC7}: Remove a pluggable device from its mote: 2
        \end{itemize}

    \paragraph{NotificationHandler}
        \begin{itemize}
            \item \emph{UC16}: Consult notification message: 5
            \item \emph{UC17}: Activate an application: 5, 6
        \end{itemize}

    \paragraph{OtherFunctionality}
        \begin{itemize}
            \item \emph{Av1}: Communication between SIoTIP gateway and Online Service \\
                               Detection: a, b, c, d
                               Resolution: a
           	\item \emph{P1}: Large number of users: a
            \item \emph{M2}: Big data analytics on pluggable data and/or application usage data: a
            \item \emph{U2}: Easy Installation: e
            \item \emph{UC1}: Register a customer organisation
            \item \emph{UC2}: Register an end-user
            \item \emph{UC3}: Unregister an end user
            \item \emph{UC4}: Install mote: 3
            \item \emph{UC5}: Uninstall mote: 2.b
            \item \emph{UC6}: Insert a pluggable device into a mote: 3: topology part; alternative 3a.1.b
            \item \emph{UC7}: Remove a pluggable device from its mote: 3.b
            \item \emph{UC8}: Initialise a pluggable device: 1, 2, 4
            \item \emph{UC9}: Configure pluggable device access rights
            \item \emph{UC10}: Consult and configure the topology
            \item \emph{UC13}: Configure pluggable device
            \item \emph{UC16}: Consult notification message: 1, 2, 3, 4
            \item \emph{UC17}: Activate an application: 1, 2
            \item \emph{UC19}: Subscribe to application
            \item \emph{UC20}: Unsubscribe from application
            \item \emph{UC21}: Send invoice
            \item \emph{UC22}: Upload an application
            \item \emph{UC23}: Consult application statistics
            \item \emph{UC24}: Consult historical data
            \item \emph{UC25}: Access topology and available devices
            \item \emph{UC26}: Send application command or message to external front-end
            \item \emph{UC27}: Receive application command or message to external front-end
            \item \emph{UC28}: Log in
            \item \emph{UC29}: Log out
        \end{itemize}

    \paragraph{PluggableDeviceDB}
        \begin{itemize}
            \item \emph{M2}: Big data analytics on pluggable data and/or application usage data: b
        \end{itemize}

    \paragraph{PluggableDeviceFacade}
        \begin{itemize}
        	\item \emph{U2}: Easy Installation: d
        \end{itemize}

    \paragraph{PluggableDeviceManager}
        \begin{itemize}
            \item \emph{U2}: Easy Installation: c, d
            \item \emph{UC4}: Install mote: 4
            \item \emph{UC5}: Uninstall mote: 2
            \item \emph{UC6}: Insert a pluggable device into a mote: 3: uninitialised part; alternative 3a.1 3a.2 3a.4; 4
            \item \emph{UC7}: Remove a pluggable device from its mote: 3.a, 3.c
            \item \emph{UC8}: Initialise a pluggable device: 3,
        \end{itemize}

    \paragraph{PluggableDeviceDataScheduler}
        \begin{itemize}
            \item \emph{P1}: Large number of users: b
            \item \emph{M2}: Big data analytics on pluggable data and/or application usage data: b, c
        \end{itemize}


% Delete the command below to remove the hints and instructions
\showdecisionsnotes{}



\section{ReqX: Requirement Name}
\todoinline{Use this section structure for each requirement}
\subsection*{Key Decisions}
\todoinline{
	Briefly list your key architectural decisions.
	Pay attention to the solutions that you employed (in your own terms or using tactics and/or patterns).}
\begin{itemize}
	\item decision 1
	\item \ldots
\end{itemize}
\emph{Employed tactics and patterns:} \ldots

\subsection*{Rationale}
\todoinline{Describe the design choices related to \emph{ReqX} together with the rationale
	of why these choices where made.}

\subsection*{Considered Alternatives}
\paragraph{Alternative(s) for choice 1} Explain what alternative(s) you
considered for this design choice and why they where not selected.

\subsection*{Deployment Decisions}
\ldots

\subsection*{Considered Deployment Alternatives}
\ldots

\section{Other decisions}
\todoinline{\emph{Optional} If you have made any other important architectural decisions that do not directly fit in the sections of the other qualities you can mention them here.

Follow the same structure as above.}
\subsection{Decision 1}
\subsubsection*{KeyDecisions}
\ldots
\subsubsection*{Rationale}
\ldots
\subsubsection*{Considered Alternatives}
\ldots
\subsubsection*{Deployment Decisions}
\ldots
\subsubsection*{Considered Deployment Alternatives}
\ldots

\section{Discussion}
\todoinline{
	Use this section to discuss your architecture in retrospect.
	For example, what are the strong points of your architecture?
	What are the weak points? Is there anything you would have done otherwise with your current experience?
	Are there any remarks about the architecture that you would give to your customers?
	Etc.
}


\chapter{Client-server view (UML Component diagram)}\label{sec:client-server}
\minilof

% Delete the command below to remove the hints and instructions
\showcsnotes{}

\todoinline{
The context diagram of the client-server view:
Discuss which components communicate with external components and what these external components represent.
}

\begin{figure}[!htp]
	\centering
	%\includegraphics[width=\textwidth]{}
	\missingfigure[figwidth=0.8\textwidth]{Context diagram of the client-server
		view.}
	\caption{Context diagram for the client-server view.
	}\label{fig:cc-context}
\end{figure}

\todoinline{The primary diagram and accompanying explanation.}

\begin{figure}[!htp]
	\centering
	%\includegraphics[width=\textwidth]{}
	\missingfigure[figwidth=0.8\textwidth]{Primary diagram of the client-server
		view.}
	\caption{Primary diagram of the client-server view.}\label{fig:cs-primary}
\end{figure}


\clearpage
\chapter{Decomposition view (UML Component diagram)}\label{sec:decomposition}
\minilof

% Delete the command below to remove the hints and instructions
\showdecompnotes{}

\begin{figure}[!htp]
	\centering
	%\includegraphics[width=\textwidth]{}
	\missingfigure[figwidth=0.8\textwidth]{Diagram showing decomposition of
		ComponentX}
	\caption{Decomposition of \texttt{ComponentX}}\label{fig:decomp-componentx}
\end{figure}

\begin{figure}[!htp]
	\centering
	%\includegraphics[width=\textwidth]{}
	\missingfigure[figwidth=0.8\textwidth]{Diagram showing decomposition of
		ComponentX}
	\caption[Decomposition of \texttt{ComponentY}]{Decomposition of \texttt{ComponentY}.\\
	This caption contains a longer explanation over multiple lines. This additional explanation is not shown in the list of figures.}\label{fig:decomp-componenty}
\end{figure}

%\clearpage \stoplist[decomp]{lof}
\chapter{Deployment view (UML Deployment diagram)}\label{sec:deployment}
\minilof

% Delete the command below to remove the hints and instructions
\showdeploynotes{}

\todoinline{
Describe the context diagram for the deployment view.
For example, which protocols are used for communication with external systems
and why?
}

\begin{figure}[!htp]
	\centering
	%\includegraphics[width=\textwidth]{}
	\missingfigure[figwidth=0.8\textwidth]{Context diagram for the deployment
		view.}
	\caption{Context diagram for the deployment view.}\label{fig:depl_context}
\end{figure}

\todoinline{
The primary deployment diagram itself.
This discussion on the parts of the deployment diagram which are crucial for
achieving certain non-functional requirements, and any alternative deployments that you considered, should be in the architectural decisions chapter.
}

\begin{figure}[!htp]
	\centering
	%\includegraphics[width=\textwidth]{}
	\missingfigure[figwidth=0.8\textwidth]{Primary diagram for the deployment
		view.}
	\caption{Primary diagram for the deployment view.}\label{fig:depl_primary}
\end{figure}

\clearpage
\chapter{Scenarios}\label{sec:scenarios}
\minilof

% Delete the command below to remove the hints and instructions
\showscenariosnotes{}

\todoinline{
	Illustrate how your architecture fulfills the most important data flows. As a rule of thumb, focus on the scenario of the assignment. Describe the scenario in terms of architectural components using UML Sequence diagrams and further explain the most important interactions in text. Illustrating the scenarios serves as a quick validation of the completeness of
	your architecture. If you notice at this point that for some reason, certain functionality or qualities are not addressed sufficiently in your architecture, it suffices to
	document this, together with a rationale of why this is the case according to you. You do not have to further refine you architecture at this point.}


\begin{figure}[!htp]
	\centering
	%\includegraphics[width=\textwidth]{}
	\missingfigure[figwidth=0.8\textwidth]{Sequence diagram scenario 1}
	\caption[Scenario 1]{The system behavior for the first scenario.
	}\label{fig:seq_scenario1}
\end{figure}


\chapter{Element Catalog and Datatypes}
% Delete the command below to remove the hints and instructions
\showcatalognotes{}

\section{Element catalog}\label{app:catalog}
\todoinline{
List all components and describe their responsibilities and provided
interfaces.
Per interface, list all methods using a Java-like syntax and describe their
effect and exceptions if any.
List all elements and interfaces alphabetically for ease of navigation.
}

\componentItem{ComponentZ}{
	\begin{itemize}[noitemsep,nolistsep]
		\item \textbf{Responsibility:} Responsibilities of the component.
		\item \textbf{Super-component:} The direct super-component, if any.
		\item \textbf{Sub-components:} the direct sub-components, if any.
	\end{itemize}
	\subsubsection*{Provided interfaces}
	\begin{itemize}[noitemsep,nolistsep]
		\item InterfaceA
		\begin{itemize}
			\item \texttt{returntType1 operation1(ParamType param) throws SomeException}
			\begin{itemize}
				\item Effect: Describe the effect of the operation
			\end{itemize}
			%
			\item \texttt{void operation2(ParamType2 param)}
			\begin{itemize}
				\item Effect: Describe the effect of the operation
				\item Exceptions: None
			\end{itemize}
		\end{itemize}
		%
		\item InterfaceB
		\begin{itemize}
			\item \texttt{returntType2 operation3()}
			\begin{itemize}
				\item Effect: Describe the effect of the operation
			\end{itemize}
		\end{itemize}
	\end{itemize}
	}

\componentItem{ComponentA}{
	\begin{itemize}[noitemsep,nolistsep]
		\item \textbf{Responsibility:} Responsibilities of the component.
		\item \textbf{Super-component:} The direct super-component, if any.
		\item \textbf{Sub-components:} the direct sub-components, if any.
	\end{itemize}
	\subsubsection*{Provided interfaces}
	\begin{itemize}[noitemsep,nolistsep]
		\item InterfaceC
		\begin{itemize}
			\item \texttt{returntType1 operation1(ParamType param) throws SomeException}
			\begin{itemize}
				\item Effect: Describe the effect of the operation
			\end{itemize}
			%
			\item \texttt{void operation2(ParamType2 param)}
			\begin{itemize}
				\item Effect: Describe the effect of the operation
			\end{itemize}
		\end{itemize}
		%
		\item InterfaceD
		\begin{itemize}[noitemsep,nolistsep]
			\item \texttt{returntType2 operation3()}
			\begin{itemize}
				\item Effect: Describe the effect of the operation
			\end{itemize}
		\end{itemize}
	\end{itemize}
}

% This will alphabetically print the list of components.
\printComponents


\section{Common interfaces}
\todoinline{If you have any common interfaces used by multiple components you may define them here and refer to them.}

\section{Defined Exceptions}
\todoinline{Instead of describing the exceptions with each operation, you may define common exceptions here and refer to them from the operation definition.}

\section{Defined data types}\label{app:datatypes}
\todoinline{
List and describe all data types defined in your interface specifications. List
them alphabetically for ease of navigation.
}

\begin{itemize}
	\item \texttt{Paramtype1}: Description of data type.
	\item \texttt{Paramtype2}: Description of data type.
	\item \texttt{returnType1}: Description of data type.
\end{itemize}

\end{document}
