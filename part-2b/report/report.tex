\documentclass[english]{sareport}
% use the option peerreview for creating an anonymized version of your report
% E.g., \documentclass[english,peerreview]{sareport}

\usepackage[colorlinks, linkcolor=black, citecolor=black, urlcolor=black]{hyperref}
\usepackage{fontawesome}
\usepackage[normalem]{ulem}

\usepackage{graphicx}
\usepackage{rotating}
\usepackage{float}
\usepackage{enumitem}

\usepackage[T1]{fontenc}
\usepackage{lmodern}

\setlist{noitemsep} % \setlist{nosep}
\setlength{\parindent}{0pt}

% Set all authors, if your group counts 2, set third author empty \authorthree{}
% Set the groupname as well
\authorone{Monika Filipcikova (r0683254)}
\authortwo{Armin Halilovic(r0679689)}
\authorthree{}
\groupname{Filipcikova-Halilovic}
\academicyear{2016--2017}

\casename{Shared Internet Of Things Infrastructure Platform}
\phasenumber{2b}
\phasename{The Complete Architecture}

\begin{document}
\maketitle

\tableofcontents
% the following two command are necessary for obtaining the mini list of figures in the cs-view, decomposition view, deployment view and scenarios chapters
\dominilof
\fakelistoffigures

% TODO: before submitting report, remove extra notes from the template

\chapter{Architectural Decisions}\label{ch:overview}
    % \chapter{Architectural Decisions}\label{ch:overview}

% Delete the command below to remove the hints and instructions
\showdecisionsnotes{}

\section{ReqX: Requirement Name}
    \todoinline{Use this section structure for each requirement}
    \subsection*{Key Decisions}
    \todoinline{
    	Briefly list your key architectural decisions.
    	Pay attention to the solutions that you employed (in your own terms or using tactics and/or patterns).}
    \begin{itemize}
    	\item decision 1
    	\item \ldots
    \end{itemize}
    \emph{Employed tactics and patterns:} \ldots

    \subsection*{Rationale}
    \todoinline{Describe the design choices related to \emph{ReqX} together with the rationale
    	of why these choices where made.}

    \subsection*{Considered Alternatives}
    \paragraph{Alternative(s) for choice 1} Explain what alternative(s) you
    considered for this design choice and why they where not selected.

    \subsection*{Deployment Decisions}
    \ldots

    \subsection*{Considered Deployment Alternatives}
    \ldots

\section{Other decisions}
    \todoinline{\emph{Optional} If you have made any other important architectural decisions that do not directly fit in the sections of the other qualities you can mention them here.

    Follow the same structure as above.}
    \subsection{Decision 1}
    \subsubsection*{KeyDecisions}
    \ldots
    \subsubsection*{Rationale}
    \ldots
    \subsubsection*{Considered Alternatives}
    \ldots
    \subsubsection*{Deployment Decisions}
    \ldots
    \subsubsection*{Considered Deployment Alternatives}
    \ldots

\section{Discussion}
    \todoinline{
    	Use this section to discuss your architecture in retrospect.
    	For example, what are the strong points of your architecture?
    	What are the weak points? Is there anything you would have done otherwise with your current experience?
    	Are there any remarks about the architecture that you would give to your customers?
    	Etc.
    }
    Nothing about application state -> maybe could force app developers to add a procedure to save application state so that we could save this and reload that state later.
    Very high coupling for ApplicationManager

    \clearpage

\chapter{Client-server view (UML Component diagram)}\label{ch:client-server}
    \minilof
    % \chapter{Client-server view (UML Component diagram)}\label{ch:client-server}

% Delete the command below to remove the hints and instructions
\showcsnotes{}

\section{Context diagram}
    The context diagram of the client-server view is displayed in figure \ref{fig:cc-context}. \\

    The external components are as follows.
    \begin{itemize}
        \item NotificationDeliveryService: blabla
        \item InfrastructureOwnerClient: blabla
        \item CustomerOrganisationClient: blabla
    \end{itemize}

    \todoinline{
    The context diagram of the client-server view:
    Discuss which components communicate with external components and what these external components represent.
    }


    \begin{landscape}
        \centering
        \vspace*{\fill}

        \begin{figure}[!htp]
            \centering
            \includegraphics[width=\linewidth]{images/component-CONTEXT}
            \caption{Context diagram for the client-server view.}\label{fig:cc-context}
        \end{figure}

        \vfill
    \end{landscape}

\section{Primary diagram}
    The primary diagram of the client-server view is displayed in figure \ref{fig:cc-primary}. \\

    \todoinline{The primary diagram and accompanying explanation.}

    \begin{landscape}
        \centering
        \vspace*{\fill}

        \begin{figure}[!htp]
            \centering
            \includegraphics[width=\linewidth]{images/component-PRIMARY}
            \caption{Primary diagram of the client-server view.}\label{fig:cc-primary}
        \end{figure}

        \vfill
    \end{landscape}

    \clearpage

\chapter{Decomposition view (UML Component diagram)}\label{ch:decomposition}
    \minilof
    %\chapter{Decomposition view (UML Component diagram)}\label{ch:decomposition}

% Delete the command below to remove the hints and instructions
\showdecompnotes{}

\begin{figure}[!htp]
	\centering
	%\includegraphics[width=\textwidth]{}
	\missingfigure[figwidth=0.8\textwidth]{Diagram showing decomposition of ComponentX}
	\caption{Decomposition of \texttt{ComponentX}}\label{fig:decomp-componentx}
\end{figure}

\begin{figure}[!htp]
	\centering
	%\includegraphics[width=\textwidth]{}
	\missingfigure[figwidth=0.8\textwidth]{Diagram showing decomposition of ComponentX}
	\caption[Decomposition of \texttt{ComponentY}]{Decomposition of \texttt{ComponentY}.\\
	This caption contains a longer explanation over multiple lines. This additional explanation is not shown in the list of figures.}\label{fig:decomp-componenty}
\end{figure}

    \clearpage
    % \stoplist[decomp]{lof}

\chapter{Deployment view (UML Deployment diagram)}\label{ch:deployment}
    \minilof
    % \chapter{Deployment view (UML Deployment diagram)}\label{ch:deployment}


\begin{landscape}
    \section{Context diagram}
    The context diagram for the deployment view is displayed in figure \ref{fig:depl_context}. \\

    \centering
    \vspace*{\fill}

        \begin{figure}[!htp]
        	\centering
            \includegraphics[width=\textwidth]{images/deployment-context}
            \caption{Context diagram for the deployment view.}\label{fig:depl_context}
        \end{figure}

    \vfill
\end{landscape}


\begin{landscape}
    \section{Primary diagram}
    The primary diagram for the deployment view is displayed in figure \ref{fig:depl_primary}.

    \centering
    \vspace*{\fill}

        \begin{figure}[!htp]
        	\centering
            \includegraphics[width=\textwidth]{images/deployment-primary}
        	\caption{Primary diagram for the deployment view.}\label{fig:depl_primary}
        \end{figure}

    \vfill
\end{landscape}

    \clearpage

\chapter{Scenarios}\label{ch:scenarios}
    \minilof
    % \chapter{Scenarios}\label{ch:scenarios}

% Delete the command below to remove the hints and instructions

\section{Scenarios}
    This section lists which sequence diagrams belong to which scenarios:
    \begin{itemize}
        \item UC11: Sensor data being processed by the system \\
              Figure \ref{fig:seq_scenario1}

        \item UC19: Subscribing to an application \\
              Figure \ref{fig:seq_scenario2}

        \item UC12: Applications issuing actuation commands \\
              Figure \ref{fig:seq_scenario31}
              Figure \ref{fig:seq_scenario32}
              Figure \ref{fig:seq_scenario33}

        \item UC14, Av3, UC18: Sensors/actuators failing \\
              Figure \ref{fig:seq_scenario4} \\

        \item Av2: Application crash \\
              Figure \ref{fig:seq_scenario5}

        \item U2, UC6: Plugging in a new pluggable device (sensor or actuator) \\
              Figure \ref{fig:seq_scenario6}

        \item UC4: Install mote \\
              Figure \ref{fig:seq_scenario666}

        \item UC22, U1: Upgrading an application \\
              Figure \ref{fig:seq_scenario8}

        \item UC26, UC27, UC12: Sending actuation commands via a mobile app \\
              Figure \ref{fig:seq_scenario9}
    \end{itemize}

    \begin{figure}[!htp]
    	\centering
    	\includegraphics[width=\textwidth]{images/sequence-UC11}
    	\caption[Sensor data being processed by the system]{The pluggable device sends data to the system. The gateway receives the data and forwards it to the Online Service.
    	 The Data are saved in the Online Service in the PluggableDeviceDataDB. }\label{fig:seq_scenario1}
    \end{figure}

    \begin{figure}[!htp]
    	\centering
    	\includegraphics[width=\textwidth]{images/sequence-UC19}
    	\caption[ Subscribing to an application]{A customer organisation actor wants to subscribe to an application. The system provides him applications to subscription
    	                        The customer organisation chooses application, sets application devices settings and user roles }\label{fig:seq_scenario2}
    \end{figure}

    \begin{figure}[!htp]
    	\centering
    	\includegraphics[width=\textwidth]{images/sequence-UC12-commandFromFrontEnd}
    	\caption[Applications issuing actuation commands]{An application sends an actuation command to the one or more pluggable devices. The actuation command is
    	 created according to the specific formatting syntax for the each device. }\label{fig:seq_scenario31}
    \end{figure}

    \begin{figure}[!htp]
    	\centering
    	\includegraphics[width=\textwidth]{images/sequence-UC12-commandFromGW}
    	\caption[Applications issuing actuation commands]{An application sends an actuation command to the one or more pluggable devices. The actuation command is
    	 created according to the specific formatting syntax for the each device. }\label{fig:seq_scenario32}
    \end{figure}

    \begin{figure}[!htp]
    	\centering
    	\includegraphics[width=\textwidth]{images/sequence-UC12-commandFromOS}
    	\caption[Applications issuing actuation commands]{An application sends an actuation command to the one or more pluggable devices. The actuation command is
    	 created according to the specific formatting syntax for the each device. }\label{fig:seq_scenario33}
    \end{figure}

    \begin{figure}[!htp]
    	\centering
    	\includegraphics[width=\textwidth]{images/sequence-Av3-UC14-UC18}
    	\caption[Scenario ]{This is the flow of pluggable device failure. If there is not redundant pluggable device, then the application is deactivated.}\label{fig:seq_scenario4}
    \end{figure}

    \begin{figure}[!htp]
    	\centering
    	\includegraphics[width=\textwidth]{images/sequence-Av2}
    	\caption[Application crash]{ In the flow is shown failure of an application. The important part of this diagram is monitoring system, that detect failure of the application. }\label{fig:seq_scenario5}
    \end{figure}

    \begin{figure}[!htp]
    	\centering
    	\includegraphics[width=\textwidth]{images/sequence-U2-UC4}
    	\caption[Plugging in a new pluggable device (sensor or actuator)]{ The new pluggable device is detected and saved as 'inactive'. In case Pluggable device is known previously, the SloTIP reactivated the pluggable device automatically.}\label{fig:seq_scenario6}
    \end{figure}

    \begin{figure}[h]
    	\centering
    	\includegraphics[width=\textwidth]{images/sequence-U1-UC22}
    	\caption[Upgrading an application]{ The application provider can upload new application or update existing.}\label{fig:seq_scenario8}
    \end{figure}

    \begin{figure}[!htp]
    	\centering
    	\includegraphics[width=\textwidth]{images/sequence-UC12-UC26-UC27}
    	\caption[Sending actuation commands via a mobile app]{ The application sends command from external frontend to the other parts of application. }\label{fig:seq_scenario9}
    \end{figure}

    \begin{figure}[!htp]
    	\centering
    	\includegraphics[width=\textwidth]{images/sequence-UC04}
    	\caption[Install mote]{The scenario for installing a mote}\ref{fig:seq_scenario666}
    \end{figure}


% TODO: before submitting report, replace the chapter line in "exported_catalog.tex" by this line and generate the report 2 times to set all references
% TODO: before submitting report, add \newpage before \section{Interfaces} in "exported_catalog.tex"
\chapter{Element Catalog and Datatypes}\label{ch:elements-datatypes}
    % TODO: before submitting report, move this into exported_catalog
    Each method contains a short note on why the method was added (under "Created for").
    This was done to keep track of our decisions and does not mean that the methods can only
    be used for the Quality Attribute/Use Case referenced in the "Created for" note.
%%% element catalog, generated on Tue May 02 10:17:09 CEST 2017

%%%%
%% The following is a minimal header that allows you to compile a 'standalone' version of the catalog.
%% Uncomment these lines and dont forget to uncomment the \end{document} at the bottom.
%%%%
%% START MINIMAL HEADER
%\documentclass[a4paper,10pt]{report}
%% this package is not strictly needed
%\usepackage[top=2cm,bottom=2cm,left=2cm,right=2cm]{geometry}
%% the following packages are necessary
%\usepackage{enumitem}
%\usepackage{tikz}
%\usepackage{nameref}
%\usepackage{hyperref}
%
%\begin{document}
%% END MINIMAL HEADER


% EXPORT CMDS
% \texttt layout modifications

\newcommand*\vpejustify{%
	\fontdimen2\font=0.4em%
	\fontdimen3\font=0.8em%
	\fontdimen4\font=0.1em%
	\fontdimen7\font=1.0em%
	\hyphenchar\font=`\-\relax%
}
\newcommand{\vpett}[1]{\vpejustify{\texttt{#1}}}

% item labels
\makeatletter
\def\vpeitemlabel#1#2{\begingroup
	#2%
	\def\@currentlabel{#2}%
	\phantomsection\label{#1}\endgroup
}
\makeatother

% vpe operation
\newcommand{\vpeoperation}[1]{#1}

% vpe data type
\newcommand{\vpedatatype}[2]{\vpeitemlabel{#1}{\textbf{\textsf{#2}}}}

% vpe exception
\newcommand{\vpeexception}[2]{\vpeitemlabel{#1}{\textsl{#2}}}



\newcommand{\iconcomponent}{%
	\begin{tikzpicture}[scale=0.3,thin,baseline=-0.5ex]
	\draw (0,0) rectangle (0.4, 0.6);
	\draw [fill=white] (-0.1,0.40) rectangle +(0.25, 0.1);
	\draw [fill=white] (-0.1,0.22) rectangle +(0.25, 0.1);	
	\end{tikzpicture}%
}
\newcommand{\iconprovided}{%
	\begin{tikzpicture}[scale=0.25,thin,baseline=-0.5ex]
	\draw (0.60,0.25) circle [radius=0.25];
	\draw (0, 0.25) -- (0.35, 0.25);
	\end{tikzpicture}%
}
\newcommand{\iconrequired}{%
	\begin{tikzpicture}[scale=0.25,thin,baseline=-0.5ex]
	\draw (0.60,0) arc (270:90:0.25);
	\draw (0, 0.25) -- (0.35, 0.25);
	\end{tikzpicture}%
}
%  END EXPORT CMDS

\chapter{Catalog}
% COMPONENTS
\section{Components}\label{sec:components}
\subsection{AccessRightsManager}\label{comp:OnlineServiceOnlineServiceAccessRightsManager}
	\begin{description}[noitemsep,nolistsep]
		\item[Responsibility:]~Responsible for all functionality related to access rights to pluggable devices.
E.g. retrieving the access rights a customer organisation has for a device,
updating access rights for customer organisations, etc.
		\item[Super-components:]~None
		\item[Sub-components:]~None
		\item[Provided interfaces:]~\iconprovided{}~\vpett{\nameref{int:OnlineServiceOnlineServiceAccessRightsManagerAccessRightsMgmt}}
		\item[Required interfaces:]~\iconrequired{}~\vpett{\nameref{int:DeviceDatabaseDeviceDBAccessRightsMgmt}}		
	\end{description}
\subsection{ActuationCommandConstructor}\label{comp:OnlineServiceOnlineServiceApplicationManagerActuationCommandConstructor}
	\begin{description}[noitemsep,nolistsep]
		\item[Responsibility:]~{\colorbox{red!30}{\underline{Undefined}}}
		\item[Super-components:]~\iconcomponent{}~\vpett{\nameref{comp:OnlineServiceOnlineServiceApplicationManager}}
		\item[Sub-components:]~None
		\item[Provided interfaces:]~\iconprovided{}~\vpett{\nameref{int:OnlineServiceOnlineServiceApplicationManagerActuation}}
		\item[Required interfaces:]~\iconrequired{}~\vpett{\nameref{int:GatewayGatewayGatewayFacadeAppDeviceMgmt}}		
	\end{description}
\subsection{ApplicationClient}\label{comp:ApplicationClient}
	\begin{description}[noitemsep,nolistsep]
		\item[Responsibility:]~{\colorbox{red!30}{\underline{Undefined}}}
		\item[Super-components:]~None
		\item[Sub-components:]~None
		\item[Provided interfaces:]~None
		\item[Required interfaces:]~\iconrequired{}~\vpett{\nameref{int:OnlineServiceOnlineServiceApplicationFacadeAppData}}, \iconrequired{}~\vpett{\nameref{int:OnlineServiceOnlineServiceApplicationFacadeDeviceMgmt}}, \iconrequired{}~\vpett{\nameref{int:OnlineServiceOnlineServiceApplicationFacadeTopologyOverview}}		
	\end{description}
\subsection{ApplicationContainer}\label{comp:OnlineServiceOnlineServiceApplicationManagerApplicationContainer}
	\begin{description}[noitemsep,nolistsep]
		\item[Responsibility:]~{\colorbox{red!30}{\underline{Undefined}}}
		\item[Super-components:]~\iconcomponent{}~\vpett{\nameref{comp:OnlineServiceOnlineServiceApplicationManager}}
		\item[Sub-components:]~None
		\item[Provided interfaces:]~\iconprovided{}~\vpett{\nameref{int:OnlineServiceOnlineServiceApplicationManagerAppInstanceMgmt}}
		\item[Required interfaces:]~None		
	\end{description}
\subsection{ApplicationContainerManager}\label{comp:OnlineServiceOnlineServiceApplicationManagerApplicationContainerManager}
	\begin{description}[noitemsep,nolistsep]
		\item[Responsibility:]~This component contains sanbox environments for application instances to execute in.
		\item[Super-components:]~\iconcomponent{}~\vpett{\nameref{comp:OnlineServiceOnlineServiceApplicationManager}}
		\item[Sub-components:]~None
		\item[Provided interfaces:]~None
		\item[Required interfaces:]~\iconrequired{}~\vpett{\nameref{int:OnlineServiceOnlineServiceApplicationManagerAppInstanceMgmt}}		
	\end{description}
\subsection{ApplicationContainerMonitor}\label{comp:OnlineServiceOnlineServiceApplicationManagerApplicationContainerMonitor}
	\begin{description}[noitemsep,nolistsep]
		\item[Responsibility:]~{\colorbox{red!30}{\underline{Undefined}}}
		\item[Super-components:]~\iconcomponent{}~\vpett{\nameref{comp:OnlineServiceOnlineServiceApplicationManager}}
		\item[Sub-components:]~None
		\item[Provided interfaces:]~None
		\item[Required interfaces:]~\iconrequired{}~\vpett{\nameref{int:OnlineServiceOnlineServiceApplicationManagerAppInstanceMgmt}}		
	\end{description}
\subsection{ApplicationFacade}\label{comp:OnlineServiceOnlineServiceApplicationFacade}
	\begin{description}[noitemsep,nolistsep]
		\item[Responsibility:]~{\colorbox{red!30}{\underline{Undefined}}}
		\item[Super-components:]~None
		\item[Sub-components:]~None
		\item[Provided interfaces:]~\iconprovided{}~\vpett{\nameref{int:OnlineServiceOnlineServiceApplicationFacadeAppData}}, \iconprovided{}~\vpett{\nameref{int:OnlineServiceOnlineServiceApplicationFacadeDeviceMgmt}}, \iconprovided{}~\vpett{\nameref{int:OnlineServiceOnlineServiceApplicationFacadeTopologyOverview}}
		\item[Required interfaces:]~\iconrequired{}~\vpett{\nameref{int:OnlineServiceOnlineServiceApplicationManagerActuation}}, \iconrequired{}~\vpett{\nameref{int:OnlineServiceOnlineServiceApplicationManagerApps}}		
	\end{description}
\subsection{ApplicationManager}\label{comp:OnlineServiceOnlineServiceApplicationManager}
	\begin{description}[noitemsep,nolistsep]
		\item[Responsibility:]~Responsible for activating/deactivating applications, setting pluggable device redundancy requirements on \vpett{\nameref{comp:GatewayGatewayDeviceManager}} components, and using \vpett{\nameref{comp:OnlineServiceOnlineServiceNotificationHandler}} to send notifications to customer organisations.
		\item[Super-components:]~None
		\item[Sub-components:]~\iconcomponent{}~\vpett{\nameref{comp:OnlineServiceOnlineServiceApplicationManagerActuationCommandConstructor}}, \iconcomponent{}~\vpett{\nameref{comp:OnlineServiceOnlineServiceApplicationManagerApplicationContainerMonitor}}, \iconcomponent{}~\vpett{\nameref{comp:OnlineServiceOnlineServiceApplicationManagerApplicationContainer}}, \iconcomponent{}~\vpett{\nameref{comp:OnlineServiceOnlineServiceApplicationManagerApplicationContainerManager}}
		\item[Provided interfaces:]~\iconprovided{}~\vpett{\nameref{int:OnlineServiceOnlineServiceApplicationManagerActuation}}, \iconprovided{}~\vpett{\nameref{int:OnlineServiceOnlineServiceApplicationManagerAppMgmt}}, \iconprovided{}~\vpett{\nameref{int:OnlineServiceOnlineServiceApplicationManagerApps}}, \iconprovided{}~\vpett{\nameref{int:OnlineServiceOnlineServiceApplicationManagerForwardData}}, \iconprovided{}~\vpett{\nameref{int:OnlineServiceOnlineServiceApplicationManagerIOAppMgmt}}
		\item[Required interfaces:]~\iconrequired{}~\vpett{\nameref{int:GatewayGatewayGatewayFacadeAppDeviceMgmt}}, \iconrequired{}~\vpett{\nameref{int:OtherDataDatabaseOtherDataDBAppMgmt}}, \iconrequired{}~\vpett{\nameref{int:GatewayGatewayGatewayFacadeAppMgmt}}, \iconrequired{}~\vpett{\nameref{int:OnlineServiceOnlineServiceInvoiceManagerInvoiceMgmt}}, \iconrequired{}~\vpett{\nameref{int:OnlineServiceOnlineServiceNotificationHandlerNotify}}, \iconrequired{}~\vpett{\nameref{int:OnlineServiceOnlineServiceDeviceDataSchedulerRequestData}}, \iconrequired{}~\vpett{\nameref{int:OnlineServiceOnlineServiceUserRolesManagerRoleMgmt}}, \iconrequired{}~\vpett{\nameref{int:OnlineServiceOnlineServiceTopologyManagerTopologyMgmt}}		
	\end{description}
\subsection{CustomerOgranisationClient}\label{comp:CustomerOgranisationClient}
	\begin{description}[noitemsep,nolistsep]
		\item[Responsibility:]~Represents the client used by a customer organisation. This is the user's dashboard.
		\item[Super-components:]~None
		\item[Sub-components:]~None
		\item[Provided interfaces:]~None
		\item[Required interfaces:]~\iconrequired{}~\vpett{\nameref{int:OnlineServiceOnlineServiceCustomerOrganisationFacadeSubscriptionMgmt}}		
	\end{description}
\subsection{CustomerOrganisationFacade}\label{comp:OnlineServiceOnlineServiceCustomerOrganisationFacade}
	\begin{description}[noitemsep,nolistsep]
		\item[Responsibility:]~Acts as an access point for CustomerOrganisationClients and handles all functionality that can be done by customer organisations.
		\item[Super-components:]~None
		\item[Sub-components:]~None
		\item[Provided interfaces:]~\iconprovided{}~\vpett{\nameref{int:OnlineServiceOnlineServiceCustomerOrganisationFacadeSubscriptionMgmt}}
		\item[Required interfaces:]~\iconrequired{}~\vpett{\nameref{int:OnlineServiceOnlineServiceApplicationManagerApps}}, \iconrequired{}~\vpett{\nameref{int:OnlineServiceOnlineServiceUserRolesManagerRoleMgmt}}, \iconrequired{}~\vpett{\nameref{int:OnlineServiceOnlineServiceSubscriptionManagerSubscriptionMgmt}}, \iconrequired{}~\vpett{\nameref{int:OnlineServiceOnlineServiceTopologyManagerTopologyMgmt}}		
	\end{description}
\subsection{DeviceDataConverter}\label{comp:OnlineServiceOnlineServiceDeviceDataConverter}
	\begin{description}[noitemsep,nolistsep]
		\item[Responsibility:]~The \vpett{\nameref{comp:OnlineServiceOnlineServiceDeviceDataConverter}} is resposible for converting pluggable device data in the data processing subsystem.
		\item[Super-components:]~None
		\item[Sub-components:]~None
		\item[Provided interfaces:]~\iconprovided{}~\vpett{\nameref{int:OnlineServiceOnlineServiceDeviceDataConverterDataConversion}}
		\item[Required interfaces:]~None		
	\end{description}
\subsection{DeviceDataScheduler}\label{comp:OnlineServiceOnlineServiceDeviceDataScheduler}
	\begin{description}[noitemsep,nolistsep]
		\item[Responsibility:]~Responsible for scheduling incoming read and write requests for pluggable device data. Monitors throughput of requests and switches between normal and overload mode when appropriate. Avoids starvation of any type of request.
		\item[Super-components:]~None
		\item[Sub-components:]~None
		\item[Provided interfaces:]~\iconprovided{}~\vpett{\nameref{int:OnlineServiceOnlineServiceDeviceDataSchedulerDeviceData}}, \iconprovided{}~\vpett{\nameref{int:OnlineServiceOnlineServiceDeviceDataSchedulerRequestData}}
		\item[Required interfaces:]~\iconrequired{}~\vpett{\nameref{int:PluggableDeviceDatabasePluggableDeviceDataDBDeviceData}}, \iconrequired{}~\vpett{\nameref{int:OnlineServiceOnlineServiceApplicationManagerForwardData}}		
	\end{description}
\subsection{DeviceDB}\label{comp:DeviceDatabaseDeviceDB}
	\begin{description}[noitemsep,nolistsep]
		\item[Responsibility:]~Contains all information related to devices in the system, but not pluggable device data such as sensor data or actuation statuses. The data includes information about pluggable devices, motes, gateways, topologies, access rights, etc.
		\item[Super-components:]~None
		\item[Sub-components:]~None
		\item[Provided interfaces:]~\iconprovided{}~\vpett{\nameref{int:DeviceDatabaseDeviceDBAccessRightsMgmt}}, \iconprovided{}~\vpett{\nameref{int:DeviceDatabaseDeviceDBDeviceMgmt}}, \iconprovided{}~\vpett{\nameref{int:DeviceDatabaseDeviceDBIODeviceMgmt}}, \iconprovided{}~\vpett{\nameref{int:DeviceDatabaseDeviceDBTopologyMgmt}}
		\item[Required interfaces:]~None		
	\end{description}
\subsection{DeviceManager}\label{comp:GatewayGatewayDeviceManager}
	\begin{description}[noitemsep,nolistsep]
		\item[Responsibility:]~Monitors connected/operational devices on a gateway. Sends notifications in case of hardware failure. Can send a command to disable or reactivate applications when necessary.
		\item[Super-components:]~None
		\item[Sub-components:]~None
		\item[Provided interfaces:]~\iconprovided{}~\vpett{\nameref{int:GatewayGatewayDeviceManagerDeviceMgmt}}
		\item[Required interfaces:]~\iconrequired{}~\vpett{\nameref{int:GatewayGatewayGatewayFacadeDeviceMgmt}}		
	\end{description}
\subsection{GatewayFacade}\label{comp:GatewayGatewayGatewayFacade}
	\begin{description}[noitemsep,nolistsep]
		\item[Responsibility:]~Main component on the gateway that allows different components to work
with each other. E.g. transmits heartbeats from motes to
\vpett{\nameref{comp:GatewayGatewayDeviceManager}}, transmits commands to shut down applications,
triggers notifications to be generated, ...
		\item[Super-components:]~None
		\item[Sub-components:]~None
		\item[Provided interfaces:]~\iconprovided{}~\vpett{\nameref{int:GatewayGatewayGatewayFacadeAppDeviceMgmt}}, \iconprovided{}~\vpett{\nameref{int:GatewayGatewayGatewayFacadeAppMgmt}}, \iconprovided{}~\vpett{\nameref{int:GatewayGatewayGatewayFacadeDeviceData}}, \iconprovided{}~\vpett{\nameref{int:GatewayGatewayGatewayFacadeDeviceMgmt}}, \iconprovided{}~\vpett{\nameref{int:GatewayGatewayGatewayFacadeHeartbeat}}
		\item[Required interfaces:]~\iconrequired{}~\vpett{\nameref{int:OnlineServiceOnlineServiceApplicationManagerAppMgmt}}, \iconrequired{}~\vpett{\nameref{int:GatewayGatewayGWApplicationContainerManagerAppMgmt}}, \iconrequired{}~\vpett{\nameref{int:OnlineServiceOnlineServiceDeviceDataConverterDataConversion}}, \iconrequired{}~\vpett{\nameref{int:OnlineServiceOnlineServiceDeviceDataSchedulerDeviceData}}, \iconrequired{}~\vpett{\nameref{int:MoteMoteFacadeDeviceMgmt}}, \iconrequired{}~\vpett{\nameref{int:GatewayGatewayDeviceManagerDeviceMgmt}}, \iconrequired{}~\vpett{\nameref{int:DeviceDatabaseDeviceDBDeviceMgmt}}, \iconrequired{}~\vpett{\nameref{int:OnlineServiceOnlineServiceNotificationHandlerNotify}}, \iconrequired{}~\vpett{\nameref{int:OnlineServiceOnlineServiceOtherFunctionality2Other}}, \iconrequired{}~\vpett{\nameref{int:OnlineServiceOnlineServiceTopologyManagerTopologyMgmt}}		
	\end{description}
\subsection{GWApplicationContainerManager}\label{comp:GatewayGatewayGWApplicationContainerManager}
	\begin{description}[noitemsep,nolistsep]
		\item[Responsibility:]~This component contains sanbox environments for application instances to execute in.
		\item[Super-components:]~None
		\item[Sub-components:]~None
		\item[Provided interfaces:]~\iconprovided{}~\vpett{\nameref{int:GatewayGatewayGWApplicationContainerManagerAppMgmt}}
		\item[Required interfaces:]~None		
	\end{description}
\subsection{InfrastructreOwnerClient}\label{comp:InfrastructreOwnerClient}
	\begin{description}[noitemsep,nolistsep]
		\item[Responsibility:]~Represents the client used by an infrastructure owner. This is the user's dashboard.
		\item[Super-components:]~None
		\item[Sub-components:]~None
		\item[Provided interfaces:]~None
		\item[Required interfaces:]~\iconrequired{}~\vpett{\nameref{int:OnlineServiceOnlineServiceInfrastructureOwnerFacadeAccessRights}}		
	\end{description}
\subsection{InfrastructureOwnerFacade}\label{comp:OnlineServiceOnlineServiceInfrastructureOwnerFacade}
	\begin{description}[noitemsep,nolistsep]
		\item[Responsibility:]~Acts as an access point for InfrastructureOwnerClients and handles all functionality that can be done by infrastructure owners.
		\item[Super-components:]~None
		\item[Sub-components:]~None
		\item[Provided interfaces:]~\iconprovided{}~\vpett{\nameref{int:OnlineServiceOnlineServiceInfrastructureOwnerFacadeAccessRights}}
		\item[Required interfaces:]~\iconrequired{}~\vpett{\nameref{int:OnlineServiceOnlineServiceAccessRightsManagerAccessRightsMgmt}}, \iconrequired{}~\vpett{\nameref{int:OnlineServiceOnlineServiceApplicationManagerIOAppMgmt}}, \iconrequired{}~\vpett{\nameref{int:OnlineServiceOnlineServiceInfrastructureOwnerManagerIOMgmt}}		
	\end{description}
\subsection{InfrastructureOwnerManager}\label{comp:OnlineServiceOnlineServiceInfrastructureOwnerManager}
	\begin{description}[noitemsep,nolistsep]
		\item[Responsibility:]~Responsible for all functionality related to infrastructure owners. E.g. looking up the devices they own, retrieving a list of customer organisations that they are associated to, etc.
		\item[Super-components:]~None
		\item[Sub-components:]~None
		\item[Provided interfaces:]~\iconprovided{}~\vpett{\nameref{int:OnlineServiceOnlineServiceInfrastructureOwnerManagerIOMgmt}}
		\item[Required interfaces:]~\iconrequired{}~\vpett{\nameref{int:DeviceDatabaseDeviceDBIODeviceMgmt}}, \iconrequired{}~\vpett{\nameref{int:OtherDataDatabaseOtherDataDBIOMgmt}}		
	\end{description}
\subsection{InvoiceManager}\label{comp:OnlineServiceOnlineServiceInvoiceManager}
	\begin{description}[noitemsep,nolistsep]
		\item[Responsibility:]~Responsible for all functionality related to access rights to invoicing.
E.g. creating invoices.
		\item[Super-components:]~None
		\item[Sub-components:]~None
		\item[Provided interfaces:]~\iconprovided{}~\vpett{\nameref{int:OnlineServiceOnlineServiceInvoiceManagerInvoiceMgmt}}
		\item[Required interfaces:]~\iconrequired{}~\vpett{\nameref{int:OtherDataDatabaseOtherDataDBInvoiceMgmt}}		
	\end{description}
\subsection{MoteFacade}\label{comp:MoteMoteFacade}
	\begin{description}[noitemsep,nolistsep]
		\item[Responsibility:]~Sends heartbeats to the \vpett{\nameref{comp:GatewayGatewayGatewayFacade}}. Includes a list
of connected pluggable devices in the heartbeats.
		\item[Super-components:]~None
		\item[Sub-components:]~None
		\item[Provided interfaces:]~\iconprovided{}~\vpett{\nameref{int:MoteMoteFacadeDeviceData}}, \iconprovided{}~\vpett{\nameref{int:MoteMoteFacadeDeviceMgmt}}
		\item[Required interfaces:]~\iconrequired{}~\vpett{\nameref{int:PuggableDevicePluggableDeviceFacadeActuate}}, \iconrequired{}~\vpett{\nameref{int:PuggableDevicePluggableDeviceFacadeConfig}}, \iconrequired{}~\vpett{\nameref{int:GatewayGatewayGatewayFacadeDeviceData}}, \iconrequired{}~\vpett{\nameref{int:GatewayGatewayGatewayFacadeHeartbeat}}, \iconrequired{}~\vpett{\nameref{int:PuggableDevicePluggableDeviceFacadeRequestData}}		
	\end{description}
\subsection{NotificationDeliveryService}\label{comp:NotificationDeliveryServiceNodeNotificationDeliveryService}
	\begin{description}[noitemsep,nolistsep]
		\item[Responsibility:]~{\colorbox{red!30}{\underline{Undefined}}}
		\item[Super-components:]~None
		\item[Sub-components:]~None
		\item[Provided interfaces:]~None
		\item[Required interfaces:]~None		
	\end{description}
\subsection{NotificationHandler}\label{comp:OnlineServiceOnlineServiceNotificationHandler}
	\begin{description}[noitemsep,nolistsep]
		\item[Responsibility:]~Responsible for generation, storage, and delivery of notifications based on users' preferred communication channel.
		\item[Super-components:]~None
		\item[Sub-components:]~None
		\item[Provided interfaces:]~\iconprovided{}~\vpett{\nameref{int:OnlineServiceOnlineServiceNotificationHandlerDeliveryMgmt}}, \iconprovided{}~\vpett{\nameref{int:OnlineServiceOnlineServiceNotificationHandlerNotify}}
		\item[Required interfaces:]~\iconrequired{}~\vpett{\nameref{int:NotificationDeliveryServiceNodeNotificationDeliveryServiceNotificationDeliveryMgmt}}, \iconrequired{}~\vpett{\nameref{int:OtherDataDatabaseOtherDataDBNotificationMgmt}}		
	\end{description}
\subsection{OtherDataDB}\label{comp:OtherDataDatabaseOtherDataDB}
	\begin{description}[noitemsep,nolistsep]
		\item[Responsibility:]~General database for data. For example, storage of data about notifications.
		\item[Super-components:]~None
		\item[Sub-components:]~None
		\item[Provided interfaces:]~\iconprovided{}~\vpett{\nameref{int:OtherDataDatabaseOtherDataDBAppMgmt}}, \iconprovided{}~\vpett{\nameref{int:OtherDataDatabaseOtherDataDBInvoiceMgmt}}, \iconprovided{}~\vpett{\nameref{int:OtherDataDatabaseOtherDataDBIOMgmt}}, \iconprovided{}~\vpett{\nameref{int:OtherDataDatabaseOtherDataDBNotificationMgmt}}, \iconprovided{}~\vpett{\nameref{int:OtherDataDatabaseOtherDataDBOther}}, \iconprovided{}~\vpett{\nameref{int:OtherDataDatabaseOtherDataDBSubscriptionMgmt}}, \iconprovided{}~\vpett{\nameref{int:OtherDataDatabaseOtherDataDBUserRoleMgmt}}
		\item[Required interfaces:]~None		
	\end{description}
\subsection{OtherFunctionality1}\label{comp:OnlineServiceOnlineServiceOtherFunctionality1}
	\begin{description}[noitemsep,nolistsep]
		\item[Responsibility:]~{\colorbox{red!30}{\underline{Undefined}}}
		\item[Super-components:]~None
		\item[Sub-components:]~None
		\item[Provided interfaces:]~\iconprovided{}~\vpett{\nameref{int:OnlineServiceOnlineServiceOtherFunctionality2Other}}
		\item[Required interfaces:]~\iconrequired{}~\vpett{\nameref{int:OtherDataDatabaseOtherDataDBOther}}, \iconrequired{}~\vpett{\nameref{int:PluggableDeviceDatabasePluggableDeviceDataDBOther}}		
	\end{description}
\subsection{OtherFunctionality2}\label{comp:OnlineServiceOnlineServiceOtherFunctionality2}
	\begin{description}[noitemsep,nolistsep]
		\item[Responsibility:]~{\colorbox{red!30}{\underline{Undefined}}}
		\item[Super-components:]~None
		\item[Sub-components:]~None
		\item[Provided interfaces:]~\iconprovided{}~\vpett{\nameref{int:OnlineServiceOnlineServiceOtherFunctionality2Other}}
		\item[Required interfaces:]~\iconrequired{}~\vpett{\nameref{int:OtherDataDatabaseOtherDataDBOther}}, \iconrequired{}~\vpett{\nameref{int:PluggableDeviceDatabasePluggableDeviceDataDBOther}}		
	\end{description}
\subsection{PluggableDeviceDataDB}\label{comp:PluggableDeviceDatabasePluggableDeviceDataDB}
	\begin{description}[noitemsep,nolistsep]
		\item[Responsibility:]~Database dedicated to pluggable device data only.
		\item[Super-components:]~None
		\item[Sub-components:]~None
		\item[Provided interfaces:]~\iconprovided{}~\vpett{\nameref{int:PluggableDeviceDatabasePluggableDeviceDataDBDeviceData}}, \iconprovided{}~\vpett{\nameref{int:PluggableDeviceDatabasePluggableDeviceDataDBOther}}
		\item[Required interfaces:]~None		
	\end{description}
\subsection{PluggableDeviceFacade}\label{comp:PuggableDevicePluggableDeviceFacade}
	\begin{description}[noitemsep,nolistsep]
		\item[Responsibility:]~Responsible for sending pluggable device data to \vpett{\nameref{comp:MoteMoteFacade}}. Needs to be initialised in order for the data to be used/stored.
		\item[Super-components:]~None
		\item[Sub-components:]~None
		\item[Provided interfaces:]~\iconprovided{}~\vpett{\nameref{int:PuggableDevicePluggableDeviceFacadeActuate}}, \iconprovided{}~\vpett{\nameref{int:PuggableDevicePluggableDeviceFacadeConfig}}, \iconprovided{}~\vpett{\nameref{int:PuggableDevicePluggableDeviceFacadeRequestData}}
		\item[Required interfaces:]~\iconrequired{}~\vpett{\nameref{int:MoteMoteFacadeDeviceData}}		
	\end{description}
\subsection{SubscriptionManager}\label{comp:OnlineServiceOnlineServiceSubscriptionManager}
	\begin{description}[noitemsep,nolistsep]
		\item[Responsibility:]~Responsible for all functionality related to access rights to subscriptions. E.g. retrieving the applications that a customer organisation can subscribe to, creating new subscriptions to ApplicationInstances, etc.
		\item[Super-components:]~None
		\item[Sub-components:]~None
		\item[Provided interfaces:]~\iconprovided{}~\vpett{\nameref{int:OnlineServiceOnlineServiceSubscriptionManagerSubscriptionMgmt}}
		\item[Required interfaces:]~\iconrequired{}~\vpett{\nameref{int:OnlineServiceOnlineServiceApplicationManagerApps}}, \iconrequired{}~\vpett{\nameref{int:OtherDataDatabaseOtherDataDBSubscriptionMgmt}}		
	\end{description}
\subsection{TopologyManager}\label{comp:OnlineServiceOnlineServiceTopologyManager}
	\begin{description}[noitemsep,nolistsep]
		\item[Responsibility:]~Responsible for all functionality related to topology. E.g. Adding a new mote to the topology of an infrastructure, checking whether or not all devices used by an application are active in the topology, etc.
		\item[Super-components:]~None
		\item[Sub-components:]~None
		\item[Provided interfaces:]~\iconprovided{}~\vpett{\nameref{int:OnlineServiceOnlineServiceTopologyManagerTopologyMgmt}}
		\item[Required interfaces:]~\iconrequired{}~\vpett{\nameref{int:DeviceDatabaseDeviceDBTopologyMgmt}}		
	\end{description}
\subsection{UserRolesManager}\label{comp:OnlineServiceOnlineServiceUserRolesManager}
	\begin{description}[noitemsep,nolistsep]
		\item[Responsibility:]~Responsible for all functionality related to access rights to user roles. E.g. retrieving the user roles that are mandatory for a certain application, updating the roles assigned to users, etc.
		\item[Super-components:]~None
		\item[Sub-components:]~None
		\item[Provided interfaces:]~\iconprovided{}~\vpett{\nameref{int:OnlineServiceOnlineServiceUserRolesManagerRoleMgmt}}
		\item[Required interfaces:]~\iconrequired{}~\vpett{\nameref{int:OtherDataDatabaseOtherDataDBUserRoleMgmt}}		
	\end{description}

% END COMPONENTS

% INTERFACES
\section{Interfaces} \label{sec:interfaces}
  %%%%%%%% AccessRights
  \subsection{AccessRights}\label{int:OnlineServiceOnlineServiceInfrastructureOwnerFacadeAccessRights}
    \begin{description}[noitemsep,nolistsep]
      \item[Provided by:] \iconcomponent{}~\vpett{\nameref{comp:OnlineServiceOnlineServiceInfrastructureOwnerFacade}}
      \item[Required by:] \iconcomponent{}~\vpett{\nameref{comp:InfrastructreOwnerClient}}
      \item[Operations:] ~
    \begin{itemize}[noitemsep,nolistsep,leftmargin=-.25cm]
      \item \textsf{ configureDevice(\ref{data:DataTypesPluggableDeviceID} pID)}
        \begin{itemize}[noitemsep,nolistsep]
           \item Effect: Returns a map of AccessRights and the IDs of customer organisations that have those AccessRights.
\item Created for: UC9.3 - UC9.4
        \end{itemize}
      \item \textsf{List\textless{}\ref{data:DataTypesPluggableDeviceInfo}\textgreater{} getAccessRights(int infrastructureOwnerID)}
        \begin{itemize}[noitemsep,nolistsep]
           \item Effect: Returns a list of PluggableDeviceInfo to display so an infrastructure owner can select a device to configure access rights.
\item Created for: UC9.1
        \end{itemize}
      \item \textsf{void updateAccessRights()}
        \begin{itemize}[noitemsep,nolistsep]
           \item Effect: Updates the access rights on a certain pluggable device for a group of customer organisations.
\item Created for: UC9.6
        \end{itemize}
    \end{itemize}
    \end{description}

  %%%%%%%% AccessRightsMgmt
  \subsection{AccessRightsMgmt}\label{int:DeviceDatabaseDeviceDBAccessRightsMgmt}
    \begin{description}[noitemsep,nolistsep]
      \item[Provided by:] \iconcomponent{}~\vpett{\nameref{comp:DeviceDatabaseDeviceDB}}
      \item[Required by:] \iconcomponent{}~\vpett{\nameref{comp:OnlineServiceOnlineServiceAccessRightsManager}}
      \item[Operations:] ~
    \begin{itemize}[noitemsep,nolistsep,leftmargin=-.25cm]
      \item \textsf{ getCustomerOrganisationsRights(\ref{data:DataTypesPluggableDeviceID} pID, List\textless{}int\textgreater{} custOrgIDs)}
        \begin{itemize}[noitemsep,nolistsep]
           \item Effect: Returns a map of AccessRights and the IDs of customer organisations that have those AccessRights on a certain pluggable device.
\item Created for: UC9.4
        \end{itemize}
      \item \textsf{void updateAccessRights()}
        \begin{itemize}[noitemsep,nolistsep]
           \item Effect: Updates the access rights on a certain pluggable device for a group of customer organisations.
\item Created for: UC9.7
        \end{itemize}
    \end{itemize}
    \end{description}

  %%%%%%%% AccessRightsMgmt
  \subsection{AccessRightsMgmt}\label{int:OnlineServiceOnlineServiceAccessRightsManagerAccessRightsMgmt}
    \begin{description}[noitemsep,nolistsep]
      \item[Provided by:] \iconcomponent{}~\vpett{\nameref{comp:OnlineServiceOnlineServiceAccessRightsManager}}
      \item[Required by:] \iconcomponent{}~\vpett{\nameref{comp:OnlineServiceOnlineServiceInfrastructureOwnerFacade}}
      \item[Operations:] ~
    \begin{itemize}[noitemsep,nolistsep,leftmargin=-.25cm]
      \item \textsf{ getCustomerOrganisationsRights(\ref{data:DataTypesPluggableDeviceID} pID, List\textless{}int\textgreater{} custOrgIDs)}
        \begin{itemize}[noitemsep,nolistsep]
           \item Effect: Returns a map of AccessRights and the IDs of customer organisations that have those AccessRights on a certain pluggable device.
\item Created for: UC9.4
        \end{itemize}
      \item \textsf{void updateAccessRights()}
        \begin{itemize}[noitemsep,nolistsep]
           \item Effect: Updates the access rights on a certain pluggable device for a group of customer organisations.
\item Created for: UC9.7
        \end{itemize}
    \end{itemize}
    \end{description}

  %%%%%%%% Actuate
  \subsection{Actuate}\label{int:PuggableDevicePluggableDeviceFacadeActuate}
    \begin{description}[noitemsep,nolistsep]
      \item[Provided by:] \iconcomponent{}~\vpett{\nameref{comp:PuggableDevicePluggableDeviceFacade}}
      \item[Required by:] \iconcomponent{}~\vpett{\nameref{comp:MoteMoteFacade}}
      \item[Operations:] ~
    \begin{itemize}[noitemsep,nolistsep,leftmargin=-.25cm]
      \item \textsf{void sendActuationCommand(string commandName)}
        \begin{itemize}[noitemsep,nolistsep]
           \item Effect: Send an actuation command to the actuator. Sending an unknown actuation command has no effect.
        \end{itemize}
    \end{itemize}
    \end{description}

  %%%%%%%% Actuation
  \subsection{Actuation}\label{int:OnlineServiceOnlineServiceApplicationManagerActuation}
    \begin{description}[noitemsep,nolistsep]
      \item[Provided by:] \iconcomponent{}~\vpett{\nameref{comp:OnlineServiceOnlineServiceApplicationManagerActuationCommandConstructor}}, \iconcomponent{}~\vpett{\nameref{comp:OnlineServiceOnlineServiceApplicationManager}}
      \item[Required by:] \iconcomponent{}~\vpett{\nameref{comp:OnlineServiceOnlineServiceApplicationFacade}}
      \item[Operations:] ~
% no operations
    \end{description}

  %%%%%%%% AppData
  \subsection{AppData}\label{int:OnlineServiceOnlineServiceApplicationFacadeAppData}
    \begin{description}[noitemsep,nolistsep]
      \item[Provided by:] \iconcomponent{}~\vpett{\nameref{comp:OnlineServiceOnlineServiceApplicationFacade}}
      \item[Required by:] \iconcomponent{}~\vpett{\nameref{comp:ApplicationClient}}
      \item[Operations:] ~
% no operations
    \end{description}

  %%%%%%%% AppDeviceMgmt
  \subsection{AppDeviceMgmt}\label{int:GatewayGatewayGatewayFacadeAppDeviceMgmt}
    \begin{description}[noitemsep,nolistsep]
      \item[Provided by:] \iconcomponent{}~\vpett{\nameref{comp:GatewayGatewayGatewayFacade}}
      \item[Required by:] \iconcomponent{}~\vpett{\nameref{comp:OnlineServiceOnlineServiceApplicationManagerActuationCommandConstructor}}, \iconcomponent{}~\vpett{\nameref{comp:OnlineServiceOnlineServiceApplicationManager}}
      \item[Operations:] ~
    \begin{itemize}[noitemsep,nolistsep,leftmargin=-.25cm]
      \item \textsf{bool areEssentialDevicesOperational(int applicationID)}
        \begin{itemize}[noitemsep,nolistsep]
           \item Effect: Returns true if all essential devices for the application with id "applicationID" are operational.
\item Created for: UC18
        \end{itemize}
      \item \textsf{void setPluggableDevicesRequirements(int applicationID, List\textless{}\ref{data:DataTypesPluggableDeviceInfo}\textgreater{} devices)}
        \begin{itemize}[noitemsep,nolistsep]
           \item Effect: Sets an application's requirements for pluggable devices.
\item Created for: Av3: "Application providers can design their applications such that they explicitly require redundancy in the available pluggable devices." \\
TODO: update this with Relationship type?
        \end{itemize}
    \end{itemize}
    \end{description}

  %%%%%%%% AppInstanceMgmt
  \subsection{AppInstanceMgmt}\label{int:OnlineServiceOnlineServiceApplicationManagerAppInstanceMgmt}
    \begin{description}[noitemsep,nolistsep]
      \item[Provided by:] \iconcomponent{}~\vpett{\nameref{comp:OnlineServiceOnlineServiceApplicationManagerApplicationContainer}}
      \item[Required by:] \iconcomponent{}~\vpett{\nameref{comp:OnlineServiceOnlineServiceApplicationManagerApplicationContainerManager}}, \iconcomponent{}~\vpett{\nameref{comp:OnlineServiceOnlineServiceApplicationManagerApplicationContainerMonitor}}
      \item[Operations:] ~
% no operations
    \end{description}

  %%%%%%%% AppMgmt
  \subsection{AppMgmt}\label{int:OtherDataDatabaseOtherDataDBAppMgmt}
    \begin{description}[noitemsep,nolistsep]
      \item[Provided by:] \iconcomponent{}~\vpett{\nameref{comp:OtherDataDatabaseOtherDataDB}}
      \item[Required by:] \iconcomponent{}~\vpett{\nameref{comp:OnlineServiceOnlineServiceApplicationManager}}
      \item[Operations:] ~
    \begin{itemize}[noitemsep,nolistsep,leftmargin=-.25cm]
      \item \textsf{void activateApplication(int applicationInstanceID, string status)}
        \begin{itemize}[noitemsep,nolistsep]
           \item Effect: Sets an ApplicationInstance's status in the \vpett{\nameref{comp:OtherDataDatabaseOtherDataDB}} to 'active'.
\item Created for: UC17.4, U2 - easy applications
        \end{itemize}
      \item \textsf{int createNewApplicationInstance(int custOrgID, int applicationID)}
        \begin{itemize}[noitemsep,nolistsep]
           \item Effect: Creates a new ApplicationInstance for an application for a customer organisation and returns its id.
\item Created for: UC19.4, U2 - easy applications
        \end{itemize}
      \item \textsf{List\textless{}\ref{data:DataTypesApplication}\textgreater{} getApplications()}
        \begin{itemize}[noitemsep,nolistsep]
           \item Effect: Returns a list of applications in the system.
\item Created for: UC19.2, U2 - easy applications
        \end{itemize}
      \item \textsf{List\textless{}int\textgreater{} getApplicationsForDevice()}
        \begin{itemize}[noitemsep,nolistsep]
           \item Effect: Returns a list of applications that can use the device with id "pID".
\item Created for: UC11: the system looks up the list of applications that use the pluggable device
        \end{itemize}
      \item \textsf{List\textless{}\ref{data:DataTypesPluggableDeviceID}\textgreater{} getDevicesForApplication(int applicationInstanceID)}
        \begin{itemize}[noitemsep,nolistsep]
           \item Effect: Returns a list of PluggableDeviceID of pluggable devices that an ApplicationInstance can use.
\item Created for: UC17.2, U2 - easy applications
        \end{itemize}
      \item \textsf{string getInstallationInstructions(int applicationID)}
        \begin{itemize}[noitemsep,nolistsep]
           \item Effect: Returns the installation instructions of a certain application. If there are no installation instructions set, returns an empty string.
\item Created for: UC17.6, U2 - easy applications
        \end{itemize}
      \item \textsf{List\textless{}\ref{data:DataTypesRoomTopology}\textgreater{} getNecessaryDevicesAndTopologyConfigurations(int applicationID)}
        \begin{itemize}[noitemsep,nolistsep]
           \item Effect: Returns a list of RoomTopology which is a minimal requirement for a certain application to run. This can used to display the requirements to a user or to check if requirements are fulfilled.
\item Created for: UC19.5, U2 - easy applications
        \end{itemize}
      \item \textsf{void updateApplication(\ref{data:DataTypesApplicationInstance} instance)}
        \begin{itemize}[noitemsep,nolistsep]
           \item Effect: Updates an application in the database (e.g. change state to 'inactive').
\item Created for: UC18, Av3: automatic suspension/reactivation of applications.
        \end{itemize}
      \item \textsf{void updateApplicationDevicesSettings(int applicationInstanceID, List\textless{}\ref{data:DataTypesPluggableDeviceID}\textgreater{} devices, List\textless{}\ref{data:DataTypesRelationship}\textgreater{} relationships)}
        \begin{itemize}[noitemsep,nolistsep]
           \item Effect: Updates an ApplicationInstance's device settings. This includes which devices the instance can use and which relationships exist between those devices.
\item Created for: UC19.6, U2 - easy applications
        \end{itemize}
      \item \textsf{void updateCriticality(int applicationInstanceID, int isCritical)}
        \begin{itemize}[noitemsep,nolistsep]
           \item Effect: Updates the criticality of an ApplicationInstance.
\item Created for: UC19.11, U2 - easy applications
        \end{itemize}
      \item \textsf{void updateSubscription(\ref{data:DataTypesSubscription} subscription)}
        \begin{itemize}[noitemsep,nolistsep]
           \item Effect: Updates a subscription in the database (e.g. change state to 'disabled').
\item Created for: UC18
        \end{itemize}
    \end{itemize}
    \end{description}

  %%%%%%%% AppMgmt
  \subsection{AppMgmt}\label{int:GatewayGatewayGatewayFacadeAppMgmt}
    \begin{description}[noitemsep,nolistsep]
      \item[Provided by:] \iconcomponent{}~\vpett{\nameref{comp:GatewayGatewayGatewayFacade}}
      \item[Required by:] \iconcomponent{}~\vpett{\nameref{comp:OnlineServiceOnlineServiceApplicationManager}}
      \item[Operations:] ~
    \begin{itemize}[noitemsep,nolistsep,leftmargin=-.25cm]
      \item \textsf{void activateApplicationInstance(int applicationInstanceID)}
        \begin{itemize}[noitemsep,nolistsep]
           \item Effect: Activates an ApplicationInstance that is running on the gateway.
\item Created for: UC17.3, U2 - easy applications
        \end{itemize}
    \end{itemize}
    \end{description}

  %%%%%%%% AppMgmt
  \subsection{AppMgmt}\label{int:GatewayGatewayGWApplicationContainerManagerAppMgmt}
    \begin{description}[noitemsep,nolistsep]
      \item[Provided by:] \iconcomponent{}~\vpett{\nameref{comp:GatewayGatewayGWApplicationContainerManager}}
      \item[Required by:] \iconcomponent{}~\vpett{\nameref{comp:GatewayGatewayGatewayFacade}}
      \item[Operations:] ~
    \begin{itemize}[noitemsep,nolistsep,leftmargin=-.25cm]
      \item \textsf{void activateApplicationInstance(int applicationInstanceID)}
        \begin{itemize}[noitemsep,nolistsep]
           \item Effect: Activates an ApplicationInstance that is running on the gateway.
\item Created for: UC17.3, U2 - easy applications
        \end{itemize}
    \end{itemize}
    \end{description}

  %%%%%%%% AppMgmt
  \subsection{AppMgmt}\label{int:OnlineServiceOnlineServiceApplicationManagerAppMgmt}
    \begin{description}[noitemsep,nolistsep]
      \item[Provided by:] \iconcomponent{}~\vpett{\nameref{comp:OnlineServiceOnlineServiceApplicationManager}}
      \item[Required by:] \iconcomponent{}~\vpett{\nameref{comp:GatewayGatewayGatewayFacade}}
      \item[Operations:] ~
    \begin{itemize}[noitemsep,nolistsep,leftmargin=-.25cm]
      \item \textsf{void activateApplicationInstance(int applicationInstanceID)}
        \begin{itemize}[noitemsep,nolistsep]
           \item Effect: Activates a new instance of an application.
\item Created for: UC18, Av3: automatic suspension/reactivation of applications.
        \end{itemize}
      \item \textsf{void checkApplicationsForActivationForInfrastructureOwner(int infrastructureOwnerID)}
        \begin{itemize}[noitemsep,nolistsep]
           \item Effect: Checks and activates applications which can now execute again. The applications checked are those that are subscribed to by customers organisations associated to the given infrastructure owner.
\item Created for: UC17, UC6.3 - reintroduced device
        \end{itemize}
      \item \textsf{void deactivateApplicationInstance(int applicationInstanceID)}
        \begin{itemize}[noitemsep,nolistsep]
           \item Effect: Deactivates a running instance of an application.
\item Created for: UC18, Av3: automatic suspension/reactivation of applications.
        \end{itemize}
      \item \textsf{void updateApplicationDevicesSettings(int applicationInstanceID, List\textless{}\ref{data:DataTypesPluggableDeviceID}\textgreater{} devices, List\textless{}\ref{data:DataTypesRelationship}\textgreater{} relationships)}
        \begin{itemize}[noitemsep,nolistsep]
           \item Effect: Updates an ApplicationInstance's device settings. This includes which devices the instance can use and which relationships exist between those devices.
\item Created for: UC19.6, U2 - easy applications
        \end{itemize}
    \end{itemize}
    \end{description}

  %%%%%%%% AppMgmt
  \subsection{AppMgmt}\label{int:OnlineServiceOnlineServiceApplicationManagerApplicationContainerManagerAppMgmt}
    \begin{description}[noitemsep,nolistsep]
      \item[Provided by:] None
      \item[Required by:] None
      \item[Operations:] ~
% no operations
    \end{description}

  %%%%%%%% Apps
  \subsection{Apps}\label{int:OnlineServiceOnlineServiceApplicationManagerApps}
    \begin{description}[noitemsep,nolistsep]
      \item[Provided by:] \iconcomponent{}~\vpett{\nameref{comp:OnlineServiceOnlineServiceApplicationManager}}
      \item[Required by:] \iconcomponent{}~\vpett{\nameref{comp:OnlineServiceOnlineServiceApplicationFacade}}, \iconcomponent{}~\vpett{\nameref{comp:OnlineServiceOnlineServiceCustomerOrganisationFacade}}, \iconcomponent{}~\vpett{\nameref{comp:OnlineServiceOnlineServiceSubscriptionManager}}
      \item[Operations:] ~
    \begin{itemize}[noitemsep,nolistsep,leftmargin=-.25cm]
      \item \textsf{void activateApplication(int applicationInstanceID)}
        \begin{itemize}[noitemsep,nolistsep]
           \item Effect: Activates an ApplicationInstance.
\item Created for: UC19.14, U2 - easy applications
        \end{itemize}
      \item \textsf{int createNewApplicationInstance(int custOrgID, int applicationID)}
        \begin{itemize}[noitemsep,nolistsep]
           \item Effect: Creates a new ApplicationInstance for an application for a customer organisation and returns its id.
\item Created for: UC19.4, U2 - easy applications
        \end{itemize}
      \item \textsf{List\textless{}\ref{data:DataTypesApplication}\textgreater{} getApplications()}
        \begin{itemize}[noitemsep,nolistsep]
           \item Effect: Returns a list of applications in the system.
\item Created for: UC19.2, U2 - easy applications
        \end{itemize}
      \item \textsf{List\textless{}\ref{data:DataTypesRoomTopology}\textgreater{} getNecessaryDevicesAndTopologyConfigurations(int applicationID)}
        \begin{itemize}[noitemsep,nolistsep]
           \item Effect: Returns a list of RoomTopology which is a minimal requirement for a certain application to run. This can used to display the requirements to a user or to check if requirements are fulfilled.
\item Created for: UC19.5, U2 - easy applications
        \end{itemize}
      \item \textsf{void updateCriticality(int applicationInstanceID, boolean isCritical)}
        \begin{itemize}[noitemsep,nolistsep]
           \item Effect: Updates the criticality of an ApplicationInstance.
\item Created for: UC19.11, U2 - easy applications
        \end{itemize}
    \end{itemize}
    \end{description}

  %%%%%%%% Config
  \subsection{Config}\label{int:PuggableDevicePluggableDeviceFacadeConfig}
    \begin{description}[noitemsep,nolistsep]
      \item[Provided by:] \iconcomponent{}~\vpett{\nameref{comp:PuggableDevicePluggableDeviceFacade}}
      \item[Required by:] \iconcomponent{}~\vpett{\nameref{comp:MoteMoteFacade}}
      \item[Operations:] ~
    \begin{itemize}[noitemsep,nolistsep,leftmargin=-.25cm]
      \item \textsf{Map\textless{}String, String\textgreater{} getConfig()}
        \begin{itemize}[noitemsep,nolistsep]
           \item Effect: Returns the current configuration of a pluggable device as a parameter-value map.
        \end{itemize}
      \item \textsf{boolean setConfig()}
        \begin{itemize}[noitemsep,nolistsep]
           \item Effect: Set the given configuration parameters of the pluggable device to the given values. Setting unknown parameters on a pluggable device (e.g., 'noise threshold' -> '3' on a light sensor) has no effect.
\item Created for: Given constraint, UC11: pluggable device needs to be initialised, M1: pluggable device must be able to be initialised
        \end{itemize}
    \end{itemize}
    \end{description}

  %%%%%%%% DataConversion
  \subsection{DataConversion}\label{int:OnlineServiceOnlineServiceDeviceDataConverterDataConversion}
    \begin{description}[noitemsep,nolistsep]
      \item[Provided by:] \iconcomponent{}~\vpett{\nameref{comp:OnlineServiceOnlineServiceDeviceDataConverter}}
      \item[Required by:] \iconcomponent{}~\vpett{\nameref{comp:GatewayGatewayGatewayFacade}}
      \item[Operations:] ~
    \begin{itemize}[noitemsep,nolistsep,leftmargin=-.25cm]
      \item \textsf{\ref{data:DataTypesDeviceData} convert(\ref{data:DataTypesPluggableDeviceID} pID, \ref{data:DataTypesDeviceData} data, string type)}
        \begin{itemize}[noitemsep,nolistsep]
           \item Effect: Converts pluggable device data into other pluggable device data that contains the same information in a different type.
\item Created for: M1: data processing subsystem should be extended with relevant data conversions
        \end{itemize}
    \end{itemize}
    \end{description}

  %%%%%%%% DeliveryMgmt
  \subsection{DeliveryMgmt}\label{int:OnlineServiceOnlineServiceNotificationHandlerDeliveryMgmt}
    \begin{description}[noitemsep,nolistsep]
      \item[Provided by:] \iconcomponent{}~\vpett{\nameref{comp:OnlineServiceOnlineServiceNotificationHandler}}
      \item[Required by:] None
      \item[Operations:] ~
    \begin{itemize}[noitemsep,nolistsep,leftmargin=-.25cm]
      \item \textsf{void acknowledgement(int notificationID)}
        \begin{itemize}[noitemsep,nolistsep]
           \item Effect: Sends an acknowledgement to the system for a certain notification to denote that a notification has been received.
\item Created for: UC15
        \end{itemize}
    \end{itemize}
    \end{description}

  %%%%%%%% DeviceData
  \subsection{DeviceData}\label{int:PluggableDeviceDatabasePluggableDeviceDataDBDeviceData}
    \begin{description}[noitemsep,nolistsep]
      \item[Provided by:] \iconcomponent{}~\vpett{\nameref{comp:PluggableDeviceDatabasePluggableDeviceDataDB}}
      \item[Required by:] \iconcomponent{}~\vpett{\nameref{comp:OnlineServiceOnlineServiceDeviceDataScheduler}}
      \item[Operations:] ~
    \begin{itemize}[noitemsep,nolistsep,leftmargin=-.25cm]
      \item \textsf{List\textless{}\ref{data:DataTypesDeviceData}\textgreater{} getData(\ref{data:DataTypesPluggableDeviceID} pID, \ref{data:DataTypesDateTime} from, \ref{data:DataTypesDateTime} to)}
        \begin{itemize}[noitemsep,nolistsep]
           \item Effect: Returns data from a specific device in a certain time period.
\item Created for: P2: lookup queries
        \end{itemize}
      \item \textsf{void rcvData(\ref{data:DataTypesPluggableDeviceID} pID, \ref{data:DataTypesDeviceData} data)}
        \begin{itemize}[noitemsep,nolistsep]
           \item Effect: Sends pluggable device data to the DB to be stored.
\item Created for: UC11, P2: storing new pluggable data
        \end{itemize}
    \end{itemize}
    \end{description}

  %%%%%%%% DeviceData
  \subsection{DeviceData}\label{int:MoteMoteFacadeDeviceData}
    \begin{description}[noitemsep,nolistsep]
      \item[Provided by:] \iconcomponent{}~\vpett{\nameref{comp:MoteMoteFacade}}
      \item[Required by:] \iconcomponent{}~\vpett{\nameref{comp:PuggableDevicePluggableDeviceFacade}}
      \item[Operations:] ~
    \begin{itemize}[noitemsep,nolistsep,leftmargin=-.25cm]
      \item \textsf{void rcvData()}
        \begin{itemize}[noitemsep,nolistsep]
           \item Effect: Propagates pluggable device data to the connected gateway by calling rcvData on the gateway. (Initiated by the device).
\item Created for: UC11, P2: storing new pluggable data
        \end{itemize}
      \item \textsf{void rcvDataCallback(\ref{data:DataTypesPluggableDeviceID} pID, \ref{data:DataTypesDeviceData} data, int requestID)}
        \begin{itemize}[noitemsep,nolistsep]
           \item Effect: Propagates pluggable device data to the connected gateway by calling rcvData on the gateway. (Callback of getDataAsync).
\item Created for: UC11, P2: storing new pluggable data
        \end{itemize}
    \end{itemize}
    \end{description}

  %%%%%%%% DeviceData
  \subsection{DeviceData}\label{int:GatewayGatewayGatewayFacadeDeviceData}
    \begin{description}[noitemsep,nolistsep]
      \item[Provided by:] \iconcomponent{}~\vpett{\nameref{comp:GatewayGatewayGatewayFacade}}
      \item[Required by:] \iconcomponent{}~\vpett{\nameref{comp:MoteMoteFacade}}
      \item[Operations:] ~
    \begin{itemize}[noitemsep,nolistsep,leftmargin=-.25cm]
      \item \textsf{void rcvData(\ref{data:DataTypesPluggableDeviceID} pID, \ref{data:DataTypesDeviceData} data)}
        \begin{itemize}[noitemsep,nolistsep]
           \item Effect: Provides pluggable device data to the gateway (Initiated by the device).
        \end{itemize}
      \item \textsf{void rcvDataCallback(\ref{data:DataTypesPluggableDeviceID} pID, \ref{data:DataTypesDeviceData} data, int requestID)}
        \begin{itemize}[noitemsep,nolistsep]
           \item Effect: Provides device data to the gateway (Callback of getDataAsync).
        \end{itemize}
    \end{itemize}
    \end{description}

  %%%%%%%% DeviceData
  \subsection{DeviceData}\label{int:OnlineServiceOnlineServiceDeviceDataSchedulerDeviceData}
    \begin{description}[noitemsep,nolistsep]
      \item[Provided by:] \iconcomponent{}~\vpett{\nameref{comp:OnlineServiceOnlineServiceDeviceDataScheduler}}
      \item[Required by:] \iconcomponent{}~\vpett{\nameref{comp:GatewayGatewayGatewayFacade}}
      \item[Operations:] ~
    \begin{itemize}[noitemsep,nolistsep,leftmargin=-.25cm]
      \item \textsf{void rcvData(\ref{data:DataTypesPluggableDeviceID} pID, \ref{data:DataTypesDeviceData} data)}
        \begin{itemize}[noitemsep,nolistsep]
           \item Effect: Sends pluggable device data to the scheduler to be processed.
\item Created for: UC11, P2: storing new pluggable data
        \end{itemize}
    \end{itemize}
    \end{description}

  %%%%%%%% DeviceMgmt
  \subsection{DeviceMgmt}\label{int:MoteMoteFacadeDeviceMgmt}
    \begin{description}[noitemsep,nolistsep]
      \item[Provided by:] \iconcomponent{}~\vpett{\nameref{comp:MoteMoteFacade}}
      \item[Required by:] \iconcomponent{}~\vpett{\nameref{comp:GatewayGatewayGatewayFacade}}
      \item[Operations:] ~
    \begin{itemize}[noitemsep,nolistsep,leftmargin=-.25cm]
      \item \textsf{List\textless{}\ref{data:DataTypesPluggableDeviceInfo}\textgreater{} getConnectedDevices()}
        \begin{itemize}[noitemsep,nolistsep]
           \item Effect: Effect: Returns a list of information about devices that are connected to the mote.
\item Created for: UC18
\item Tradeoff: send PluggableDeviceID instead of DeviceInfo. If you send DeviceInfo, then \vpett{\nameref{comp:OnlineServiceOnlineServiceApplicationManager}} does not have to fetch this info. If you send PluggableDeviceID's, then less bandwidth is used and the Gateways do less work.
        \end{itemize}
      \item \textsf{void setConfig(\ref{data:DataTypesPluggableDeviceID} pID, Map\textless{}String, String\textgreater{} config)}
        \begin{itemize}[noitemsep,nolistsep]
           \item Effect: Set the given configuration parameters of a PluggableDevice to the given values. Setting unknown parameters on a PluggableDevice has no effect. \\
\item Created for: UC11: pluggable device needs to be initialised, M1: pluggable device must be able to be initialised
        \end{itemize}
    \end{itemize}
    \end{description}

  %%%%%%%% DeviceMgmt
  \subsection{DeviceMgmt}\label{int:DeviceDatabaseDeviceDBDeviceMgmt}
    \begin{description}[noitemsep,nolistsep]
      \item[Provided by:] \iconcomponent{}~\vpett{\nameref{comp:DeviceDatabaseDeviceDB}}
      \item[Required by:] \iconcomponent{}~\vpett{\nameref{comp:GatewayGatewayGatewayFacade}}
      \item[Operations:] ~
    \begin{itemize}[noitemsep,nolistsep,leftmargin=-.25cm]
      \item \textsf{void addDevice(\ref{data:DataTypesPluggableDeviceID} pID, \ref{data:DataTypesPluggableDeviceType} type, Map\textless{}string, string\textgreater{} configurations, int moteID)}
        \begin{itemize}[noitemsep,nolistsep]
           \item Effect: Adds a new pluggable device in the \vpett{\nameref{comp:DeviceDatabaseDeviceDB}} and adds a reference to a mote. The device's status is 'uninitialised' by default and it's current configurations (which are now the default configurations) are stored as well. If the device already exists, removes the data first (in case the device is plugged into a different mote or on a different network).
\item Created for: UC6.3, U2 - easy pluggable device installation
        \end{itemize}
      \item \textsf{int addMote(\ref{data:DataTypesMoteInfo} mote, int gatewayID, \ref{data:DataTypesIPAddress} moteIPAddress)}
        \begin{itemize}[noitemsep,nolistsep]
           \item Effect: Adds a new mote in the \vpett{\nameref{comp:DeviceDatabaseDeviceDB}} along with an IP address and a reference to a gateway. Returns the DB id for the mote.
\item Created for: UC4.3, U2 - easy mote installation
        \end{itemize}
      \item \textsf{Map\textless{}string, string\textgreater{} getConfigDB(\ref{data:DataTypesPluggableDeviceID} pID)}
        \begin{itemize}[noitemsep,nolistsep]
           \item Effect: Gets the last set configurations of a pluggable device from the \vpett{\nameref{comp:DeviceDatabaseDeviceDB}}.
\item Created for: UC6.3 - reintroduced device
        \end{itemize}
      \item \textsf{void reactivateDevice(\ref{data:DataTypesPluggableDeviceID} pID)}
        \begin{itemize}[noitemsep,nolistsep]
           \item Effect: Changes the status of a pluggable device to 'active'.
\item Created for: UC6.3 - reintroduced device
        \end{itemize}
      \item \textsf{void reactivateMote(int moteID)}
        \begin{itemize}[noitemsep,nolistsep]
           \item Effect: Changes the status of the mote with DB id 'moteID' to 'active'.
\item Created for: U2 - Reintroducing a previously known mote should not require any conguration.
        \end{itemize}
      \item \textsf{void registerGateway(int gatewayID, \ref{data:DataTypesIPAddress} address)}
        \begin{itemize}[noitemsep,nolistsep]
           \item Effect: Sets a gateway's status to 'active' and updates its IP address.
\item Created for: U2 - gateway installation
        \end{itemize}
    \end{itemize}
    \end{description}

  %%%%%%%% DeviceMgmt
  \subsection{DeviceMgmt}\label{int:GatewayGatewayDeviceManagerDeviceMgmt}
    \begin{description}[noitemsep,nolistsep]
      \item[Provided by:] \iconcomponent{}~\vpett{\nameref{comp:GatewayGatewayDeviceManager}}
      \item[Required by:] \iconcomponent{}~\vpett{\nameref{comp:GatewayGatewayGatewayFacade}}
      \item[Operations:] ~
    \begin{itemize}[noitemsep,nolistsep,leftmargin=-.25cm]
      \item \textsf{bool areEssentialDevicesOperational(int applicationID)}
        \begin{itemize}[noitemsep,nolistsep]
           \item Effect: Returns true if all essential devices for the application with id "applicationID" are operational.
\item Created for: UC18
        \end{itemize}
      \item \textsf{void heartbeat(int moteID, List\textless{}\ref{data:DataTypesPluggableDeviceInfo}\textgreater{} devicesmeter)}
        \begin{itemize}[noitemsep,nolistsep]
           \item Effect: Sends a heartbeat from a mote to check/update timers for operational devices.
\item Created for: UC14, Av3: failure detection
        \end{itemize}
      \item \textsf{bool isDeviceInitialised(\ref{data:DataTypesPluggableDeviceID} pID)}
        \begin{itemize}[noitemsep,nolistsep]
           \item Effect: Returns true if the device with id "pID" has been initialized.
\item Created for: UC11: pluggable device needs to be initialised, M1: pluggable device must be able to be initialised
\item TODO: need this check? is 'initialized' status stored in DB or on gateways? or both?
        \end{itemize}
      \item \textsf{void setPluggableDevicesRequirements(int applicationID, List\textless{}\ref{data:DataTypesPluggableDeviceInfo}\textgreater{} devices)}
        \begin{itemize}[noitemsep,nolistsep]
           \item Effect: Sets an application's requirements for pluggable devices.
\item Created for: Av3: "Application providers can design their applications such that they explicitly require redundancy in the available pluggable devices."
        \end{itemize}
    \end{itemize}
    \end{description}

  %%%%%%%% DeviceMgmt
  \subsection{DeviceMgmt}\label{int:GatewayGatewayGatewayFacadeDeviceMgmt}
    \begin{description}[noitemsep,nolistsep]
      \item[Provided by:] \iconcomponent{}~\vpett{\nameref{comp:GatewayGatewayGatewayFacade}}
      \item[Required by:] \iconcomponent{}~\vpett{\nameref{comp:GatewayGatewayDeviceManager}}
      \item[Operations:] ~
    \begin{itemize}[noitemsep,nolistsep,leftmargin=-.25cm]
      \item \textsf{void addDevice(\ref{data:DataTypesPluggableDeviceID} pID, \ref{data:DataTypesPluggableDeviceType} type, Map\textless{}string, string\textgreater{} configurations, int moteID)}
        \begin{itemize}[noitemsep,nolistsep]
           \item Effect: Adds a new pluggable device in the \vpett{\nameref{comp:DeviceDatabaseDeviceDB}} and adds a reference to a mote. The device's status is 'uninitialised' by default and it's current configurations (which are now the default configurations) are stored as well. If the device already exists, removes the data first (in case the device is plugged into a different mote or on a different network). \\
Adds a new pluggable device to the topology of the infrastructure owner and links it to a mote. The device gets the mote's location by default. If the device is already linked to another mote, overwrites that link. \\
Notifies the infrastructure owner that owns the gateway that the pluggable device was detected.
\item Created for: UC6.3, U2 - easy pluggable device installation
        \end{itemize}
      \item \textsf{int addMote(\ref{data:DataTypesMoteInfo} mote, int gatewayID, \ref{data:DataTypesIPAddress} moteIPAddress)}
        \begin{itemize}[noitemsep,nolistsep]
           \item Effect: Adds a new mote in the \vpett{\nameref{comp:DeviceDatabaseDeviceDB}} along with an IP address and a reference to a gateway. Returns the DB id for the mote. \\
Adds a new mote to the topology of the infrastructure owner. The mote is linked to a gateway and gets status 'unplaced' by default. \\
Notifies the infrastructure owner that owns the gateway that a new mote is available for conguration in the topology.
\item Created for: UC4.3, U2 - easy mote installation
        \end{itemize}
      \item \textsf{void checkApplicationsForActivationForInfrastructureOwner(int infrastructureOwnerID)}
        \begin{itemize}[noitemsep,nolistsep]
           \item Effect: Checks and activates applications which can now execute again. The applications checked are those that are subscribed to by customers organisations associated to the given infrastructure owner.
\item Created for: UC17, UC6.3 - reintroduced device
        \end{itemize}
      \item \textsf{void deactivateApplicationInstance(int applicationInstanceID)}
        \begin{itemize}[noitemsep,nolistsep]
           \item Effect: Deactivates a certain application. This could happen when mandatory pluggable devices for the application are missing.
\item Created for: Av3: automatic suspension/reactivation of applications.
        \end{itemize}
      \item \textsf{Map\textless{}string, string\textgreater{} getConfigDB(\ref{data:DataTypesPluggableDeviceID} pID)}
        \begin{itemize}[noitemsep,nolistsep]
           \item Effect: Gets the last set configurations of a pluggable device from the \vpett{\nameref{comp:DeviceDatabaseDeviceDB}}.
\item Created for: UC6.3 - reintroduced device
        \end{itemize}
      \item \textsf{List\textless{}\ref{data:DataTypesPluggableDeviceInfo}\textgreater{} getConnectedDevices()}
        \begin{itemize}[noitemsep,nolistsep]
           \item Effect: Returns a list of information about devices that are connected to the gateway.
\item Created for: UC18
\item Tradeoff: send PluggableDeviceID instead of DeviceInfo. If you send DeviceInfo, then \vpett{\nameref{comp:OnlineServiceOnlineServiceApplicationManager}} does not have to fetch this info. If you send PluggableDeviceID's, then less bandwidth is used and the Gateways do less work.
        \end{itemize}
      \item \textsf{void pluggableDevicePersisentFailure()}
        \begin{itemize}[noitemsep,nolistsep]
           \item Effect: Lets the gateway know that a timer for pluggable device or mote has expired. This will generate a notification for an infrastructure owner.
\item Created for: Av3: The infrastructure owner should be notified of any persistent pluggable device or mote failures.
        \end{itemize}
      \item \textsf{void pluggableDevicePluggedIn(Map\textless{}string, string\textgreater{} mInfo, \ref{data:DataTypesPluggableDeviceID} pID, \ref{data:DataTypesPluggableDeviceType} type)}
        \begin{itemize}[noitemsep,nolistsep]
           \item Effect: Notifies the gateway that a new pluggable device of the given type is connected to the mote.
        \end{itemize}
      \item \textsf{void pluggableDeviceRemoved(\ref{data:DataTypesPluggableDeviceID} pID)}
        \begin{itemize}[noitemsep,nolistsep]
           \item Effect: Notifies the gateway that a pluggable device is removed.
        \end{itemize}
      \item \textsf{void reactivateApplicationInstance(int applicationInstanceID)}
        \begin{itemize}[noitemsep,nolistsep]
           \item Effect: Reactivate an application instance. This could happen automatically after a broken sensor has been replaced.
\item Created for: Av3: automatic suspension/reactivation of applications.
        \end{itemize}
      \item \textsf{void reactivateDevice(\ref{data:DataTypesPluggableDeviceID} pID)}
        \begin{itemize}[noitemsep,nolistsep]
           \item Effect: Changes the status of a pluggable device to 'active'. \\
Changes the status of a pluggable device in the topology to 'placed'. \\
Notifies the infrastructure owner that owns the gateway that the pluggable device was detected and reactivated.
\item Created for: UC6.3 - reintroduced device
        \end{itemize}
      \item \textsf{void reactivateMote(int moteID)}
        \begin{itemize}[noitemsep,nolistsep]
           \item Effect: Changes the status of the mote with DB id 'moteID' to 'active'. \\
Changes the status of the mote in the topology to 'placed'. \\
Notifies the infrastructure owner that owns the gateway that the mote was detected and reactivated.
\item Created for: U2 - Reintroducing a previously known mote should not require any conguration.
        \end{itemize}
      \item \textsf{void setConfig(\ref{data:DataTypesPluggableDeviceID} pID, Map\textless{}String, String\textgreater{} config)}
        \begin{itemize}[noitemsep,nolistsep]
           \item Effect: Set the given configuration parameters of a PluggableDevice to the given values. Setting unknown parameters on a PluggableDevice has no effect. \\
\item Created for: UC11: pluggable device needs to be initialised, M1: pluggable device must be able to be initialised
        \end{itemize}
    \end{itemize}
    \end{description}

  %%%%%%%% DeviceMgmt
  \subsection{DeviceMgmt}\label{int:OnlineServiceOnlineServiceApplicationFacadeDeviceMgmt}
    \begin{description}[noitemsep,nolistsep]
      \item[Provided by:] \iconcomponent{}~\vpett{\nameref{comp:OnlineServiceOnlineServiceApplicationFacade}}
      \item[Required by:] \iconcomponent{}~\vpett{\nameref{comp:ApplicationClient}}
      \item[Operations:] ~
% no operations
    \end{description}

  %%%%%%%% ForwardData
  \subsection{ForwardData}\label{int:OnlineServiceOnlineServiceApplicationManagerForwardData}
    \begin{description}[noitemsep,nolistsep]
      \item[Provided by:] \iconcomponent{}~\vpett{\nameref{comp:OnlineServiceOnlineServiceApplicationManager}}
      \item[Required by:] \iconcomponent{}~\vpett{\nameref{comp:OnlineServiceOnlineServiceDeviceDataScheduler}}
      \item[Operations:] ~
    \begin{itemize}[noitemsep,nolistsep,leftmargin=-.25cm]
      \item \textsf{List\textless{}int\textgreater{} getApplicationsForDevice(\ref{data:DataTypesPluggableDeviceID} pID)}
        \begin{itemize}[noitemsep,nolistsep]
           \item Effect: Returns a list of application instances that can use the device with id "pID". \\
\item Created for: UC11: the system looks up the list of applications that use the pluggable device
        \end{itemize}
      \item \textsf{void rcvData(\ref{data:DataTypesPluggableDeviceID} pID, \ref{data:DataTypesDeviceData} data)}
        \begin{itemize}[noitemsep,nolistsep]
           \item Effect: Sends pluggable device data to an application that wants to use it \\
\item Created for: UC11: system relays data to applications
        \end{itemize}
    \end{itemize}
    \end{description}

  %%%%%%%% Heartbeat
  \subsection{Heartbeat}\label{int:GatewayGatewayGatewayFacadeHeartbeat}
    \begin{description}[noitemsep,nolistsep]
      \item[Provided by:] \iconcomponent{}~\vpett{\nameref{comp:GatewayGatewayGatewayFacade}}
      \item[Required by:] \iconcomponent{}~\vpett{\nameref{comp:MoteMoteFacade}}
      \item[Operations:] ~
    \begin{itemize}[noitemsep,nolistsep,leftmargin=-.25cm]
      \item \textsf{void heartbeat(Map\textless{}string, string\textgreater{} moteInfo, List\textless{}Tuple\textless{}\ref{data:DataTypesPluggableDeviceID}, \ref{data:DataTypesPluggableDeviceType}\textgreater{}\textgreater{} pds)}
        \begin{itemize}[noitemsep,nolistsep]
           \item Effect: Sends a heartbeat from a mote to a gateway, including a list of the pluggable devices and their device types (i.e. those currently plugged into the mote)
\item Created for: Given constraint, UC14, Av3: failure detection
        \end{itemize}
    \end{itemize}
    \end{description}

  %%%%%%%% InvoiceMgmt
  \subsection{InvoiceMgmt}\label{int:OtherDataDatabaseOtherDataDBInvoiceMgmt}
    \begin{description}[noitemsep,nolistsep]
      \item[Provided by:] \iconcomponent{}~\vpett{\nameref{comp:OtherDataDatabaseOtherDataDB}}
      \item[Required by:] \iconcomponent{}~\vpett{\nameref{comp:OnlineServiceOnlineServiceInvoiceManager}}
      \item[Operations:] ~
    \begin{itemize}[noitemsep,nolistsep,leftmargin=-.25cm]
      \item \textsf{void markActivatedApplication(int applicationInstanceID, int custOrgID, \ref{data:DataTypesDateTime} date)}
        \begin{itemize}[noitemsep,nolistsep]
           \item Effect: Updates an ApplicationInstance's billing information: marks the start of a billing period.
\item Created for: UC17.4, U2 - easy applications
        \end{itemize}
    \end{itemize}
    \end{description}

  %%%%%%%% InvoiceMgmt
  \subsection{InvoiceMgmt}\label{int:OnlineServiceOnlineServiceInvoiceManagerInvoiceMgmt}
    \begin{description}[noitemsep,nolistsep]
      \item[Provided by:] \iconcomponent{}~\vpett{\nameref{comp:OnlineServiceOnlineServiceInvoiceManager}}
      \item[Required by:] \iconcomponent{}~\vpett{\nameref{comp:OnlineServiceOnlineServiceApplicationManager}}
      \item[Operations:] ~
    \begin{itemize}[noitemsep,nolistsep,leftmargin=-.25cm]
      \item \textsf{void markActivatedApplication(int applicationInstanceID, int custOrgID, \ref{data:DataTypesDateTime} date)}
        \begin{itemize}[noitemsep,nolistsep]
           \item Effect: Updates an ApplicationInstance's billing information: marks the start of a billing period.
\item Created for: UC17.4, U2 - easy applications
        \end{itemize}
    \end{itemize}
    \end{description}

  %%%%%%%% IOAppMgmt
  \subsection{IOAppMgmt}\label{int:OnlineServiceOnlineServiceApplicationManagerIOAppMgmt}
    \begin{description}[noitemsep,nolistsep]
      \item[Provided by:] \iconcomponent{}~\vpett{\nameref{comp:OnlineServiceOnlineServiceApplicationManager}}
      \item[Required by:] \iconcomponent{}~\vpett{\nameref{comp:OnlineServiceOnlineServiceInfrastructureOwnerFacade}}
      \item[Operations:] ~
    \begin{itemize}[noitemsep,nolistsep,leftmargin=-.25cm]
      \item \textsf{void checkApplicationsForActivationForCustomerOrganisations(List\textless{}int\textgreater{} custOrgIDs)}
        \begin{itemize}[noitemsep,nolistsep]
           \item Effect: Checks and activates 'inactive' ApplicationInstances which can now execute again for a list of customer organisations.
\item Created for: UC9.7
        \end{itemize}
      \item \textsf{void checkApplicationsForDeactivationForCustomerOrganisations(List\textless{}int\textgreater{} custOrgsIDs)}
        \begin{itemize}[noitemsep,nolistsep]
           \item Effect: Checks for ApplicationInstances that require deactivation for a list of customer organisations.
\item Created for: UC9.7
        \end{itemize}
    \end{itemize}
    \end{description}

  %%%%%%%% IODeviceMgmt
  \subsection{IODeviceMgmt}\label{int:DeviceDatabaseDeviceDBIODeviceMgmt}
    \begin{description}[noitemsep,nolistsep]
      \item[Provided by:] \iconcomponent{}~\vpett{\nameref{comp:DeviceDatabaseDeviceDB}}
      \item[Required by:] \iconcomponent{}~\vpett{\nameref{comp:OnlineServiceOnlineServiceInfrastructureOwnerManager}}
      \item[Operations:] ~
    \begin{itemize}[noitemsep,nolistsep,leftmargin=-.25cm]
      \item \textsf{List\textless{}\ref{data:DataTypesPluggableDeviceInfo}\textgreater{} getDevices(int infrastructureOwnerID)}
        \begin{itemize}[noitemsep,nolistsep]
           \item Effect: Returns a list of PluggableDeviceInfo of devices owned by an infrastructure owner.
\item Created for: UC9.2
        \end{itemize}
    \end{itemize}
    \end{description}

  %%%%%%%% IOMgmt
  \subsection{IOMgmt}\label{int:OtherDataDatabaseOtherDataDBIOMgmt}
    \begin{description}[noitemsep,nolistsep]
      \item[Provided by:] \iconcomponent{}~\vpett{\nameref{comp:OtherDataDatabaseOtherDataDB}}
      \item[Required by:] \iconcomponent{}~\vpett{\nameref{comp:OnlineServiceOnlineServiceInfrastructureOwnerManager}}
      \item[Operations:] ~
    \begin{itemize}[noitemsep,nolistsep,leftmargin=-.25cm]
      \item \textsf{List\textless{}int\textgreater{} getCustomerOrganisations(int infrastructureOwnerID)}
        \begin{itemize}[noitemsep,nolistsep]
           \item Effect: Returns a list of IDs of all customer organisations associated with an infrastructure owner.
\item Created for: UC9.4
        \end{itemize}
    \end{itemize}
    \end{description}

  %%%%%%%% IOMgmt
  \subsection{IOMgmt}\label{int:OnlineServiceOnlineServiceInfrastructureOwnerManagerIOMgmt}
    \begin{description}[noitemsep,nolistsep]
      \item[Provided by:] \iconcomponent{}~\vpett{\nameref{comp:OnlineServiceOnlineServiceInfrastructureOwnerManager}}
      \item[Required by:] \iconcomponent{}~\vpett{\nameref{comp:OnlineServiceOnlineServiceInfrastructureOwnerFacade}}
      \item[Operations:] ~
    \begin{itemize}[noitemsep,nolistsep,leftmargin=-.25cm]
      \item \textsf{List\textless{}int\textgreater{} getCustomerOrganisations(int infrastructureOwnerID)}
        \begin{itemize}[noitemsep,nolistsep]
           \item Effect: Returns a list of IDs of all customer organisations associated with an infrastructure owner.
\item Created for: UC9.4
        \end{itemize}
      \item \textsf{List\textless{}\ref{data:DataTypesPluggableDeviceInfo}\textgreater{} getDevices(int infrastructureOwnerID)}
        \begin{itemize}[noitemsep,nolistsep]
           \item Effect: Returns a list of PluggableDeviceInfo of devices owned by an infrastructure owner.
\item Created for: UC9.2
        \end{itemize}
    \end{itemize}
    \end{description}

  %%%%%%%% NotificationDeliveryMgmt
  \subsection{NotificationDeliveryMgmt}\label{int:NotificationDeliveryServiceNodeNotificationDeliveryServiceNotificationDeliveryMgmt}
    \begin{description}[noitemsep,nolistsep]
      \item[Provided by:] None
      \item[Required by:] \iconcomponent{}~\vpett{\nameref{comp:OnlineServiceOnlineServiceNotificationHandler}}
      \item[Operations:] ~
    \begin{itemize}[noitemsep,nolistsep,leftmargin=-.25cm]
      \item \textsf{void notify(Map\textless{}string, string\textgreater{} data)}
        \begin{itemize}[noitemsep,nolistsep]
           \item Effect: Delivers a notification to an end user using a specific delivery service.
\item Created for: UC15
        \end{itemize}
    \end{itemize}
    \end{description}

  %%%%%%%% NotificationMgmt
  \subsection{NotificationMgmt}\label{int:OtherDataDatabaseOtherDataDBNotificationMgmt}
    \begin{description}[noitemsep,nolistsep]
      \item[Provided by:] \iconcomponent{}~\vpett{\nameref{comp:OtherDataDatabaseOtherDataDB}}
      \item[Required by:] \iconcomponent{}~\vpett{\nameref{comp:OnlineServiceOnlineServiceNotificationHandler}}
      \item[Operations:] ~
    \begin{itemize}[noitemsep,nolistsep,leftmargin=-.25cm]
      \item \textsf{int lookupNotificationChannelForUser()}
        \begin{itemize}[noitemsep,nolistsep]
           \item Effect: Returns the id of the type of communication channel a user prefers.
\item Created for: UC15
        \end{itemize}
      \item \textsf{int storeNotification(\ref{data:DataTypesNotification} notification)}
        \begin{itemize}[noitemsep,nolistsep]
           \item Effect: Stores a new notification entry in the database. Returns the id of the new notification. \\
\item Created for: UC15, Av3: notifications
        \end{itemize}
      \item \textsf{int updateNotification(\ref{data:DataTypesNotification} notification)}
        \begin{itemize}[noitemsep,nolistsep]
           \item Effect: Updates an existing notification (e.g. change status to "sent").
\item Created for: UC15
        \end{itemize}
    \end{itemize}
    \end{description}

  %%%%%%%% Notify
  \subsection{Notify}\label{int:OnlineServiceOnlineServiceNotificationHandlerNotify}
    \begin{description}[noitemsep,nolistsep]
      \item[Provided by:] \iconcomponent{}~\vpett{\nameref{comp:OnlineServiceOnlineServiceNotificationHandler}}
      \item[Required by:] \iconcomponent{}~\vpett{\nameref{comp:OnlineServiceOnlineServiceApplicationManager}}, \iconcomponent{}~\vpett{\nameref{comp:GatewayGatewayGatewayFacade}}
      \item[Operations:] ~
    \begin{itemize}[noitemsep,nolistsep,leftmargin=-.25cm]
      \item \textsf{void notify(int userID, string message)}
        \begin{itemize}[noitemsep,nolistsep]
           \item Effect: Stores a new notification in the system and causes it to be sent to a user.
\item Created for: UC14, Av3: notifications
        \end{itemize}
    \end{itemize}
    \end{description}

  %%%%%%%% Other
  \subsection{Other}\label{int:PluggableDeviceDatabasePluggableDeviceDataDBOther}
    \begin{description}[noitemsep,nolistsep]
      \item[Provided by:] \iconcomponent{}~\vpett{\nameref{comp:PluggableDeviceDatabasePluggableDeviceDataDB}}
      \item[Required by:] \iconcomponent{}~\vpett{\nameref{comp:OnlineServiceOnlineServiceOtherFunctionality1}}, \iconcomponent{}~\vpett{\nameref{comp:OnlineServiceOnlineServiceOtherFunctionality2}}
      \item[Operations:] ~
% no operations
    \end{description}

  %%%%%%%% Other
  \subsection{Other}\label{int:OtherDataDatabaseOtherDataDBOther}
    \begin{description}[noitemsep,nolistsep]
      \item[Provided by:] \iconcomponent{}~\vpett{\nameref{comp:OtherDataDatabaseOtherDataDB}}
      \item[Required by:] \iconcomponent{}~\vpett{\nameref{comp:OnlineServiceOnlineServiceOtherFunctionality1}}, \iconcomponent{}~\vpett{\nameref{comp:OnlineServiceOnlineServiceOtherFunctionality2}}
      \item[Operations:] ~
% no operations
    \end{description}

  %%%%%%%% Other
  \subsection{Other}\label{int:OnlineServiceOnlineServiceOtherFunctionality2Other}
    \begin{description}[noitemsep,nolistsep]
      \item[Provided by:] \iconcomponent{}~\vpett{\nameref{comp:OnlineServiceOnlineServiceOtherFunctionality1}}, \iconcomponent{}~\vpett{\nameref{comp:OnlineServiceOnlineServiceOtherFunctionality2}}
      \item[Required by:] \iconcomponent{}~\vpett{\nameref{comp:GatewayGatewayGatewayFacade}}
      \item[Operations:] ~
% no operations
    \end{description}

  %%%%%%%% RequestData
  \subsection{RequestData}\label{int:PuggableDevicePluggableDeviceFacadeRequestData}
    \begin{description}[noitemsep,nolistsep]
      \item[Provided by:] \iconcomponent{}~\vpett{\nameref{comp:PuggableDevicePluggableDeviceFacade}}
      \item[Required by:] \iconcomponent{}~\vpett{\nameref{comp:MoteMoteFacade}}
      \item[Operations:] ~
    \begin{itemize}[noitemsep,nolistsep,leftmargin=-.25cm]
      \item \textsf{\ref{data:DataTypesDeviceData} getData()}
        \begin{itemize}[noitemsep,nolistsep]
           \item Effect: Synchronously retrieve the device data of a device. \\
        \end{itemize}
      \item \textsf{void getDataAsync(int requestID)}
        \begin{itemize}[noitemsep,nolistsep]
           \item Effect: Asynchronously retrieve the device data of a device (by calling rcvDataCallback).
        \end{itemize}
    \end{itemize}
    \end{description}

  %%%%%%%% RequestData
  \subsection{RequestData}\label{int:OnlineServiceOnlineServiceDeviceDataSchedulerRequestData}
    \begin{description}[noitemsep,nolistsep]
      \item[Provided by:] \iconcomponent{}~\vpett{\nameref{comp:OnlineServiceOnlineServiceDeviceDataScheduler}}
      \item[Required by:] \iconcomponent{}~\vpett{\nameref{comp:OnlineServiceOnlineServiceApplicationManager}}
      \item[Operations:] ~
    \begin{itemize}[noitemsep,nolistsep,leftmargin=-.25cm]
      \item \textsf{List\textless{}\ref{data:DataTypesDeviceData}\textgreater{} getData(int applicationID, \ref{data:DataTypesPluggableDeviceID} pID, \ref{data:DataTypesDateTime} from, \ref{data:DataTypesDateTime} to)}
        \begin{itemize}[noitemsep,nolistsep]
           \item Effect: Requests data from a specific device in a certain time period.
\item Created for: P2: requests from applications
        \end{itemize}
    \end{itemize}
    \end{description}

  %%%%%%%% RoleMgmt
  \subsection{RoleMgmt}\label{int:OnlineServiceOnlineServiceUserRolesManagerRoleMgmt}
    \begin{description}[noitemsep,nolistsep]
      \item[Provided by:] \iconcomponent{}~\vpett{\nameref{comp:OnlineServiceOnlineServiceUserRolesManager}}
      \item[Required by:] \iconcomponent{}~\vpett{\nameref{comp:OnlineServiceOnlineServiceApplicationManager}}, \iconcomponent{}~\vpett{\nameref{comp:OnlineServiceOnlineServiceCustomerOrganisationFacade}}
      \item[Operations:] ~
    \begin{itemize}[noitemsep,nolistsep,leftmargin=-.25cm]
      \item \textsf{boolean areMandatoryUserRolesAssigned(int applicationInstanceID)}
        \begin{itemize}[noitemsep,nolistsep]
           \item Effect: Returns true if all mandatory UserRoles for the application have been assigned to users. Finds the relevant customer organisations through the ApplicationInstance.
\item Created for: UC17.1, U2 - easy applications
        \end{itemize}
      \item \textsf{List\textless{}\ref{data:DataTypesUser}\textgreater{} getEndUsers(int custOrgID)}
        \begin{itemize}[noitemsep,nolistsep]
           \item Effect: Returns a list of Users which are associated to a customer organisation.
\item Created for: UC19.8, U2 - easy applications
        \end{itemize}
      \item \textsf{List\textless{}\ref{data:DataTypesUserRole}\textgreater{} getMandatoryUserRoles(int applicationID)}
        \begin{itemize}[noitemsep,nolistsep]
           \item Effect: Returns a list of UserRoles that which need to be assigned in order for an ApplicationInstance to run.
\item Created for: UC19.7, U2 - easy applications
        \end{itemize}
      \item \textsf{List\textless{}\ref{data:DataTypesUserRole}\textgreater{} getOptionalUserRoles(int applicationID)}
        \begin{itemize}[noitemsep,nolistsep]
           \item Effect: Returns a list of UserRoles which can optionally be assigned for an ApplicationInstance.
\item Created for: UC19.7, U2 - easy applications
        \end{itemize}
      \item \textsf{List\textless{}\ref{data:DataTypesUser}\textgreater{} getUsersWithRoles(int applicationInstanceID)}
        \begin{itemize}[noitemsep,nolistsep]
           \item Effect: Returns a list of Users associated to an ApplicationInstance that were assigned UserRoles.
\item Created for: UC17.6, U2 - easy applications
        \end{itemize}
      \item \textsf{void updateUserRoles(int applicationInstanceID, Map\textless{}int, int\textgreater{} usersAndRoles)}
        \begin{itemize}[noitemsep,nolistsep]
           \item Effect: Updates the UserRoles assigned to Users for a certain ApplicationInstance. 'usersAndRoles' maps User IDs to UserRole IDs.
\item Created for: UC19.9, U2 - easy applications
        \end{itemize}
    \end{itemize}
    \end{description}

  %%%%%%%% SubscriptionMgmt
  \subsection{SubscriptionMgmt}\label{int:OtherDataDatabaseOtherDataDBSubscriptionMgmt}
    \begin{description}[noitemsep,nolistsep]
      \item[Provided by:] \iconcomponent{}~\vpett{\nameref{comp:OtherDataDatabaseOtherDataDB}}
      \item[Required by:] \iconcomponent{}~\vpett{\nameref{comp:OnlineServiceOnlineServiceSubscriptionManager}}
      \item[Operations:] ~
    \begin{itemize}[noitemsep,nolistsep,leftmargin=-.25cm]
      \item \textsf{void createSubscription(int custOrgID, int applicationInstanceID)}
        \begin{itemize}[noitemsep,nolistsep]
           \item Effect: Creates a subscription for a customer organisation to an ApplicationInstance. If the customer organisation is already subscribed to an older version of the the application, then the organisation is unsubscribed from that earlier version.
\item Created for: UC19.12-13, U2 - easy applications
        \end{itemize}
      \item \textsf{List\textless{}\ref{data:DataTypesSubscription}\textgreater{} getSubscriptions(int custOrgID)}
        \begin{itemize}[noitemsep,nolistsep]
           \item Effect: Returns a list of subscriptions a customer organisation has.
\item Created for: UC19.2, U2 - easy applications
        \end{itemize}
    \end{itemize}
    \end{description}

  %%%%%%%% SubscriptionMgmt
  \subsection{SubscriptionMgmt}\label{int:OnlineServiceOnlineServiceCustomerOrganisationFacadeSubscriptionMgmt}
    \begin{description}[noitemsep,nolistsep]
      \item[Provided by:] \iconcomponent{}~\vpett{\nameref{comp:OnlineServiceOnlineServiceCustomerOrganisationFacade}}
      \item[Required by:] \iconcomponent{}~\vpett{\nameref{comp:CustomerOgranisationClient}}
      \item[Operations:] ~
    \begin{itemize}[noitemsep,nolistsep,leftmargin=-.25cm]
      \item \textsf{Map\textless{}\ref{data:DataTypesApplication}, \ref{data:DataTypesSubscription}\textgreater{} getApplicationsToSubscribe(int custOrgID)}
        \begin{itemize}[noitemsep,nolistsep]
           \item Effect: Returns a map of Applications and Subscriptions a given customer organisation has to those applications
\item Created for: UC19.1, U2 - easy applications
        \end{itemize}
      \item \textsf{int subscribeToApplication(int custOrgID, int applicationID)}
        \begin{itemize}[noitemsep,nolistsep]
           \item Effect: Creates a new ApplicationInstance for an application for a customer organisation and returns its id.
\item Created for: UC19.4, U2 - easy applications
        \end{itemize}
      \item \textsf{void updateApplicationDevicesSettings(int applicationInstanceID, List\textless{}\ref{data:DataTypesPluggableDeviceID}\textgreater{} devices, List\textless{}\ref{data:DataTypesRelationship}\textgreater{} relationships)}
        \begin{itemize}[noitemsep,nolistsep]
           \item Effect: Updates an ApplicationInstance's device settings. This includes which devices the instance can use and which relationships exist between those devices.
\item Created for: UC19.6, U2 - easy applications
        \end{itemize}
      \item \textsf{void updateCriticality(int applicationInstanceID, boolean isCritical)}
        \begin{itemize}[noitemsep,nolistsep]
           \item Effect: Updates the criticality of an ApplicationInstance.
\item Created for: UC19.11, U2 - easy applications
        \end{itemize}
      \item \textsf{void updateUserRoles(int applicationInstanceID, Map\textless{}int, int\textgreater{} usersAndRoles)}
        \begin{itemize}[noitemsep,nolistsep]
           \item Effect: Updates the UserRoles assigned to Users for a certain ApplicationInstance. 'usersAndRoles' maps User IDs to UserRole IDs.
\item Created for: UC19.9, U2 - easy applications
        \end{itemize}
    \end{itemize}
    \end{description}

  %%%%%%%% SubscriptionMgmt
  \subsection{SubscriptionMgmt}\label{int:OnlineServiceOnlineServiceSubscriptionManagerSubscriptionMgmt}
    \begin{description}[noitemsep,nolistsep]
      \item[Provided by:] \iconcomponent{}~\vpett{\nameref{comp:OnlineServiceOnlineServiceSubscriptionManager}}
      \item[Required by:] \iconcomponent{}~\vpett{\nameref{comp:OnlineServiceOnlineServiceCustomerOrganisationFacade}}
      \item[Operations:] ~
    \begin{itemize}[noitemsep,nolistsep,leftmargin=-.25cm]
      \item \textsf{void createSubscription(int custOrgID, int applicationInstanceID)}
        \begin{itemize}[noitemsep,nolistsep]
           \item Effect: Creates a subscription for a customer organisation to an ApplicationInstance. If the customer organisation is already subscribed to an older version of the the application, then the organisation is unsubscribed from that earlier version.
\item Created for: UC19.12-13, U2 - easy applications
        \end{itemize}
      \item \textsf{Map\textless{}\ref{data:DataTypesApplication}, \ref{data:DataTypesSubscription}\textgreater{} getApplicationsToSubscribe(int custOrgID)}
        \begin{itemize}[noitemsep,nolistsep]
           \item Effect: Returns a map of Applications and Subscriptions a given customer organisation has to those applications
\item Created for: UC19.2, U2 - easy applications
        \end{itemize}
    \end{itemize}
    \end{description}

  %%%%%%%% TopologyMgmt
  \subsection{TopologyMgmt}\label{int:DeviceDatabaseDeviceDBTopologyMgmt}
    \begin{description}[noitemsep,nolistsep]
      \item[Provided by:] \iconcomponent{}~\vpett{\nameref{comp:DeviceDatabaseDeviceDB}}
      \item[Required by:] \iconcomponent{}~\vpett{\nameref{comp:OnlineServiceOnlineServiceTopologyManager}}
      \item[Operations:] ~
    \begin{itemize}[noitemsep,nolistsep,leftmargin=-.25cm]
      \item \textsf{void addDevice(\ref{data:DataTypesPluggableDeviceID} pID, int moteID)}
        \begin{itemize}[noitemsep,nolistsep]
           \item Effect: Adds a new pluggable device to the topology of the infrastructure owner and links it to a mote. The device gets the mote's location by default. If the device is already linked to another mote, overwrites that link.
\item Created for: UC6.3, U2 - easy pluggable device installation
        \end{itemize}
      \item \textsf{void addMote(int moteID, int infrastructureOwnerID, int gatewayID)}
        \begin{itemize}[noitemsep,nolistsep]
           \item Effect: Adds a new mote to a topology of an infrastructure owner. The mote is linked to a gateway and gets status 'unplaced' by default.
\item Created for: UC4.3, U2 - easy mote installation
        \end{itemize}
      \item \textsf{boolean arePluggableDevicesPlaced(List\textless{}\ref{data:DataTypesPluggableDeviceID}\textgreater{} devices)}
        \begin{itemize}[noitemsep,nolistsep]
           \item Effect: Returns true if all pluggable devices in the given list have status 'placed' in the topology.
\item Created for: UC17.2, U2 - easy applications
        \end{itemize}
      \item \textsf{List\textless{}\ref{data:DataTypesRoomTopology}\textgreater{} getTopology(int custOrgID)}
        \begin{itemize}[noitemsep,nolistsep]
           \item Effect: Returns a list of RoomTopology associated to a customer organisation.
\item Created for: UC19.5, U2 - easy applications
        \end{itemize}
      \item \textsf{void reactivateDevice(\ref{data:DataTypesPluggableDeviceID} id)}
        \begin{itemize}[noitemsep,nolistsep]
           \item Effect: Changes the status of a pluggable device in the topology to 'placed'.
\item Created for: UC6.3 - reintroduced device
        \end{itemize}
      \item \textsf{void reactivateMote(int moteID)}
        \begin{itemize}[noitemsep,nolistsep]
           \item Effect: Changes the status of the mote in the topology to 'placed'. The location of the mote is unchanged, it has already been set.
\item Created for: U2 - Reintroducing a previously known mote should not require any conguration.
        \end{itemize}
    \end{itemize}
    \end{description}

  %%%%%%%% TopologyMgmt
  \subsection{TopologyMgmt}\label{int:OnlineServiceOnlineServiceTopologyManagerTopologyMgmt}
    \begin{description}[noitemsep,nolistsep]
      \item[Provided by:] \iconcomponent{}~\vpett{\nameref{comp:OnlineServiceOnlineServiceTopologyManager}}
      \item[Required by:] \iconcomponent{}~\vpett{\nameref{comp:OnlineServiceOnlineServiceApplicationManager}}, \iconcomponent{}~\vpett{\nameref{comp:OnlineServiceOnlineServiceCustomerOrganisationFacade}}, \iconcomponent{}~\vpett{\nameref{comp:GatewayGatewayGatewayFacade}}
      \item[Operations:] ~
    \begin{itemize}[noitemsep,nolistsep,leftmargin=-.25cm]
      \item \textsf{void addDevice(\ref{data:DataTypesPluggableDeviceID} id, int moteID)}
        \begin{itemize}[noitemsep,nolistsep]
           \item Effect: Adds a new pluggable device to the topology of the infrastructure owner and links it to a mote. The device gets the mote's location by default. If the device is already linked to another mote, overwrites that link.
\item Created for: UC6.3, U2 - easy pluggable device installation
        \end{itemize}
      \item \textsf{void addMote(int moteID, int gatewayID, int infrastructureOwnerID)}
        \begin{itemize}[noitemsep,nolistsep]
           \item Effect: Adds a new mote to a topology of an infrastructure owner. The mote is linked to a gateway and gets status 'unplaced' by default.
\item Created for: UC4.3, U2 - easy mote installation
        \end{itemize}
      \item \textsf{boolean arePluggableDevicesPlaced(List\textless{}\ref{data:DataTypesPluggableDeviceID}\textgreater{} devices)}
        \begin{itemize}[noitemsep,nolistsep]
           \item Effect: Returns true if all pluggable devices in the given list have status 'placed' in the topology.
\item Created for: UC17.2, U2 - easy applications
        \end{itemize}
      \item \textsf{List\textless{}\ref{data:DataTypesRoomTopology}\textgreater{} getTopology(int custOrgID)}
        \begin{itemize}[noitemsep,nolistsep]
           \item Effect: Returns a list of RoomTopology associated to a customer organisation.
\item Created for: UC19.5, U2 - easy applications
        \end{itemize}
      \item \textsf{void reactivateDevice(\ref{data:DataTypesPluggableDeviceID} id)}
        \begin{itemize}[noitemsep,nolistsep]
           \item Effect: Changes the status of a pluggable device in the topology to 'placed'.
\item Created for: UC6.3 - reintroduced device
        \end{itemize}
      \item \textsf{void reactivateMote(int moteID)}
        \begin{itemize}[noitemsep,nolistsep]
           \item Effect: Changes the status of the mote in the topology to 'placed'. The location of the mote is unchanged, it has already been set.
\item Created for: U2 - Reintroducing a previously known mote should not require any conguration.
        \end{itemize}
    \end{itemize}
    \end{description}

  %%%%%%%% TopologyOverview
  \subsection{TopologyOverview}\label{int:OnlineServiceOnlineServiceApplicationFacadeTopologyOverview}
    \begin{description}[noitemsep,nolistsep]
      \item[Provided by:] \iconcomponent{}~\vpett{\nameref{comp:OnlineServiceOnlineServiceApplicationFacade}}
      \item[Required by:] \iconcomponent{}~\vpett{\nameref{comp:ApplicationClient}}
      \item[Operations:] ~
% no operations
    \end{description}

  %%%%%%%% UserRoleMgmt
  \subsection{UserRoleMgmt}\label{int:OtherDataDatabaseOtherDataDBUserRoleMgmt}
    \begin{description}[noitemsep,nolistsep]
      \item[Provided by:] \iconcomponent{}~\vpett{\nameref{comp:OtherDataDatabaseOtherDataDB}}
      \item[Required by:] \iconcomponent{}~\vpett{\nameref{comp:OnlineServiceOnlineServiceUserRolesManager}}
      \item[Operations:] ~
    \begin{itemize}[noitemsep,nolistsep,leftmargin=-.25cm]
      \item \textsf{boolean areMandatoryUserRolesAssigned(int applicationInstanceID)}
        \begin{itemize}[noitemsep,nolistsep]
           \item Effect: Returns true if all mandatory UserRoles for the application have been assigned to users. Finds the relevant customer organisations through the ApplicationInstance.
\item Created for: UC17.1, U2 - easy applications
        \end{itemize}
      \item \textsf{List\textless{}\ref{data:DataTypesUser}\textgreater{} getEndUsers(int custOrgID)}
        \begin{itemize}[noitemsep,nolistsep]
           \item Effect: Returns a list of Users which are associated to a customer organisation.
\item Created for: UC19.8, U2 - easy applications
        \end{itemize}
      \item \textsf{List\textless{}\ref{data:DataTypesUserRole}\textgreater{} getMandatoryUserRoles(int applicationID)}
        \begin{itemize}[noitemsep,nolistsep]
           \item Effect: Returns a list of UserRoles that which need to be assigned in order for an ApplicationInstance to run.
\item Created for: UC19.7, U2 - easy applications
        \end{itemize}
      \item \textsf{List\textless{}\ref{data:DataTypesUserRole}\textgreater{} getOptionalUserRoles(int applicationID)}
        \begin{itemize}[noitemsep,nolistsep]
           \item Effect: Returns a list of UserRoles which can optionally be assigned for an ApplicationInstance.
\item Created for: UC19.7, U2 - easy applications
        \end{itemize}
      \item \textsf{List\textless{}\ref{data:DataTypesUser}\textgreater{} getUsersWithRoles(int applicationInstanceID)}
        \begin{itemize}[noitemsep,nolistsep]
           \item Effect: Returns a list of Users associated to an ApplicationInstance that were assigned UserRoles.
\item Created for: UC17.6, U2 - easy applications
        \end{itemize}
      \item \textsf{void updateUserRoles(int applicationInstanceID, Map\textless{}int, int\textgreater{} usersAndRoles)}
        \begin{itemize}[noitemsep,nolistsep]
           \item Effect: Updates the UserRoles assigned to Users for a certain ApplicationInstance. 'usersAndRoles' maps User IDs to UserRole IDs.
\item Created for: UC19.9, U2 - easy applications
        \end{itemize}
    \end{itemize}
    \end{description}

% END INTERFACES

% EXCEPTIONS
\section{Exceptions}\label{sec:exceptions}
\begin{itemize}[nolistsep,noitemsep]
\item[] No exceptions
\end{itemize}
% END EXCEPTIONS

% DATA TYPES
\section{Data types}\label{sec:datatypes}
\begin{itemize}[nolistsep,noitemsep]
\item \vpedatatype{data:DataTypesApplication}{Application}: 
\begin{itemize}[noitemsep,nolistsep]

\item[] {\colorbox{red!30}{\underline{Undefined}}}
\end{itemize}
\item \vpedatatype{data:DataTypesApplicationInstance}{ApplicationInstance}: 
\begin{itemize}[noitemsep,nolistsep]
\item[] Attributes: int id, int status, int customerOrganisationID
\item[] Contains information on an application instance.
\end{itemize}
\item \vpedatatype{data:DataTypesDateTime}{DateTime}: 
\begin{itemize}[noitemsep,nolistsep]

\item[] Represents an instant in time, expressed as a date and time of day.
\end{itemize}
\item \vpedatatype{data:DataTypesDeviceData}{DeviceData}: 
\begin{itemize}[noitemsep,nolistsep]

\item[] Data from a pluggable device. For sensors, this contains sensor values. For actuators, this contains the state of the actuator. The data is encapsulated within a JSON message, and should be converted into something meaningful based on the device type of the pluggable device that sent the data.
\end{itemize}
\item \vpedatatype{data:DataTypesIPAddress}{IPAddress}: 
\begin{itemize}[noitemsep,nolistsep]

\item[] {\colorbox{red!30}{\underline{Undefined}}}
\end{itemize}
\item \vpedatatype{data:DataTypesMoteInfo}{MoteInfo}: 
\begin{itemize}[noitemsep,nolistsep]
\item[] Attributes: int moteID, int manufacturerID, int productID, int batteryLevel
\item[] An object containing information on a mote. This is a list of key-value pairs. The values depend on the type of mote. For example, only a battery-powered mote would include the batterylevel info.
\end{itemize}
\item \vpedatatype{data:DataTypesNotification}{Notification}: 
\begin{itemize}[noitemsep,nolistsep]
\item[] Attributes: int id, int recipientUserID, string message, int communicationChannelID, int notificationTypeID
\item[] Contains information about a notification. The communicationChannellD represents the communication channel that will be used to send the notification to the user. The notificationTypeID denotes the type of the notification (normal / alarm / ...).
\end{itemize}
\item \vpedatatype{data:DataTypesPluggableDeviceID}{PluggableDeviceID}: 
\begin{itemize}[noitemsep,nolistsep]

\item[] A unique identifier of a pluggable device.
\end{itemize}
\item \vpedatatype{data:DataTypesPluggableDeviceInfo}{PluggableDeviceInfo}: 
\begin{itemize}[noitemsep,nolistsep]
\item[] Attributes: \ref{data:DataTypesPluggableDeviceID} id, \ref{data:DataTypesPluggableDeviceType} type, Map\textless{}string, string\textgreater{} config
\item[] Contains information on a pluggable device.
\end{itemize}
\item \vpedatatype{data:DataTypesPluggableDeviceType}{PluggableDeviceType}: 
\begin{itemize}[noitemsep,nolistsep]

\item[] Denotes the type of a pluggable device.
\end{itemize}
\item \vpedatatype{data:DataTypesRelationship}{Relationship}: 
\begin{itemize}[noitemsep,nolistsep]

\item[] {\colorbox{red!30}{\underline{Undefined}}}
\end{itemize}
\item \vpedatatype{data:DataTypesRoomTopology}{RoomTopology}: 
\begin{itemize}[noitemsep,nolistsep]

\item[] {\colorbox{red!30}{\underline{Undefined}}}
\end{itemize}
\item \vpedatatype{data:DataTypesSubscription}{Subscription}: 
\begin{itemize}[noitemsep,nolistsep]
\item[] Attributes: int id, int status, int customerOrganisationID, int applicationInstanceID
\item[] Contains data about a subscription by a customer organisation for an application instance. Data about period/length of the subscription is stored in invoices.
\end{itemize}
\item \vpedatatype{data:DataTypesUser}{User}: 
\begin{itemize}[noitemsep,nolistsep]

\item[] {\colorbox{red!30}{\underline{Undefined}}}
\end{itemize}
\item \vpedatatype{data:DataTypesUserRole}{UserRole}: 
\begin{itemize}[noitemsep,nolistsep]

\item[] {\colorbox{red!30}{\underline{Undefined}}}
\end{itemize}
\end{itemize}
% END TYPES

%\end{document}

\appendix
\chapter{Attribute-driven design documentation}\label{sec:add}
    \section{Introduction}
        This chapter contains our ADD log. First, we list the changes we
        made the ADD process so it fits our workflow better. The remaining part
        of this chapter is the ADD log. Decompositions 1 and 2 have been changed
        relative to phase 2a of this project, because we forgot about the given
        interfaces for gateways and pluggable devices and made up our own
        (but similar) interfaces instead. The decompositions have been updated
        to use the given interfaces.

    \section{Adapted ADD process}
        We left off step a ("Pick an element that needs to be decomposed") since we
        never really chose a single Element to decompose. Instead, we chose the drivers
        for each decomposition first and then looked at which elements/subsystems
        would require changes or which new elements we would need to satisfy those drivers.

        For component, interfaces, datatypes: we list the new ones, but refer
        to the plugin exported catalog for descriptions

        \subsection{Decomposition X: DRIVERS (Elements/Subsystem to decompose/expand)}
            We changed these titles to reflect the architectural drivers we chose first and then
            denote which elements/subsystems needed changes to satisfy the drivers.

        % \subsection{Selected architectural drivers}
        %     \paragraph{Rationale}
        %         The rationale for selecting and grouping architectural drivers is similar
        %         for each decomposition, so we left this paragraph out. \\
        %         The selection process goes as follows. Of the remaining quality attributes,
        %         choose those that have highest priority assigned to them.

        \subsection{New data types and Interfaces for child modules}
            For each decomposition, we have listed all new interfaces and data types
            that we added during the decomposition, but all details have been left out.
            We used the Visual Paradigm plugin provided by the SA team to generate
            the element catalog of chapter \ref{ch:elements-datatypes}. All details
            can be found in there. \\

            Also, since an ADD log was no longer a requirement for phase 2b, we
            have left out intermediary "OtherFunctionality" components and
            figures of diagrams. This was done to save time.

        \subsection{Verify and refine}
            We have skipped the verify and refine step because we chose to
            handle all chosen architectural drivers completely in every decomposition.
            We did not find this step to be useful after decompositions 1 and 2.

    \newpage
    \section{Decomposition 1: Av3, UC14, UC15, UC18 (SIoTIP System)}


\subsection*{Selected architectural drivers}
    The non-functional drivers for this decomposition are:
    \begin{itemize}
    	\item \emph{Av3}: Pluggable device or mote failure
    \end{itemize}

    The related functional drivers are:
    \begin{itemize}
    	\item \emph{UC14}: Send heartbeat (Av3) \\
              This use case checks whether or not motes and pluggable devices
              are still operational.
    	\item \emph{UC15}: Send notification (Av3) \\
              This use case sends a notification to a registered user.
    	\item \emph{UC18}: Check and deactivate applications (Av3) \\
              This use case deactivates any application that requires deactivation,
              because of unavailability of essential pluggable devices
              or unassigned mandatory roles.
    \end{itemize}

    \paragraph{Rationale}
        Av3 was chosen first since it has high priority and it is more relevant to
        the core of the system than the other quality requirements with high
        priority (M1 and U2).
        We believe that handling pluggable device failure/connectivity is
        more important to the whole of the system than M1 and U2, and that
        handling this first would give a stronger starting point for later ADD iterations
        than M1 or U2.


\subsection*{Architectural design}\label{sec:architectural-design}
    This section describes what needs to be done to satisfy the requirements for
    this decomposition and how involved problems/obstacles are solved.

    \paragraph{Av3: Failure detection}
        Gateway need to be able to autonomously detect failure of one of its
        connected motes and pluggable devices. This is achieved by making motes
        send heartbeats to their connected gateways. The gateways can
        then monitor their connected devices. The heartbeats contain a list
        of devices that are connected/operational at the moment the mote sends
        the heartbeat. Each gateway makes use of a \texttt{DeviceManager}
        component to monitor the devices. This component uses timers to keep track
        of how long it has been since a device has sent a heartbeat or occured in
        a list of connected devices. Once a timer expires, this is treated as
        a failure. \\

        A mote has failed when 3 consecutive heartbeats do not arrive within 1
        second of their expected arrival time. \\
        A pluggable device has failed when it does not occur in a heartbeat of the
        mote in which it is expected to be in. This is is detected within 2
        seconds after the arrival of the heartbeat.

    \paragraph{Av3: Automatic application deactivation and redundancy settings}
        Applications should be automatically suspended when they can no longer
        operate due to failure of a pluggable device or mote and reactivated
        once the failure is resolved. Application providers can design their
        applications such that they explicitly require redundancy in
        the available pluggable devices. \\
        This problem is tackled by the \texttt{DeviceManager}. It
        stores the requirements for pluggable devices set by applications for all
        applications that use the gateway that the the \texttt{DeviceManager}
        runs on. When it detects that an application can no longer operate
        due to failures, it will send a command to the \texttt{ApplicationManager}
        (via the \texttt{GatewayFacade})
        to suspend that application. When the required devices are operational
        again, the \texttt{DeviceManager} detects this and sends a
        command to reactivate the application. \\

        Applications are suspended within 1 minute after detecting
        the failure of an essential pluggable device. \\
        Application are reactivated within 1 minute after the failure is resolved.

    \paragraph{Av3: Notifications}
        The infrastructure owner should be notified of any persistent
        pluggable device or mote failures. Customer organisations should be
        notified if one or more of their applications is suspended or
        reactivated. Applications using a failed pluggable device or any device
        on a failed mote should be notified. \\
        The \texttt{NotificationHandler} was put in place to deal with
        notifications. Other components can use it to generate notifications for
        certain users in the system. The \texttt{NotificationHandler} will then
        insert information relevant to the notification in the database (message,
        status, date and time, source, ...), and use an external delivery
        service to deliver the notification to users. The used delivery medium
        is based on the user's preferences. \\
        Since they are stored in the database, users can always view
        their notifications via their dashboard. However, this funcionality is not
        expanded on in this decomposition yet. \\

        Infrastructure owners are notified within 1 minute after detecting a mote outage lasting at
        least 10 seconds. \\
        Infrastructure owners are notified within 1 minute after the detection of the unavailability of
        a pluggable device for 30 seconds. \\
        Applications are notified of the failure of relevant pluggable devices within 10 seconds.

    \subsubsection{Alternatives considered}
        \paragraph{Av3: Failure detection}
            An alternative would have been to move the \texttt{DeviceManager}
            component from gateways to the Online Service. This solution would make the
            gateways do less work, but would be very unscalable. The reason is
            that as the customer base (and thus the amount of devices) increases,
            the Online Service would need to keep track of huge amounts of devices.
            This would also flood the network to the Online Service with heartbeats.

        \paragraph{Av3: Failure detection}
            Another alternative for failure detection could have been the use of
            a Ping/Echo mechanism instead of Heartbeats. Pings could then be used
            to check if a device is currently operational. However, as a device could
            not be operational for a moment because of e.g. interference, timers
            would still be necessary to keep track of operational devices. We opted
            to use heartbeats, as this would reduce the amount of data sent over
            the network used by the motes, and as motes would have to do slightly
            more work to process each Ping request in order to generate a reply.

        \paragraph{Av3: Notifications}
            Reliable and quick delivery of notifications is crucial to the
            system in order to solve problems should things go wrong. Currently,
            the solution is to use a third party service for delivery of
            notifications. In the case that no external services are found
            satisfactory, or if this dependency on an external service is
            unwanted, it is possible to build an internal solution for this.
            For example, a \texttt{NotificationSender} component could make use
            of the \texttt{Factory pattern} for different message channels for
            different delivery methods (each with their own sendNotification method).
            This solution allows us to easily add new message channels in the
            future with little effort. The disadvantage of this is that an
            internal solution takes a lot more time to implement.


\subsection*{Instantiation and allocation of functionality}
    This section lists the new components which instantiate our solutions
    described in the section above. For each component we note the quality
    attribute or use case that prompted us to create it. Descriptions about
    the components can be found under chapter \ref{ch:elements-datatypes}. \\

    \begin{itemize}
        \item ApplicationManager: Av3
        \item Database: /
        \item DeviceManager: Av3
        \item GatewayFacade: /
        \item Mote: UC14
        \item NotificationHandler: UC15
    \end{itemize}

    \paragraph{Decomposition}
        Figure \ref{fig:it1-cc_main} shows the components resulting from the
        decomposition in this run.

        \begin{figure}[!htp]
        	\centering
        	\includegraphics[width=1.00\textwidth]{images/component-diagram-1}
        	\caption{Component-and-connector diagram of this decomposition.}
            \label{fig:it1-cc_main}
        \end{figure}

    \paragraph{Deployment}
        Figure \ref{fig:it1-depl_main} shows the allocation of components
        to physical nodes.

        \begin{figure}[!htp]
        	\centering
        	\includegraphics[width=0.55\textwidth]{images/deployment-diagram-1}
        	\caption{Deployment diagram of this decomposition.}\label{fig:it1-depl_main}
        \end{figure}


\subsection*{Interfaces for child modules}\label{add1-interfaces}
    This section lists new interfaces assigned to the components defined
    in the section above. Detailed information about each interface and
    its methods can be found under chapter \ref{ch:elements-datatypes}.

    \subsubsection{ApplicationManager}
        \begin{itemize}
            \item GWAppInstanceMgmt
        \end{itemize}

    \subsubsection{Database}
        \begin{itemize}
            \item NotificationMgmt
            \item AppMgmt
        \end{itemize}

    \subsubsection{GatewayFacade}
        \begin{itemize}
            \item Heartbeat
            \item DeviceData
            \item DeviceMgmt
            \item AppDeviceMgmt
        \end{itemize}

    \subsubsection{Mote}
        \begin{itemize}
            \item DeviceMgmt
        \end{itemize}

    \subsubsection{NotificationHandler}
        \begin{itemize}
            \item Notify
            \item DeliveryMgmt
        \end{itemize}

    \subsubsection{External notification delivery service}
        \begin{itemize}
            \item NotificationDeliveryMgmt
        \end{itemize}

    \subsubsection{DeviceManager}
        \begin{itemize}
        	\item DeviceMgmt
        \end{itemize}


\subsection*{New data types}
    This section lists the data types introduced in this decomposition.

    \begin{itemize}
        \item{PluggableDeviceInfo}
        \item{Notification}
        \item{ApplicationInstance}
        \item{Subscription}
        \item{PluggableDeviceID}
        \item{PluggableDeviceType}
        \item{DeviceData}
        \item{Map<String,String>}
    \end{itemize}

\subsection*{Verify and refine}
    The selected architectural drivers have been handled completely
    in this decomposition.
    This section describes per component which (parts of) the remaining
    requirements it is responsible for. If requirements are split in
    multiple parts, this is indicated by the addition of a letter
    (or number, depending on the structure of the requirement) after their title.

    \paragraph{ApplicationManager}
        \begin{itemize}
            \item \emph{Av2}: Application failure \\
                   Prevention: a, b \\
                   Detection: a, b, c \\
                   Resolution: a, b, c
           \item \emph{P1}: Large number of users: c
           \item \emph{M1}: Integrate new sensor or actuator manufacturer: 1.c, 2.a
           \item \emph{M2}: Big data analytics on pluggable data and/or application usage data: d, e
           \item \emph{U1}: Application updates: a, b, c, d
           \item \emph{U2}: Easy Installation: e
           \item \emph{UC12}: Perform actuation command
           \item \emph{UC17}: Activate an application: 3, 4
        \end{itemize}

    \paragraph{Database}
        \begin{itemize}
          	\item None
        \end{itemize}

    \paragraph{GatewayFacade}
        \begin{itemize}
            \item \emph{Av1}: Communication between SIoTIP gateway and Online Service \\
                              Resolution: b, c, d
            \item \emph{M1}: Integrate new sensor or actuator manufacturer: 1.a, 2.b
            \item \emph{U2}: Easy Installation: a, c, d
            \item \emph{UC11}: Send pluggable device data: 1
        \end{itemize}

    \paragraph{Mote}
        \begin{itemize}
            \item \emph{M1}: Integrate new sensor or actuator manufacturer: 1.a, 2.b
            \item \emph{U2}: Easy Installation: b, c, d
            \item \emph{UC04}: Install mote: 1, 2
            \item \emph{UC05}: Uninstall mote: 1
            \item \emph{UC06}: Insert a pluggable device into a mote: 2
            \item \emph{UC07}: Remove a pluggable device from its mote: 2
            \item \emph{UC11}: Send pluggable device data: 1
        \end{itemize}

    \paragraph{NotificationHandler}
        \begin{itemize}
            \item \emph{UC16}: Consult notification message: 5
            \item \emph{UC17}: Activate an application: 5, 6
        \end{itemize}

    \paragraph{OtherFunctionality1}
        \begin{itemize}
            \item \emph{Av1}: Communication between SIoTIP gateway and Online Service \\
                               Detection: a, b, c, d
                               Resolution: a
           	\item \emph{P1}: Large number of users: a
            \item \emph{P2}: Requests to the pluggable data database
            \item \emph{M1}: Integrate new sensor or actuator manufacturer: 1.d
            \item \emph{M2}: Big data analytics on pluggable data and/or application usage data: a
            \item \emph{U2}: Easy Installation: e
            \item \emph{UC01}: Register a customer organisation
            \item \emph{UC02}: Register an end-user
            \item \emph{UC03}: Unregister an end user
            \item \emph{UC04}: Install mote: 3
            \item \emph{UC05}: Uninstall mote: 2.b
            \item \emph{UC06}: Insert a pluggable device into a mote: 3: topology part; alternative 3a.1.b
            \item \emph{UC07}: Remove a pluggable device from its mote: 3.b
            \item \emph{UC08}: Initialise a pluggable device: 1, 2, 4
            \item \emph{UC09}: Configure pluggable device access rights
            \item \emph{UC10}: Consult and configure the topology
            \item \emph{UC11}: Send pluggable device data: 3
            \item \emph{UC13}: Configure pluggable device
            \item \emph{UC16}: Consult notification message: 1, 2, 3, 4
            \item \emph{UC17}: Activate an application: 1, 2
            \item \emph{UC19}: Subscribe to application
            \item \emph{UC20}: Unsubscribe from application
            \item \emph{UC21}: Send invoice
            \item \emph{UC22}: Upload an application
            \item \emph{UC23}: Consult application statistics
            \item \emph{UC24}: Consult historical data
            \item \emph{UC25}: Access topology and available devices
            \item \emph{UC26}: Send application command or message to external front-end
            \item \emph{UC27}: Receive application command or message to external front-end
            \item \emph{UC28}: Log in
            \item \emph{UC29}: Log out
        \end{itemize}

    \paragraph{DeviceManager}
        \begin{itemize}
            \item \emph{U2}: Easy Installation: c, d
            \item \emph{UC04}: Install mote: 4
            \item \emph{UC05}: Uninstall mote: 2
            \item \emph{UC06}: Insert a pluggable device into a mote: 3: uninitialised part; alternative 3a.1 3a.2 3a.4; 4
            \item \emph{UC07}: Remove a pluggable device from its mote: 3.a, 3.c
            \item \emph{UC08}: Initialise a pluggable device: 3
            \item \emph{UC11}: Send pluggable device data: 2, 3a
        \end{itemize}

    \newpage
    \section{Decomposition 2: OtherFunctionality (M1, P2, UC11)}

\subsection{Module to decompose}
    In this run we decompose OtherFunctionality.


\subsection{Selected architectural drivers}
    The non-functional drivers for this decomposition are:
    \begin{itemize}
    	\item \emph{M1}: Integrate new sensor or actuator manufacturer
        \item \emph{P2}: Requests to the pluggable data database
    \end{itemize}

    \noindent The related functional drivers are:
    \begin{itemize}
        \item \emph{UC11}: Send pluggable device data (P2) \\
              This use case stores pluggable device data in the pluggable device data storage.
              This could be a sensor reading, or an actuator status.
    \end{itemize}

    \paragraph{Rationale}
    We choose M1 because it belogs to the quality attributes with hight priority. M1 is about
    integration of new sensor or actuator. And it is very important to easily add new devices, because market grows very fast
    and new applications are developing. So we want to focus on this quality attribute
    in the early stages and then based on that create other functionality and components.
    And we also choose P2 because it is related to M1. M1 required minimal changes to data processing and storage,
    so we have to deal with good solution for this topic.  
        
     %Why we do dis???? One was high priority and P2 is related. THey are family and family belongs together.


\subsection{Architectural design}
    % Tactics:
    %     Limit event response? reply within response measure deadlines
    %     Prioritize events
    %     Introduce concurrency
    %     Schedule resources

    \paragraph{Handling new types of pluggable devices for M1}
        The developers have to make changes to: component1, component2, datatype X.
        The new type of sensor needs to be able to be initialised so that it can send data.
        Thus, the PluggableDeviceFacade code that initialises devices should be updated for
        each new type of sensor. The PluggableDeviceData datatype should be updated to
        represent the new type of data. In this case, the new type will have to be added
        to the database that contains all different types of sensor data.

    \paragraph{Data conversions for M1}
        \texttt{The PluggableDeviceDataConverter} is resposible for converting data 
         in system, for instance converting temperature in degrees Fahrenheit 
         to degrees Celsius. System has to work with relevant data, 
         otherwise problem may arise. 
         

    \paragraph{Usage of new data by applications for M1}
        This is possible through the RequestData interface provided by PluggableDeviceDataScheduler.
        The application manager can get device data from the PluggableDeviceDB and return this
        data to applications in the PluggableDeviceData datatype. This datatype can easily be
        updated for new types of pluggable devices.

    \paragraph{Configuration of new device by infrastructure owners for M1}
        Initialisation: IO triggers the initialise() function which has been
        updated for the new pluggable device -> OK\\
        Configure access rights: has absolutely fucking nothing to do with the
        new sensor type -> OK \\
        Consult and configure topology: same as configure access rights

    \paragraph{Scheduling for P2}
        dynamic priority scheduling \\
        tactics: schedule resource, prioritize events, also limit event response?\\
        starvation avoidance

    \paragraph{Pluggable data separation for P2}
        "pluggable data has no impact on other data"
        two databases

    \subsubsection{Alternatives considered}
        \paragraph{Alternatives for solution}
            A discussion of the alternative solutions and why that were not selected.


\subsection{Instantiation and allocation of functionality}
    \paragraph{Decomposition}
        Main aspects of the resulting decomposition.

        \begin{figure}[!htp]
        	\centering
            \includegraphics[width=1.00\textwidth]{component-diagram-2}
        	\caption{Component-and-connector diagram of this decomposition.}
            \label{fig:it1-cc_main}
        \end{figure}

    \subparagraph{PluggableDeviceDB}
        store data related to pluggable devices

    \subparagraph{PluggableDeviceDataScheduler}
        scheduling, detect overload mode, store data, forward data

    \subparagraph{PluggableDeviceDataConverter}
        M1: conversion of new type of data of new type of device

    % \paragraph{Behaviour}
        % A SEQUENCE DIAGRAM FOR UC11 WOULD BE ACTUALLY VERY USEFUL (shows how the gateway checks if devices are initialised)
        % REMOVE THIS PART BECAUSE MONEYKA IS LAAAAZZZZYYYYYY BAD STUDENT "IT IS NOT NECESSARY"

    \paragraph{Deployment}
        Rationale of the allocation of components to physical nodes.

        \begin{figure}[!htp]
        	\centering
        	\includegraphics[width=1.00\textwidth]{deployment-diagram-2}
        	\caption{Deployment diagram of this decomposition.
        	}\label{fig:it1-depl_main}
        \end{figure}


\subsection{Interfaces for child modules}

    \subsubsection{GatewayFacade}
        See "\ref{ADD1-int-gatewayfacade}: GatewayFacade" for the rest of the interfaces provided by this component.
        \begin{itemize}
            \item MoteDataMgmt
            \begin{itemize}
                \item \texttt{void sendData(PluggableDeviceData data)}
                \begin{itemize}
                    \item Effect: Sends pluggable device data to the connected mote.
                    \item Exceptions: None
                \end{itemize}
            \end{itemize}

            \item DeviceMgmt
            \begin{itemize}
                \item \texttt{void initialiseDevice(int deviceID, PluggableDeviceSettings settings)}
                \begin{itemize}
                    \item Effect: Initialises a pluggable device for use with the system.
                    \item Exceptions: None
                \end{itemize}
            \end{itemize}

            \item AppDeviceMgmt
            \begin{itemize}
                \item \texttt{void configurePluggableDevice(int deviceID, PluggableDeviceSettings settings)}
                \begin{itemize}
                    \item For: Use case 11 step 3.b
                    \item Effect: Causes certain settings to be set on a pluggable
                          device that the gateway is connected to.
                    \item Exceptions: None
                \end{itemize}
            \end{itemize}
        \end{itemize}

    \subsubsection{MoteFacade}
        See "\ref{add1-int-motefacade}: MoteFacade" for the rest of the interfaces provided by this component.
        \begin{itemize}
            \item PluggableDeviceDataMgmt
            \begin{itemize}
                \item \texttt{void sendData(PluggableDeviceData data)}
                \begin{itemize}
                    \item Effect: Sends pluggable device data to the connected mote.
                    \item Exceptions: None
                \end{itemize}
            \end{itemize}

            \item PluggableDeviceMgmt
            \begin{itemize}
                \item \texttt{void initialise(int deviceID, PluggableDeviceSettings settings)}
                \begin{itemize}
                    \item Effect: Initialises a connected pluggable device according to some settings
                    \item Exceptions: None
                \end{itemize}
            \end{itemize}
        \end{itemize}

    \subsubsection{PluggableDeviceFacade}
        \begin{itemize}
        	\item PluggableDeviceMgmt
        	\begin{itemize}
                \item \texttt{void initialise(PluggableDeviceSettings settings)}
                \begin{itemize}
                    \item Effect: Initialises the pluggable device according to some settings
                    \item Exceptions: None
                \end{itemize}
        	\end{itemize}
        \end{itemize}

    \subsubsection{PluggableDeviceManager}
        \begin{itemize}
        	\item DeviceListMgmt
        	\begin{itemize}
        		\item \texttt{bool isDeviceInitialised(int deviceID)}
        		\begin{itemize}
        			\item Effect: Returns true if the device with id "deviceID" has been initialized.
        			\item Exceptions: None
        		\end{itemize}
        	\end{itemize}
        \end{itemize}

    \subsubsection{PluggableDeviceDataScheduler}
        \begin{itemize}
            \item RequestData
            \begin{itemize}
                \item \texttt{List<PluggableDeviceData> requestData(int applicationID, int deviceID, DateTime from, DateTime to)}
                \begin{itemize}
                    \item Effect: Request data from a specific device in a certain time period
                    \item Exceptions: None
                \end{itemize}
            \end{itemize}

            \item PluggableDeviceDataMgmt
            \begin{itemize}
                \item \texttt{void sendData(PluggableDeviceData data)}
                \begin{itemize}
                    \item Effect: Sends pluggable device data to the scheduler to be processed.
                    \item Exceptions: None
                \end{itemize}
            \end{itemize}
        \end{itemize}

    \subsubsection{PluggableDeviceDB}
        \begin{itemize}
            \item PluggableDeviceDataMgmt
            \begin{itemize}
                \item \texttt{void sendData(PluggableDeviceData data)}
                \begin{itemize}
                    \item Effect: Sends pluggable device data to the DB to be stored.
                    \item Exceptions: None
                \end{itemize}
                \item \texttt{List<PluggableDeviceData> getData(int deviceID, DateTime from, DateTime to)}
                \begin{itemize}
                    \item Effect: Returns data from a specific device in a certain time period.
                    \item Exceptions: None
                \end{itemize}
                \item \texttt{List<int> getApplicationsForDevice(int deviceID)}
                \begin{itemize}
                    \item Effect: Returns a list of applications that can use the device with id "deviceID."
                    \item Exceptions: None
                \end{itemize}
            \end{itemize}
        \end{itemize}


\subsection{Data type definitions}
    \paragraph{DateTime} Represents an instant in time, typically expressed as a date and time of day.


\subsection{Verify and refine}
    Completely handled: M1, P2, UC11 \\

    \noindent This section describes per component which (parts of) the remaining
    requirements it is responsible for.

    \paragraph{ApplicationManager}
        \begin{itemize}
            \item  \emph{Av2}: Application failure \\
                   Prevention: a, b \\
                   Detection: a, b, c \\
                   Resolution: a, b, c
           \item \emph{P1}: Large number of users: c
           \item \emph{M2}: Big data analytics on pluggable data and/or application usage data: d, e
           \item \emph{U1}: Application updates: a, b, c, d
           \item \emph{U2}: Easy Installation: e
           \item \emph{U12}: Perform actuation command
           \item \emph{UC17}: Activate an application: 3, 4
        \end{itemize}

    \paragraph{Database}
        \begin{itemize}
          	\item None
        \end{itemize}

    \paragraph{GatewayFacade}
        \begin{itemize}
            \item \emph{Av1}: Communication between SIoTIP gateway and Online Service \\
                               Resolution: b, c, d
            \item \emph{U2}: Easy Installation: a, c, d
        \end{itemize}

    \paragraph{MoteFacade}
        \begin{itemize}
            \item \emph{U2}: Easy Installation: b, c, d
            \item \emph{UC4}: Install mote: 1, 2
            \item \emph{UC5}: Uninstall mote: 1
            \item \emph{UC6}: Insert a pluggable device into a mote: 2
            \item \emph{UC7}: Remove a pluggable device from its mote: 2
        \end{itemize}

    \paragraph{NotificationHandler}
        \begin{itemize}
            \item \emph{UC16}: Consult notification message: 5
            \item \emph{UC17}: Activate an application: 5, 6
        \end{itemize}

    \paragraph{OtherFunctionality}
        \begin{itemize}
            \item \emph{Av1}: Communication between SIoTIP gateway and Online Service \\
                               Detection: a, b, c, d
                               Resolution: a
           	\item \emph{P1}: Large number of users: a
            \item \emph{M2}: Big data analytics on pluggable data and/or application usage data: a
            \item \emph{U2}: Easy Installation: e
            \item \emph{UC1}: Register a customer organisation
            \item \emph{UC2}: Register an end-user
            \item \emph{UC3}: Unregister an end user
            \item \emph{UC4}: Install mote: 3
            \item \emph{UC5}: Uninstall mote: 2.b
            \item \emph{UC6}: Insert a pluggable device into a mote: 3: topology part; alternative 3a.1.b
            \item \emph{UC7}: Remove a pluggable device from its mote: 3.b
            \item \emph{UC8}: Initialise a pluggable device: 1, 2, 4
            \item \emph{UC9}: Configure pluggable device access rights
            \item \emph{UC10}: Consult and configure the topology
            \item \emph{UC13}: Configure pluggable device
            \item \emph{UC16}: Consult notification message: 1, 2, 3, 4
            \item \emph{UC17}: Activate an application: 1, 2
            \item \emph{UC19}: Subscribe to application
            \item \emph{UC20}: Unsubscribe from application
            \item \emph{UC21}: Send invoice
            \item \emph{UC22}: Upload an application
            \item \emph{UC23}: Consult application statistics
            \item \emph{UC24}: Consult historical data
            \item \emph{UC25}: Access topology and available devices
            \item \emph{UC26}: Send application command or message to external front-end
            \item \emph{UC27}: Receive application command or message to external front-end
            \item \emph{UC28}: Log in
            \item \emph{UC29}: Log out
        \end{itemize}

    \paragraph{PluggableDeviceDB}
        \begin{itemize}
            \item \emph{M2}: Big data analytics on pluggable data and/or application usage data: b
        \end{itemize}

    \paragraph{PluggableDeviceFacade}
        \begin{itemize}
        	\item \emph{U2}: Easy Installation: d
        \end{itemize}

    \paragraph{PluggableDeviceManager}
        \begin{itemize}
            \item \emph{U2}: Easy Installation: c, d
            \item \emph{UC4}: Install mote: 4
            \item \emph{UC5}: Uninstall mote: 2
            \item \emph{UC6}: Insert a pluggable device into a mote: 3: uninitialised part; alternative 3a.1 3a.2 3a.4; 4
            \item \emph{UC7}: Remove a pluggable device from its mote: 3.a, 3.c
            \item \emph{UC8}: Initialise a pluggable device: 3,
        \end{itemize}

    \paragraph{PluggableDeviceDataScheduler}
        \begin{itemize}
            \item \emph{P1}: Large number of users: b
            \item \emph{M2}: Big data analytics on pluggable data and/or application usage data: b, c
        \end{itemize}

    \newpage
    \section{Decomposition 3: U2, UC4, UC6, UC9, UC10, UC17, UC19.7-11}

\subsection{Elements/Subsystem to decompose/expand}
    In this run we decompose/expand ...


\subsection{Selected architectural drivers}
    The non-functional drivers for this decomposition are:
    \begin{itemize}
    	\item \emph{U2}: Easy installation
    \end{itemize}

    The related functional drivers are:
    \begin{itemize}
        \item \emph{UC4}: Install mote \\
              Short description
        \item \emph{UC6}: Insert a pluggable device into a mote \\
              Short description
        \item \emph{UC9}: Configure pluggable device access rights \\
              Short description
        \item \emph{UC10}: Consult and configure topology \\
              Short description
        \item \emph{UC17}: Activate an application \\
              Short description
        \item \emph{UC19}: Subscribe to application \\
              Short description
    \end{itemize}


\subsection{Architectural design}
    This section describes what needs to be done to satisfy the requirements for
    this decomposition and how involved problems/obstacles are solved.

    topology
        gateways(id, floor, room_number, wall, status)
        motes(id, gateway_id, floor, room_number, status)
        pluggable_devices(id, mote_id, gateway_id, status, physical location)
        pluggable_devices_redundancies(deviceID1, deviceID2)

    access rights
        rights(id, origranisation_id, device_id, can_read, can_configure, can_send_actuation_command)

    \paragraph{U2: Gateway installation}
        The gateway should not require any configuration, other than being connected
        to the local wired or WiFi network, after it is plugged into an electrical
        socket. An infrastructure owner should be able get the SIoTIP gateway
        up-and-running (connected) within 10 minutes given that the information
        (e.g. WiFi SSID and passphrase) is available to the person responsible for
        the installation. \\
        TODO: ask \\
        We need something that registers the gateway automatically with the
        online service after bootup. A connection to the internet is a constraint
        of the GatewayFacade.

    \paragraph{U2: Mote installation}
        Installing a new mote should not require more configuration than adding it
        to the topology. Adding new motes, sensors or actuators should not involve
        more than just starting motes, and plugging devices into motes – plug-and-play! TODO: ask \\
        Reintroducing a previously known mote, with the same pluggable devices attached to it,
        should not require any configuration. It is automatically re-added on
        its last known location on the topology. The attached pluggable devices
        are automatically initialised and configured with their last known
        configuration and access rights. \\
        Thing that need to happen automatically:
        *) mote should find the gateway (mote sends a broadcast message->ReceiveBroadcast)
        *) gateway should register the mote (DeviceManager update, store entry in DB)
        *) on reintroduction of motes: DeviceManager notices this, makes the gateway send a message to online service to reuse some old topology

        UC4:
            Remark : The mote is pre-congured to connect to a specic gateway by
             the hardware manufacturer. This linking process is out of scope for
             this assignment. Likewise, the automatic assignment of an IPv6 address
             to the mote is out of scope.

            if new mote:
                for step 2., we can use the heartbeat system. heartbeat is sent from mote to gateway,
                the DeviceManager in the gateway notices that this is from a new mote and starts
                the registerMote procedure
                FOR RATIONALE: The IPAddress of the mote can be parsed out of the 6lowpan header in the heartbeat messages
                ALTERNATIVE IS GatewayFacade: registerMote, but this is more work (battery power and implementation)
                % DONT NEED THIS ONE 2. MoteFacade -> GatewayFacade: registerMote(moteID, IPAddress moteIPAddress)
                % ALSO DONT NEED THIS ONE 3. GatewayFacade -> DeviceManager: registerMote(moteID, IPAddress moteIPAddress)
                % 3. DeviceManager -> GatewayFacade -> PluggableDeviceDataScheduler: addMote(moteID, gatewayID, IPAddress moteIPAddress)
                                                    %  PluggableDeviceDataScheduler -> PluggableDeviceDB: addMote(moteID, gatewayID, IPAddress moteIPAddress)
                                                %   -> TopologyManager: addMote(infrastructureOwnerID, gatewayID, moteID) // status = unplaced
                                                    %  TopologyManager -> PluggableDeviceDataScheduler -> PluggableDeviceDB: addMoteInTopology(infrastructureOwnerID, gatewayID, moteID)
                % 4. DeviceManager -> GatewayFacade -> NotificationHandler: notify(infrastructureOwnerID, message)

        if reintroduced mote:
            It is automatically re-added on its last known location on the topology.
                % 3. DeviceManager -> GatewayFacade -> PluggableDeviceDataScheduler: reactivateMote(moteID)
                % 3. DeviceManager -> GatewayFacade -> TopologyManager: reactivateMote(moteID) // status = placed? (TODO ASK?), location is still there from the past, it was not removed

            The attached pluggable devices are automatically initialised and configured with their last known configuration.
            The attached pluggable devices are automatically initialised and configured with their last known access rights.
                Already done by DeviceManager, it detects the devices, updates DB, and configures the devices
                % 3. DeviceManager -> GatewayFacade -> PluggableDeviceDataScheduler: reactivateDevice(deviceID)


    \paragraph{U2: Pluggable device installation}
        Adding new sensors or actuators should require no further customer
        actions besides plugging it into the mote. Configurable sensors and
        actuators should have a working default configuration.
        Pluggable devices added to an already known mote are automatically
        added in the right location on the topology.
        Making (initialised) sensors and actuators available to customer
        organisations and applications should not require more effort than
        configuring access rights (cf. UC9). \\
        *) After devices are plugged in: connect to mote, set up default configurations
        *) if the mote is already known, the device is added to the right location on the topology
        *) need something for configuration of access rights, can only happen for initialised devices

        *) for reactivating last configurations: just set status to active and don't change configuration field, it will still be the same as in the past
            alternative: current_configuration and last_configuration in DB
            alternative: store all configurations on Gateway -> but it has bad resources
            alternative: store all versions on PluggableDeviceDB -> but lots of useless data then = extra work for db

        *) Pluggable devices added to an already known mote are automatically added in "the right location" on the topology.
            what exactly is a location?
            => when a pluggable device is connected to a new mote, the pluggable device gets the location of the mote by default

        UC6: insert a pluggable device into a mote
            mote is already installed

            when device is plugged into a mote:
                mote -> DeviceManager: registerDevice(id, type) (registers device as uninitialised)
                DeviceManager -> PluggableDeviceDatabase: addDevice(id, type, status) (status = uninitialised)
                DeviceManager -> TopologyManager: addDeviceToMote(deviceID, moteID, status) (status = unplaced) (sets the location to the same of the mote)
                DeviceManager -> NotificationHandler: newPluggableDevice

                In these methods, if the device already exists and is plugged in in another mote, clear the data

            if the device is a known previously active device (ON THE SAME MOTE):
                mote -> DeviceManager: registerDevice(id, type)

                ∗ marks the pluggable device as ‘active’: DB pluggable_devices
                    DeviceManager -> PluggableDeviceDataScheduler: reactivateDevice(deviceID)

                ∗ updates the topology: DB topology_pluggable_devices
                    DeviceManager -> TopologyManager: reactivateDevice(deviceID)

                ∗ configures the pluggable device with the last known access rights: DB permissions_pluggable_devices
                    % DeviceManager -> DeviceAccessRightsManager: reactivate(deviceID)
                    nothing needs to happen here, permission information will just not be used if the device is inactive
                                                  if the device is reactivated, the permissions are already there

                * configures the pluggable device with the last known configuration: DB pluggable_devices
                    DeviceManager -> PluggableDevice: setConfiguration(Map<String, String> lastKnownConfiguratoin) (lastKnown.. taken from DB)

                ∗ checks and activates applications which can now execute again:
                    DeviceManager -> ApplicationManager: checkPluggableDevices() something like this

                * send notification
                    DeviceManager -> NotificationHandler: reactivatedPluggableDevice

        UC9: configure pluggable device access rights
            1. InfrastructureOwnerClient -> InfrastructureOwnerFacade: getAccessRights()
                2. InfrastructureOwnerFacade -> DeviceManager: getPluggableDevices(infrastructureOwnerID)
                3. InfrastructureOwnerFacade -> AccessRightsManager: getAccessRights(pluggableDeviceID)
                4. InfrastructureOwnerFacade -> OtherFunctionality: getCustomerOrganisations(infrastructureOwnerID)
            6. InfrastructureOwnerClient -> InfrastructureOwnerFacade: setAccessRights/updateAccessRights(pluggableDeviceID, List<int> customerOrganisations)
                6. InfrastructureOwnerFacade -> DeviceManager: configureAccessRights(pluggableDeviceID, List<int> customerOrganisations)
                7.a. AccessRightsManager -> PluggableDeviceDatabase: updateDeviceAccessRights(deviceID, custOrg) / deleteAccessRights()
                7.b. AccessRightsManager -> ApplicationManager: checkAccessRights(custOrg)
                7.c. AccessRightsManager -> ApplicationManager: checkPluggableDevices(custOrg)


    \paragraph{U2: Easy applications}
        Applications should work out of the box if the required sensors and
        actuators are available. Only when mandatory end-user roles must be
        assigned, additional explicit configuration actions are required
        from a customer organisation (cf. UC17, UC19). \\
        *) if there is a subsription and new hardware is plugged in: need something to check
           if some application can be activated now
        *) need something to assign user roles to users during UC19

        UC17:
            application needs to be activated because new subscription, changed topology, new version of application

            ApplicationManager is triggered by something else do to the following: method activateApplication(applicationInstanceID)
            1. ApplicationManager -> UserRolesManager: checkMandatoryUserRoles(applicationInstanceID)
            2. ApplicationManager -> DeviceManager: checkPluggableDevices(applicationInstanceID)
            3. ApplicationManager -> GatewayFacade: activateApplication(applicationInstanceID) -> ApplicationSandbox component on gateway
            4. application.start();
            4. ApplicationManager -> InvoiceManager: activatedApplication(applicationInstanceID, custOrgID, date)
            5. ApplicationManager -> NotificationManager: activatedApplication(custOrgID)
            6. ApplicationManager -> UserRolesManager: List<User> getUsersWithRoles(custOrgID)
            6. ApplicationManager -> getInstallationInstructions(applicationID)
            6. ApplicationManager -> NotificationManager: notify(userID, message)

            TODO notifications for alternative scenario's

        UC19:
            2. CustomerOrganisationClient -> CustomerOrganisationFacade: List<Application> getApplicationsToSubscribe(custOrgID)
                   CustomerOrganisationFacade -> ApplicationManager: List<Application> getApplicationsToSubscribe(custOrgID)
            4. CustomerOrganisationClient -> CustomerOrganisationFacade: subscribeToApplication(custOrgID, applicationID)
            5. CustomerOrganisationFacade -> ApplicationManager: getMandatoryDevicesAndTopologyConfigurations(applicationID)
            6. TODO: The primary actor carries out the topology configuration ??????????????????????????????????????????????????????????????????????????????????
            7. CustomerOrganisationFacade -> ApplicationManager: getAllUserRoles(applicationID)
            8. CustomerOrganisationFacade -> UserRolesManager: getAllUsers(custOrgID)
            9. CustomerOrganisationClient -> CustomerOrganisationFacade: setUserRoles(custOrgID, map<String, String> userRoles)
                   CustomerOrganisationFacade -> UserRolesManager: setUserRoles(custOrgID, map<String, String> userRoles)
                   10. <- return value is next page for selection of criticality
            11. CustomerOrganisationClient -> CustomerOrganisationFacade: setCriticality(applicationID, bool isCritical)
                    12. CustomerOrganisationFacade -> ApplicationManager: setCriticality(applicationID, bool isCritical)
            13. CustomerOrganisationFacade -> SubscriptionManager: subscribe(customerOrganisationID, applicationID) // returns applicationInstanceID of new application instance, if the org is subscribed to a older version, automatically unsubscribe
            14. CustomerOrganisationFacade -> ApplicationManager: activateApplication(applicationInstanceID)


\subsection{Instantiation and allocation of functionality}
    This section describes the new components which instantiate our solutions described
    in the section above and how components are deployed on physical nodes. \\
    Unless stated otherwise the responsibilities assigned in the first decomposition are unchanged.

    \paragraph{Decomposition}
        Figure \ref{fig:FIGURELABEL} shows the components resulting from the
        decomposition in this run.

        \begin{figure}[!h]
        	\centering
            %\includegraphics[width=1\textwidth]{IMAGE FILE NAME}
        	\missingfigure[figwidth=0.8\textwidth]{Component-and-connector diagram of this decomposition}
        	\caption{Component-and-connector diagram of this decomposition.}
            \label{fig:FIGURELABEL}
        \end{figure}

        The responsibilities of the components are as follows:

    \subparagraph{Component}
        Short description of its responsibilities. (Relevant QA or UC)


    % \paragraph{Behaviour}
        % USEFUL SEQUENCE DIAGRAMS FOR CHOSEN USE CASES


    \paragraph{Deployment}
        Figure \ref{fig:FIGURELABEL} shows the allocation of components
        to physical nodes.

        \begin{figure}[!h]
        	\centering
        	%\includegraphics[width=0.8\textwidth]{IMAGE FILE NAME}
        	\missingfigure[figwidth=0.8\textwidth]{Deployment diagram}
        	\caption{Deployment diagram of this decomposition.}
            \label{fig:FIGURELABEL}
        \end{figure}


\subsection{Interfaces for child modules}\label{add2-interfaces}
    This section describes the interfaces assigned to the components defined
    in the section above. Per interface, we list its methods by means of its
    syntax. The data types used in these interfaces are defined in the following section. \\

    Each method shows which (part of a) quality attribute or use case caused
    a need for the method. However, this does not mean that a method is
    only to be used to satisfy that quality  attribute or use case, it could
    be used for other causes not yet mentioned here.

    The interfaces and methods defined here are to be seen as an
    extension of the interfaces defined in previous sections, unless
    explicitly stated otherwise.

    \subsubsection{DeviceDataScheduler}
        \begin{itemize}
            \item DeviceMgmt
            \begin{itemize}
                \item \texttt{void addMote(int moteID, int gatewayID, IPAddress moteIPAddress)}
                    \begin{itemize}
                        \item Effect:
                        \item Created for: UC4.3
                    \end{itemize}
                \item \texttt{void reactivateMote(int moteID)}
                    \begin{itemize}
                        \item Effect:
                        \item Created for: UC4.3
                    \end{itemize}
            \end{itemize}

        	\item TopologyMgmt
        	\begin{itemize}
        		\item \texttt{void addMoteInTopology(int infrastructureOwnerID, int gatewayID, int moteID)}
        		\begin{itemize}
        			\item Effect:
        			\item Created for: UC4.3
        		\end{itemize}

                \item \texttt{void reactivateMoteInTopology(int moteID)}
                \begin{itemize}
                    \item Effect: Sets status to placed.
                    \item Created for: UC4.3
                \end{itemize}
        	\end{itemize}
        \end{itemize}

    \subsubsection{GatewayFacade}
        \begin{itemize}
            \item DeviceMgmt, last defined in section \ref{add1-interfaces}
            \begin{itemize}
                \item \texttt{void addMote(moteID, int gatewayID, IPAddress moteIPAddress)}
                \begin{itemize}
                    \item Effect:
                    \item Created for: UC11: UC4.3
                \end{itemize}
                \item \texttt{void reactivateMote(int moteID)}
                \begin{itemize}
                    \item Effect: Uses the DeviceDataScheduler to change the status of the mote to active.
                          Uses the TopologyManager to change the status of the mote to placed again.
                          Uses the NotificationHandler to send a notification of this event to the infrastructure owner.
                    \item Created for: UC4.3 - reintroduced mote
                \end{itemize}
            \end{itemize}
        \end{itemize}

    \subsubsection{PluggableDeviceDB}
        \begin{itemize}
        	\item DeviceMgmt
        	\begin{itemize}
        		\item \texttt{void addMote(int moteID, int gatewayID, IPAddress moteIPAddress)}
        		\begin{itemize}
        			\item Effect:
        			\item Created for: UC4.3
        		\end{itemize}
                \item \texttt{void reactivateMote(int moteID)}
                    \begin{itemize}
                        \item Effect:
                        \item Created for: UC4.3
                    \end{itemize}
        	\end{itemize}

        	\item TopologyMgmt
        	\begin{itemize}
        		\item \texttt{void addMoteInTopology(int infrastructureOwnerID, int gatewayID, int moteID)}
        		\begin{itemize}
        			\item Effect:
        			\item Created for: UC4.3
        		\end{itemize}
                \item \texttt{void reactivateMoteInTopology(int moteID)}
                    \begin{itemize}
                        \item Effect: Sets status to placed.
                        \item Created for: UC4.3
                    \end{itemize}
        	\end{itemize}

        \end{itemize}

    \subsubsection{TopologyManager}
        \begin{itemize}
        	\item TopologyMgmt
        	\begin{itemize}
        		\item \texttt{void addMote(infrastructureOwnerID, int gatewayID, int moteID)}
        		\begin{itemize}
        			\item Effect: // status = unplaced
        			\item Created for: UC4.3
        		\end{itemize}
                \item \texttt{void reactivateMote(int moteID)}
                    \begin{itemize}
                        \item Effect:
                        \item Created for: UC4.3
                    \end{itemize}
        	\end{itemize}
        \end{itemize}

\subsection{Data type definitions}
    This section defines new data types that are used in the interface descriptions above.

    \paragraph{DataType}
        Description of data type

    \newpage
    \section{Decomposition 4: Av2, UC12, UC25, UC26, UC27 (application execution subsystem)}


\subsection{Selected architectural drivers}
    The non-functional drivers for this decomposition are:
    \begin{itemize}
    	\item \emph{Av2}: Application failure
    \end{itemize}

    The related functional drivers are:
    \begin{itemize}
        \item \emph{UC12}: Perform actuation command \\
              Short description of the UC.
        \item \emph{UC13}: Configure pluggable device \\
              Short description of the UC.
        \item \emph{UC25}: Access topology and available devices \\
              Short description of the UC.
        \item \emph{UC24}: Consult historical data \\
              Short description of the UC.
        \item \emph{UC26}: Send application command or message to external front-end \\
              Short description of the UC.
        \item \emph{UC27}: Receive application command or message from external front-end \\
              Short description of the UC.
    \end{itemize}

    \paragraph{Rationale}
        At this point the remaining drivers were Av1, Av2, and P1,
        which all had medium priority. We chose decompositions 4, 5,
        and 6 based on the priorities of the use cases that are related to the quality attributes. \\
        The related use cases from now on are the ones that would use components
        that are going to be changed in the decomposition.


\subsection{Architectural design}
    This section describes what needs to be done to satisfy the requirements for
    this decomposition and how involved problems/obstacles are solved.

    Pattern: Container
    Define a container to provide the execution environment for a
    component that supports the necessary technical infrastructure
    to integrate components into application-specific usage scenarios,
    and on specific system platforms, without tightly coupling
    the components with the applications or platforms.

    Use the container to initialize and provide the runtime context for the
components it manages. Define operations that enable component
objects to access their connections to ports of other components,
as well as to access common middleware services such as persistence,
event notification, transactions, replication, load balancing,
and security.



    \paragraph{Av2: Detection of failures}
        The system is able to autonomously detect failures of its individual
        application execution components, failing applications, and failing application containers. \\
        Upon detection, a SIoTIP system administrator is notified. \\
        The failure of an internal application execution component is detected within 30 seconds.
        Detection of failed hardware or crashed software happens within 5 seconds.
        SIoTIP system administrators are notified within 1 minute.\\

        *) APPLICATION CRASH
        -> Monitor

        ApplicationManager: that does management of instances and communication with other components

        ApplicationInstanceContainer: is a container/sandbox that has
        1 running application instance
        = application execution component
        ?? run on different hardware maybe?

        ContainerMonitor: that monitors the ContainerInstance instances

        ContainerMonitor -> ApplicationInstance: boolean check()
        ApplicationInstance -> ContainerMonitor: void applicationCrashed(id applicationInstanceID)

        *) OR APPLICATION EXECUTION COMPONENT CRASH
        -> we say container has crashed/failed when the following has no response:
        ContainerMonitor -> ApplicationInstance: boolean check()
        is ok????

        *) Detection of failed hardware
        TODO ask what hardware?? separate hardware that runs application instances?

        *) Send notification

    \paragraph{Av2: Resolution of application failures}
        In case of application crash, the system autonomously restarts failed applications.
        If part of an application fails, the remaining parts remain operational,
        possibly in a degraded mode (graceful degradation). \\
        After 3 failed restarts the application is suspended, and the
        application developer and customer organisation are notified within 5 minutes.\\
        Application fails -> ContainerMonitor detects this -> ApplicationInstance: restart
        3 failed restarts -> suspend app and send notification

        *) Graceful degradation:
        -> send message to other parts to notify of failed part
        we ca have component responsible for controlling state of application
        and can start, stop or restart application.



    \paragraph{Av2: Resolution of application execution component failures}
        In case of failure of application execution components or an application
        container, a system administrator is notified. \\
        SIoTIP system administrators are notified within 1 minute.\\
        Solution for the problem.
        ApplicationInstance failure -> send notification to system administrator
        SIoTIP system administrators are notified within 1 minute.
        application return diferent metrics

        (usage of memory, number of served requestst and so ) and some monitoring
        system (component), that monitors container sends regularly request to this endpoint
        and  collect this information about aplications

    \paragraph{Av2: Failures do not impact other applications or other functionality of the system}
        This does not affect other applications that are executing on the Online
        service or SIoTIP gateway. This does not affect the availability of
        other functionality of the system, such as the dashboards. \\
        Applications fail independently: they are executed within their own
        container to avoid application crashes to affect other applications.\\
       -> we already have this, the ApplicationContainer component.

    \paragraph{Av2: Up-time of application execution subsystem}
        The subsystem for executing applications in the Online Service must
        have a guaranteed minimal up-time. The subsystem for executing
        applications in the Online Service must be available
        99\% of the time, measured per month. \\
        Solution for the problem.
        Subsystem = ApplicationManager + ApplicationContainer + new components ???

        containers can have replicas, for example every container has 3 replicas and
        when one crash  there has to be sometnig that find out that there are
        just 2 replicas now and it is needed to create new replica.

        There is also needed component for load balancing.


        DEVELOPERS WRITE THIS: command = "on"
        actuators = getActuatorsOfType("lightswitch")
        foreach (actuator) {
            actuator.command("on")
        }

        WE NEED TO CONVERT "on" TO A COMMAND THAT THE ACTUATOR UNDERSTANDS
        "on" => "turnOn"
        "on" => "lightOn"
        "on" => "switch"

        UC12:
            1. An application indicates that it wants one or more pluggable devices to perform an actuation command
                from client application:
                    ApplicationClient -> ApplicationFacade:           interface AppMgmt:   sendCommand(List<PluggableDeviceID> devices, string command)
                        Effect: Sends command from application to the pluggable device.
                        \item Created for: UC12.1
                    ApplicationFacade -> ActuationCommandConstructor: interface Actuation: sendCommand(List<PluggableDeviceID> devices, string command)
                        Effect: Sends command from application to the pluggable device.
                        \item Created for:UC12.1

                from application on online service:
                    ApplicationContainer -> ApplicationContainerManager: interface AppRequests

                from application on gateway:

            2. The system
                -constructs the actuation command message according to the specific formatting syntax for the involved pluggable device(s)
                    ActuationCommandConstructor/ApplicationManager -> ApplicationManager: interface X:
                      void sendCommand(PluggableDeviceID device, int moteID, Map<Pluggable,String> data)
                        Effect: Sends command from application to pluggable device.
                        \item Created for: UC12.2
                    Json message createData(PluggableDeviceID device, int moteID, Map<String,String> data)
                        Effect: Constructs actuation command message for involved pluggable device(s)
                        \item Created for: UC12.2
                   

                -sends the command message to the intended pluggable device(s).
                    ApplicationManager -> GatewayFacade: interface AppDeviceMgmt:
                                          void sendActuationCommand(List<PluggableDeviceID> actuators, List<string> commands)
                         Effect: Sends command to the pluggable device
                        \item Created for: UC12.2
                   


            3. The pluggable device(s) receive(s) the actuation command message and perform(s) the contained actuation command.
                    GatewayFacade -> MoteFacade:         interface DeviceMgmt: void sendActuationCommand(PluggableDeviceID device, string commandName)
                    Effect: Sends command from GatewayFacade to Mote for the pluggable device
                        \item Created for: UC12.2
                    MoteFacade -> PluggableDeviceFacade: interface Actuate:    void sendActuationCommand(string commandName)


            Find the gateway to send the commands  to:
                ApplicationManager -> OtherDataDB: interface AppMgmt: IPAddress getDeviceGateway(PluggableDeviceID device)
                    Effect: Returns the IP address of the gateway that a pluggable device is connected to
                    \item Created for: UC12 Remarks


        UC13:
            1. The primary actor specifies that it wants to set a configuration parameter of a pluggable
            device.
                ApplicationClient -> ApplicationFacade: interface AppMgmt:
                                                        void setConfiguration(PluggableDevideID id, Map<String,String> parameters)
                ApplicationFacade -> ApplicationManager: interface DeviceMgmt:
                                                        void setConfiguration(PluggableDevideID id, Map<String,String> parameters)


            2. The system verifies that the value of the configuration parameter is valid for the device (for
               example, a sensor which provides temperature information may have hardware limits on the
               sampling frequency).
               ApplicationManager -> DeviceDB: interface AppDeviceMgmt: bool checkDeviceParameter(PluggableDeviceID):

            3. The system determines whether the pluggable device needs to be reconfigured, and if so,
               constructs a reconfiguration command according to the specific formatting syntax for the
               pluggable device and sends it to the pluggable device. ????how

            4. The system updates the internal configuration of the pluggable device.
                ApplicationManager -> Datatase: interface AppMgmt:
                                                     void setConfiguration(PluggableDeviceID id, Map<String,String> parameters)

            5. The system informs the primary actor that the reconfiguration was done successfully.
                ApplicationManager -> NotificationManager: interface Notify: notify(int userID, string message)

            Alternative scenarios:
            2a. If the value is invalid for the pluggable device, the system informs the application via an
                exception. The use case ends.
                - first exception
            Remarks:
                Note that different applications may have different preferences for a single pluggable device

        UC24:
            1. The primary actor indicates that it wants to consult a specified collection of historical data in
                a specified timeframe.
                ApplicationClient -> ApplicationFacade: interface AppData:
                                                void getHistoricalData(DateTime from, DateTime to, int custOrgID)
                ApplicationFacade -> ApplicationManager: interface Apps:
                                                void getHistoricalData(DateTime from, DateTime to, int custOrgID)
            2. The system determines from which pluggable devices the data is required and looks up the data.
                 ????
            3. The system presents the primary actor with the requested historical overview, e.g. as a table.
                ApplicationManager -> DeviceDataScheduler: interface RequestData:
                                      HistoricalData getHistoricalData(List<PluggableDeviceID> id,DateTime from, DateTime to)
                DeviceDataScheduler -> PluggableDeviceDB: interface DeviceData:
                                      HistoricalData getHistoricalData(List<PluggableDeviceID> id,DateTime from, DateTime to)

        UC25:
            1. The primary actor indicates that it wants an overview of the topology.
                ApplicationClient -> ApplicationFacade: interface TopologyOverview
                                                               void getTopologyOverview(int custOrgID)
            2. The system looks up the pluggable devices that are available to the customer organisation
           that owns the primary actor, and composes a view on the topology including these pluggable devices.
                ApplicationFacade -> ApplicationManager: interface Apps: List<RoomTopology> getTopology(int custOrgID)
                TopologyManager -> DeviceDB: interface TopologyMgmt: List<RoomTopology> getTopology(int custOrgID)
            3. The system presents the topology view to the primary actor.

        UC26:
            1. The primary actor indicates it wants to send an application command or message to an external
                front-end and specifies the destination (e.g., as an application identifier for SIoTIP applications,
                or a hostname and port for external systems).
                Map<String,String> destination = type of destination, value
                Map<String,String> requestType = Message/Command, value
                GWApplicationContainer -> GWApplicationContainerManager: interface AppRequests:
                                                   void sendRequest(Map<String,String> destination, Map<String,String> requestType)
                       Effect: Command or message is sent to  to an external front-end
                       \item Created for: UC26.1
                GWApplicationContainerManager -> GWActuationCommandConstructor: interface AppRequests:
                                                      void sendRequest(Map<String,String> destination, Map<String,String> requestType)
                         Effect: Command or message is sent to  to an external front-end
                       \item Created for: UC26.1         
            2. The system checks that the primary actor is allowed to send to the specified destination.
            AccesRightManager???
            3. If the primary actor is allowed to send to the destination, and if the destination is another
                application, the system delivers the application command to that destination (Include: UC27:
                Receive application command or message from external front-end).
                TODO: add containers and monitor to gatewayfacade
                GWActuationCommandConstructor -> ActuationCommandConstructor: interaface Actuation
                                        void sendRequest(Map<String,String> destination,Map<String,String> requestType)
                See UC27
            4. The system informs the primary actor that the message was sent.
       
       UC27:     
        1. The system receives an application command or message for a SIoTIP application.
            ActuationCommandConstructor -> ApplicationContainerManager: interface AppRequests:
                                                 void rcvApplicationRequest(Map<String,String> destination,Map<String,String> requestType)
                   Effect: ApplicationContainerManager receives message or application command
                   \item Created for: UC27.1
            
        2. The system checks that the destination is available.
            ApplicationContainerManager -> ApplicationContainerMonitor: interface Availability: 
                                                bool checkDestination(Map<String,String>  destination)
                   Effect: Return true if destionation is available
                   \item Created for: UC27.2
                   
            ApplicationContainerMonitor -> ApplicationContainer: interface AppInstanceMgmt: 
            Map<Parameters,String> checkApplicationAvailability(AppInstanceID id)
                    Effect: Return parameters and their values of application
                   \item Created for: UC27.2
            
        3. If the destination is available, the system delivers the message to the destination application.
            ApplicationContainerManager -> ApplicationContainer: interface AppRequests
                               void rcvApplicationRequest(Map<String,String> destination,Map<String,String> requestType)
                   Effect: Receive message or application command
                   \item Created for: UC27.3

\subsection{Instantiation and allocation of functionality}
    This section lists the new components which instantiate our solutions
    described in the section above. For each component we note the quality
    attribute or use case that prompted us to create it. Descriptions about
    the components can be found under chapter \ref{ch:elements-datatypes}. \\

    \begin{itemize}
        \item Component: ApplicationClient
        \item Component: ApplicationFacade
        \item Component: ActuationCommandConstructor
        \item Component: ApplicationContainer
        \item Component: ApplicationContainerManager
        \item Component: ApplicationContainerMonitor
        \item Component: GWActuationCommandConstructor
        \item Component: GWApplicationContainer
        \item Component: GWApplicationContainerManager
        \item Component: GWApplicationContainerMonitor
    \end{itemize}


\subsection{Interfaces for child modules}
    This section lists new interfaces assigned to the components defined
    in the section above. Detailed information about each interface and
    its methods can be found under chapter \ref{ch:elements-datatypes}. \\

    \subsubsection{ActuationCommandConstructor}
        \begin{itemize}
            \item Actuation
        \end{itemize}
    \subsubsection{GWActuationCommandConstructor}
        \begin{itemize}
            \item AppRequests
        \end{itemize}
   \subsubsection{ApplicationContainer}
        \begin{itemize}
            \item AppRequests
        \end{itemize}
    \subsubsection{ApplicationContainerManager}
        \begin{itemize}
            \item AppRequests
        \end{itemize}
    \subsubsection{GWApplicationContainerManager}
        \begin{itemize}
            \item AppRequests
        \end{itemize}
    \subsubsection{ApplicationFacade}
        \begin{itemize}
            \item AppMgmt
        \end{itemize}
    \subsubsection{ApplicationFacade}
        \begin{itemize}
            \item DeviceMgmt
        \end{itemize}
    \subsubsection{ApplicationFacade}
        \begin{itemize}
            \item TopologyOverview
        \end{itemize}
    \subsubsection{PluggableDeviceFacade}
        \begin{itemize}
            \item Actuate
        \end{itemize}

\subsection{Data type definitions}
    This section lists the new data types introduced during this decomposition.

    \begin{itemize}
        \item Parameter: Parameter of application e.g. usage of memory, handle requests...
    \end{itemize}

    \newpage
    \section{Decomposition 5: Av1 (Gateway - Online Service communication subsystem)}


\subsection{Selected architectural drivers}
    The non-functional drivers for this decomposition are:
    \begin{itemize}
    	\item \emph{Av1}: Communication between SIoTIP gateway and Online Service
    \end{itemize}


\subsection{Architectural design}
    This section describes what needs to be done to satisfy the requirements for
    this decomposition and how involved problems/obstacles are solved.

    \paragraph{Av1: New Gateway responsibilities}
        The SIoTIP gateway is able to autonomously detect failures of its individual internal communication components.\\
        The Online Service should acknowledge each message sent by the SIoTIP gateway so that the gateway can detect failures.\\
        If an internal SIoTIP gateway component fails, the gateway first tries to restart the affected component.
        If the failure persists, the SIoTIP gateway reboots itself entirely. Note that the SIoTIP gateway,
        due to the occurred failure, cannot contact a system administrator itself.\\
        If (an internal communication component of) the Online Service or the communication
        channel has failed, the SIoTIP gateway will temporarily store all incoming pluggable data
        and any issued application commands internally.\\
        If the Online Service becomes unreachable, application parts running locally on the SIoTIP
        gateway continue to operate normally.\\
        The SIoTIP gateway will start synchronising with the Online service within 1 minute after the
        communication channel becomes available.\\
        The SIoTIP gateway can store at least 3 days of pluggable data before old data has to be overwritten. \\

        OnlineServiceBroker: Isolates communication-related concerns between Gateways and the Online Service along with GatewayBroker on the Online Service.
                             Forwards requests from one party to the other and transmits results and possible exceptions.

        OnlineServiceBrokerMonitor: Monitors the communication component on Gateways. If the communication component fails, the monitor
                                    tries to restart it. If the failure persists, the gateway reboots itself entirely.

        In OnlineServiceBroker:
            BrokerLogic: Handles all functionality related to communication.
            RequestStore: Temporarily stores all pluggable data and issued application commands until they can be deleted (= until an acknowledgement has been received for the request by the Online Service). Can store at least 3 days of pluggable data before old data has to be overwritten.
            OnlineServiceMonitor: Monitors the Gateway's connectivity to the Online Service. If the Online Service or the communication channel has failed, all requests to the Online Service will be stopped and stored in the RequestStore. An explicit command for this is not necessary,
                                  because the requests in the RequestStore will not be deleted, since no acknowledgements are received anymore from the Online Service. After the monitor detects that a connection to the Online Service is possible again, it makes the gateway start
                                  synchronising again. When the Online Service is unreachable, application parts running locally on the SIoTIP gateway continue to operate normally.

        INTERFACES:
            OnlineServiceBroker:
                HAS ALL INTERFACES PROVIDED BY COMPONENTS INSIDE OF IT

                interface CommunicationComponentMonitoring used by OnlineServiceBrokerMonitor:
                    boolean check()
                    void restart()

            BrokerLogic:
                interface OSCommunication used by Online Service:
                    void acknowledgement(int requestID)

                    void send(...)
                    void receive(...)

                interface OSMonitoring used by OnlineServiceMonitor:
                    Echo pingOnlineService()
                    void synchroniseWithOnlineService()

            RequestStore:
                All interfaces that are going to the online service now go to the broker
                All requests come in here and are stored
                The requests are then forwarded to BrokerLogic containing a unique requestID

                Interface Communication used by BrokerLogic:
                    void deleteRequest(int requestID)
                    List<Object> getNewRequestsForSynchronisation()

            OnlineServiceMonitor:
                Online service keeps track of connection to Online Service.
                If the Online Service becomes unreachable (= does not send ACKs anymore), then start pinging the Online Service

                Interface OSUpdates used by BrokerLogic:
                    void onlineServiceUpdate()


    \paragraph{Av1: New Online Service responsibilities}
        The Online Service is able to autonomously detect failures of its individual internal communication components.\\
        The Online Service is able to detect that a SIoTIP gateway is not sending data anymore based on the expected synchronisation interval.\\
        The Online Service notifies the infrastructure manager and a SIoTIP system administrator when the outage of a SIoTIP gateway is detected.\\
        The failure of an internal SIoTIP Online Service component is detected within 30 seconds.\\
        The detection time for a failed SIoTIP gateway or channel depends on the transmission rate
        of the gateway. An outage is dened as 3 consecutive expected synchronisations that do not
        arrive within 1 minute of their expected arrival time.\\
        The infrastructure owner is notied within 5 minutes after the detection of an outage of their gateway.\\
        A SIoTIP system administrator should be notied within 1 minute after the detection
        of a simultaneous outage of more than 1\% of the registered gateways.\\

        GatewayBroker: Isolates communication-related concerns between the Online Service Gateways and along with OnlineServiceBroker on Gateways.
                       Forwards requests from one party to the other and transmits results and possible exceptions. \\
                       Sends acknowledgements for all messages sent by Gateways so that they can detect failures.
        GatewayBrokerMonitor: Monitor the communication component for communication with gateways on the Online Service.

        In GatewayBroker:
            BrokerLogic: Handles all functionality related to communication.
            GatewayMonitor: Monitors the connectivity status all gateways. Can detect that a gateway is not sending data anymore based on the expected synchronisation interval. If 3 consecutive expected synchronisations do not arrive within 1 minute of their expected arrival time,
                            this is detected as a gateway outage. When outages of gateways are detected, the infrastructure owners that own the gateways and a SIoTIP system administrator are notified.
                            When the connectivity status change of a Gateway is detected, this is saved in the DeviceDB.

        INTERFACES:
            GatewayBroker:
                HAS ALL INTERFACES PROVIDED BY COMPONENTS INSIDE OF IT

                interface CommunicationComponentMonitoring used by GatewayBrokerMonitor:
                    boolean check()

            BrokerLogic:
                ALL INTERFACES FROM OS TO GATEWAY ARE NOW TO THIS COMPONENT

                interface GWCommunication used by Gateway/BrokerLogic:

            GatewayMonitor:
                % table(int id, int countSynchronisationsMissed, DateTime nextSyncPeriod)
                %       32 bit, 4 bit, 64 bit
                %     5000 gateways
                %     5000*(32+4+64)/8/1024 kilobyte ~~ 64 kb

                interface GatewayUpdates used by BrokerLogic:
                    void gatewayUpdate(int gatewayID, DateTime time)

            NotificationHandler:
                Interface Notify used by GatewayMonitor

            DeviceDB:
                Interface DeviceMgmt used by GatewayMonitor:
                    void setGatewayUnreachable()
                    void getPercentageOfUnreachableGateways()


    In BROKER, many clients can make remote method invocations on
    specific remote component objects hosted by a server. Clients thus
    communicate with the server objects in a many-to-one fashion, and
    their functional interfaces are often statically typed. The remote
    method invocation style of communication provided by B ROKER is
    best suited for systems that try to hide the presence of the network.

    MESSAGING relaxes this coupling and typing: clients send dynamically
    typed messages to specific remote services that reside at commu-
    nication endpoints, not (necessarily) to specific methods. M ESSAGING
    thus enables many-to-one communication without statically pre-
    defining the interface dependencies of clients to services.Distribution Infrastructure

    PUBLISHER-SUBSCRIBER decouples an application’s components even
    more: they can exchange events in a one-to-many manner with-
    out knowing one another’s identity explicitly, and without having
    to make a request each time new events are available. P UBLISHER -
    S UBSCRIBER middleware is therefore responsible for tracking which
    subscribers receive specific events sent asynchronously by pub-
    lishers. Subscribers react when receiving an event by performing
    some action, but publishers do not directly initiate the execution of
    a specific method on the subscribers.

    BROKER makes invocations
    on remote component objects look and act as much as possible
    like invocations on component objects in the same address space
    as their clients. MESSAGING and PUBLISHER-SUBSCRIBER are most appropriate
    for integration scenarios in which multiple, independently
    developed and self-contained services or applications must collaborate
    and form a coherent software system.

    A message channel does not come without cost, however, since it
    needs memory, networking resources, and persistent storage to support
    GUARANTEED DELIVERY [HoWo03]. Developers must therefore plan
    and configure the number and types of message channels explicitly
    and thoughtfully to ensure the desired quality of service in a given
    system deployment. A well-designed set of message channels forms a
    MESSAGE BUS [HoWo03] that acts like a messaging API for the clients
    and services in the distributed system.

    Employed tactics and patterns: broker, heartbeat, ping/echo

    DEPLOYMENT RATIONALE:
        OnlineServiceBroker
        OnlineServiceBrokerMonitor

        GatewayBroker
        GatewayBrokerMonitor

        For both the Online Service and Gateways, the components used for communication
        (\texttt{OnlineServiceBroker}, \texttt{GatewayBroker}) are to be deployed on different nodes than
        their monitoring components (\texttt{OnlineServiceBrokerMonitor}, \texttt{GatewayBrokerMonitor}).
        Otherwise, if the node of a communication
        component fails, its monitoring component would also fail and thus
        nothing would be detected.

    Alternative for monitoring of gateways:
        Gateway updated come to GatewayMonitor
        We could make GatewayMonitor ping all the gateways
        However, this would increase traffic on the network

    Alternative for communication:
        Messaging, Publishher Subscriber


\subsection{Instantiation and allocation of functionality}
    This section lists the new components which instantiate our solutions
    described in the section above. For each component we note the quality
    attribute or use case that prompted us to create it. Descriptions about
    the components can be found under chapter \ref{ch:elements-datatypes}. \\

    \begin{itemize}
        \item BrokerLogic: Av1
        \item OnlineServiceBroker: Av1
        \item OnlineServiceBrokerMonitor: Av1
        \item GatewayBroker: Av1
        \item GatewayBrokerMonitor: Av1
        \item RequestStore: Av1
        \item GatewayMonitor: Av1
        \item OnlineServiceMonitor: Av1
    \end{itemize}


\subsection{Interfaces for child modules}
    This section lists new interfaces assigned to the components defined
    in the section above. Detailed information about each interface and
    its methods can be found under chapter \ref{ch:elements-datatypes}.

    \subsubsection{BrokerLogic}
        \begin{itemize}
            \item Communication
            \item OSCommunication
            \item GWCommunication
            \item OSMonitoring
        \end{itemize}

    \subsubsection{OnlineServiceBroker}
        \begin{itemize}
            \item CommunicationComponentMonitoring
        \end{itemize}

    \subsubsection{GatewayBroker}
        \begin{itemize}
            \item CommunicationComponentMonitoring
        \end{itemize}

    \subsubsection{RequestStore}
        \begin{itemize}
            \item Communication
        \end{itemize}

    \subsubsection{GatewayMonitor}
        \begin{itemize}
            \item GatewayUpdates
        \end{itemize}

    \subsubsection{OnlineServiceMonitor}
        \begin{itemize}
            \item OSMonitoring
        \end{itemize}

\subsection{New data types}
    This section lists the new data types introduced during this decomposition.

    \begin{itemize}
        \item None
    \end{itemize}

    \newpage
    \section{Decomposition 6: P1, UC1, UC2, UC3, UC5, UC7, UC8, UC16, UC20 (Elements/Subsystem to decompose/expand)}


\subsection{Selected architectural drivers}
    The non-functional drivers for this decomposition are:
    \begin{itemize}
    	\item \emph{P1}: Large number of users
    \end{itemize}

    The related functional drivers are:
    \begin{itemize}
        \item \emph{UC1}: Register a customer organisation \\
            Short description of the UC.
        \item \emph{UC2}: Register an end-user \\
            Short description of the UC.
        \item \emph{UC3}: Unregister an end-user \\
            Short description of the UC.
        \item \emph{UC5}: Uninstall mote \\
            Short description of the UC.
        \item \emph{UC7}: Remove a pluggable device from its mote \\
            Short description of the UC.
        \item \emph{UC8}: Initialise a pluggable device \\
            Short description of the UC.
        \item \emph{UC16}: Consult notification message \\
            Short description of the UC.
        \item \emph{UC20}: Unsubscribe from application \\
            Short description of the UC.
    \end{itemize}


\subsection{Architectural design}
    This section describes what needs to be done to satisfy the requirements for
    this decomposition and how involved problems/obstacles are solved.

    \paragraph{P1: Problem title}
        The SIoTIP Online Service replies to the service requests of the
        infrastructure owner and customer organisations.

    \paragraph{P1: Problem title}
        The Online Service processes the data received from the gateways.

    \paragraph{P1: Problem title}
        The application execution subsystem should be able to execute an increasing
        number of active applications.

    \paragraph{P1: Problem title}
        The initial deployment of SIoTIP should be able to deal with at least 5000
        gateways in total, and should be provisioned to service at least 3000
        registered users simultaneously connected to SIoTIP.
        -> 5000 gateways * 4 motes per gateway * 3 devices per mote = 60000 devices
        -> Keep communication with gateways at a minimum
           e.g. gateway messages are of type "send data", "new device connected", ...
        -> LOAD BALANCING

    \paragraph{P1: Problem title}
        Scaling up to service an increasing amount of infrastructure owners,
        customers organisations and applications should (in worst case) be linear ;
        i.e. it should not require proportionally more resources (machines, etc.)
        than the initial amount of resources provisioned per customer
        organisation/infrastructure owner and per gateway.


\subsection{Instantiation and allocation of functionality}
    This section lists the new components which instantiate our solutions
    described in the section above. For each component we note the quality
    attribute or use case that prompted us to create it. Descriptions about
    the components can be found under chapter \ref{ch:elements-datatypes}. \\

    \begin{itemize}
        \item Component: (P1 or UC)
    \end{itemize}


\subsection{Interfaces for child modules}
    This section lists new interfaces assigned to the components defined
    in the section above. Detailed information about each interface and
    its methods can be found under chapter \ref{ch:elements-datatypes}. \\

    \subsubsection{Component}
        \begin{itemize}
            \item Interface
        \end{itemize}

\subsection{Data type definitions}
    This section lists the new data types introduced during this decomposition.

    \begin{itemize}
        \item DateTime: Represents an instant in time, typically expressed as a date and time of day.
    \end{itemize}

    \newpage
    %\section{Decomposition 7: UC28, UC29 (Elements/Subsystem to decompose/expand)}
    At this point, all quality attributes have been handled. The remaining
    decompositions handle all of the use cases that are left. The order
    is based on the priority of the use cases.


\subsection{Selected architectural drivers}
    The functional drivers are:
    \begin{itemize}
        \item \emph{UC28}: Log in \\
              Short description of the UC.
        \item \emph{UC29}: Log out \\
              Short description of the UC.
    \end{itemize}


\subsection{Architectural design}
    This section describes what needs to be done to satisfy the requirements for
    this decomposition and how involved problems/obstacles are solved.

    \paragraph{UC: Problem title}
        Short description of the problem.\\
        Solution for the problem.


\subsection{Instantiation and allocation of functionality}
    This section describes the new components (and their responsibilities)
    which instantiate our solutions described in the section above. \\
    Unless stated otherwise, responsibilities assigned in previous decompositions are unchanged.

    \subparagraph{Component}
        Short description of its responsibilities. (Relevant UC)


\subsection{Interfaces for child modules}
    This section lists new interfaces assigned to the components defined
    in the section above. Detailed information about each interface and
    its methods can be found under chapter \ref{ch:elements-datatype}. \\

    \subsubsection{Component}
        \begin{itemize}
            \item Interface
        \end{itemize}

\subsection{Data type definitions}
    This section lists the new data types introduced during this decomposition.

    \begin{itemize}
        \item DateTime: Represents an instant in time, typically expressed as a date and time of day.
    \end{itemize}

    \newpage
    %\section{Decomposition 8: UC22, UC23 (Elements/Subsystem to decompose/expand)}


\subsection{Selected architectural drivers}
    The functional drivers are:
    \begin{itemize}
        \item \emph{UC22}: Upload an application
        \item \emph{UC23}: Consult application statistics
    \end{itemize}

    UC22:
        1. The primary actor indicates that he/she wants to upload an application.
            IN CLIENT

        2. The system asks the primary actor if he/she wants to update an existing application.
            IN CLIENT

        3. The primary actor provides his/her choice.

            TODO

        4. If the primary actor indicated that he/she wants to update an existing application,
           the system composes an overview of applications uploaded by the primary actor
           (e.g., as a list or table), presents this, and requests the primary actor to indicate which application to update.

            ApplicationProviderFacade -> ApplicationManager: interface FrontEndAppRequests:
                                                List<Application> getApplicationsForAppProvider(int appProviderID)
                Effect: Returns list of applications uploaded by the application provider.
                \item Created for: UC22.4

            ApplicationManager -> OtherDataDB: interface DBAppMgmt:
                                                List<Application> getApplicationsForAppProvider(int appProviderID)
                Effect: Returns list of applications uploaded by the application provider.
                \item Created for: UC22.4


        5. The primary actor chooses the application to be updated.

            ApplicationProviderClient -> ApplicationProviderFacade: interface Apps:
                                                void updateApplication(int applicationID)


        6. The system asks the primary actor if this is an update that should be automatically activated for existing subscriptions.

            TODO

        7. The primary actor provides his/her choice.

            TODO

        8. If the primary actor has indicated that existing subscriptions should be automatically updated, the system requests the versions (or range of versions) that should be automatically updated.

            TODO

        9. The primary actor provides the requested information.

            TODO

        10. The system asks the primary actor to upload the application code, a description for display to customer organisations, and meta-data (such as the version number and subscription price).

            TODO

        11. The primary actor uploads the requested information.

            TODO

        12. The system performs automated application checks.

            TODO

        13. If the checks were successful, the system sends a notification to the primary actor (Include: UC15 : Send notification).

            TODO

        14. The system makes the application available in the application store. In case of an updated application, it replaces the previous versions of the application.

            TODO

        15. In case of an updated application that should automatically be activated for existing subscriptions (cf. step 6), the system updates the existing subscriptions of the application to the new version (Include: UC17 : Activate an application).

            TODO

        Alternative
            4a. If the primary actor has indicated that he/she wants to upload a new application, the use case
                continues from step 10.
            8a. If the primary actor has indicated that existing subscriptions should not be automatically
                updated, the use case continues from step 10.
            13a.If the checks were not successful, the system notifies the primary actor and a SysAdmin
                (Include: UC15: Send notification). The use case ends.

        Remarks:
            { For updated applications, application providers have the choice between automatically updating
              all application subscriptions of the previous version, or requiring customer organisations
              to subscribe to the new version individually (cf. UC19: Subscribe to application). The latter
              could be used if the application provider requires the customer organisations to pay for the
              new version. They may, however, also use the former model for some updates, such as minor
              improvements and bugfixes. Requiring the customer organisations to explicitly subscribe to
              the new version can also be useful in case the new version requires new hardware.
            { Explicitly indicating the versions from which to automatically update (in step 8) is necessary
              to prevent the automatic (and free) update from v2.0 to v2.1 to also be installed for v1.x
              subscriptions, thereby bypassing the paid upgrade from v1.x to v2.0.
            { For simplicity, the workflow of a sysadmin that manually reviews and approves/rejects an
              application that has failed automatic testing (as mentioned in Part 1 of the assignment) is not
              part of this assignment.


    UC23:
        1. The primary actor indicates that he/she wants to an overview of uploaded applications.

            ApplicationProviderClient -> ApplicationProviderFacade: interface Apps:
                                                    void getApplicationDetailsForAppProvider(int appProviderID)
                Effect: Returns list of applications uploaded by the application provider and information about the amount of subscribers
                        for that application.
                \item Created for: UC23.2

        2. The system retrieves all applications uploaded by the primary actor, and information about the amount of subscribers for that
           application. It presents this overview to the primary actor.

            ApplicationProviderFacade -> ApplicationManager: interface FrontEndAppRequests:
                                                List<Application,int> getApplicationDetailsForAppProvider(int appProviderID)
                Effect: Returns list of applications uploaded by the application provider and information about the amount of subscribers
                        for that application.
                \item Created for: UC23.2

            ApplicationManager -> OtherDataDB: interface DBAppMgmt:
                                                List<Map<Application,int> getApplicationDetailsForAppProvider(int appProviderID)
                Effect: Returns list of applications uploaded by the application provider and information about the amount of subscribers
                        for that application.
                \item Created for: UC23.2

        3. The primary actor selects an application.

            ApplicationProviderClient -> ApplicationProviderFacade: interface Apps:
                                                    ApplicationStatistics presentApplicationStatistics(int applicationID)
                Effect: Returns statistics about application if the selected application is approved.
                \item Created for: UC23.3

        4. If the selected application is an approved application, the system presents detailed statistics including the amount of
        subscribers.

        ApplicationProviderFacade -> ApplicationManager: interface FrontEndAppRequests:
                                                    ApplicationStatistics presentApplicationStatistics(int applicationID)
                Effect: Returns statistics about application if the selected application is approved.
                \item Created for: UC23.4

        ApplicationManager -> OtherDataDB: interface DBAppMgmt:
                                                    ApplicationStatistics presentApplicationStatistics(int applicationID)
                Effect: Returns statistics about application if the selected application is approved.
                \item Created for: UC23.4




\subsection{Instantiation and allocation of functionality}
    This section lists the new components which instantiate our solutions
    described in the section above. For each component we note the quality
    attribute or use case that prompted us to create it. Descriptions about
    the components can be found under chapter \ref{ch:elements-datatypes}. \\

    \begin{itemize}
        \item Component: (Relevant UC)
    \end{itemize}


\subsection{Interfaces for child modules}
    This section lists new interfaces assigned to the components defined
    in the section above. Detailed information about each interface and
    its methods can be found under chapter \ref{ch:elements-datatypes}.

    \subsubsection{Component}
        \begin{itemize}
            \item Interface
        \end{itemize}

\subsection{New data types}
    This section lists the new data types introduced during this decomposition.

    \begin{itemize}
        \item DateTime: Represents an instant in time, typically expressed as a date and time of day.
    \end{itemize}

    \newpage
    %\section{Decomposition 9: UC21 (Elements/Subsystem to decompose/expand)}


\subsection{Selected architectural drivers}
    The functional drivers are:
    \begin{itemize}
        \item \emph{UC21}: Send invoice
    \end{itemize}

    UC21: Send invoice
        1. The system constructs an invoice containing:
            { the provided service or product (e.g. monthly subscription fee for applications or purchased hardware),
            { the amount to be paid,
            { the date the payment is due and
            { the contact details of the primary actor (e.g. Zoomit identifier or credit card data)

                SubscriptionManager -> InvoiceManager: constructInvoice(Map<string,string>)

        2. The system sends the invoice to the third-party invoicing service.

            InvoiceManager -> InvoicingService: interface Invoices: boolean sendInvoice(Map<string,string>)
                Effect: Returns true if the ivoice was sent successfully.


\subsection{Instantiation and allocation of functionality}
    This section lists the new components which instantiate our solutions
    described in the section above. For each component we note the quality
    attribute or use case that prompted us to create it. Descriptions about
    the components can be found under chapter \ref{ch:elements-datatypes}. \\

    \begin{itemize}
        \item Component: (Relevant UC)
    \end{itemize}


\subsection{Interfaces for child modules}
    This section lists new interfaces assigned to the components defined
    in the section above. Detailed information about each interface and
    its methods can be found under chapter \ref{ch:elements-datatypes}.

    \subsubsection{ThirdPartyInvoicingService}
        \begin{itemize}
            \item InvoiceDeliveryMgmt
        \end{itemize}

    \subsubsection{InvoiceManager}
        \begin{itemize}
            \item DeliveryMgmt
        \end{itemize}

\subsection{New data types}
    This section lists the new data types introduced during this decomposition.

    \begin{itemize}
        \item DateTime: Represents an instant in time, typically expressed as a date and time of day.
    \end{itemize}


\end{document}
