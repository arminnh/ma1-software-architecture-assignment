\section{Av3: Pluggable device or mote failure}

    \subsection*{Key Decisions}

    \begin{itemize}
    	\item \texttt{DeviceManager} monitors connected/operational devices for a gateway.
    	\item \texttt{DeviceManager} stores which application instances use which devices and stores the application instances' requirements.
    \end{itemize}
    \emph{Employed tactics and patterns:} heartbeat

    \subsection*{Rationale}
        Gateways need to be able to autonomously detect failure of one of its
        connected motes and pluggable devices. This is achieved by making motes
        send heartbeats to their connected gateways. The gateways can
        then monitor their connected devices. The heartbeats contain a list
        of devices that are connected/operational at the moment the mote sends
        the heartbeat. Each gateway makes use of a \texttt{DeviceManager}
        component to monitor devices. This component uses timers to keep track
        of how long it has been since a device has sent a heartbeat or occured in
        a list of connected devices. Once a timer expires, this is treated as
        a failure. \\
        A mote has failed when 3 consecutive heartbeats do not arrive within 1
        second of their expected arrival time. \\
        A pluggable device has failed when it does not occur in a heartbeat of the
        mote in which it is expected to be in. This is is detected within 2
        seconds after the arrival of the heartbeat. \\

        Av3 states that applications should be automatically suspended when they can no longer
        operate due to failure of a pluggable device or mote and reactivated
        once the failure is resolved. \\
        This problem is also tackled by the \texttt{DeviceManager}. It
        stores the requirements (devices used and relationships between devices)
        for pluggable devices set by application instances for all
        instances that use the gateway that the \texttt{DeviceManager}
        runs on. When it detects that an application can no longer operate
        due to failures, it sends a command to the \texttt{ApplicationContainerManager}
        on the Gateway itself and on the Online Service
        to suspend that application. When the required devices are operational
        again, the \texttt{DeviceManager} detects this and sends a
        command to check if application instances can be reactivated. 

    \subsection*{Considered Alternatives}
        \paragraph{Alternative for failure detection}
            An alternative would have been to move the \texttt{DeviceManager}
            component from gateways to the Online Service. This solution would make the
            gateways do less work, but would be very unscalable. The reason is
            that as the customer base (and thus the amount of devices) increases,
            the Online Service would need to keep track of huge amounts of devices.
            This would also flood the network to the Online Service with heartbeats.

        \paragraph{Alternative for Failure detection}
            Another alternative for failure detection could have been the use of
            a Ping/Echo mechanism instead of Heartbeats. Pings could then be used
            to check if a device is currently operational. However, as a device could
            not be operational for a moment because of e.g. interference, timers
            would still be necessary to keep track of operational devices. We opted
            to use heartbeats, as this would reduce the amount of data sent over
            the network used by the motes, and as motes would have to do slightly
            more work to process each Ping request in order to generate a reply.
