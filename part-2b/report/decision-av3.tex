\section{Av3: Pluggable device or mote failure}

    \todoinline{Use this section structure for each requirement}

    \subsection*{Key Decisions}

    \todoinline{
    	Briefly list your key architectural decisions.
    	Pay attention to the solutions that you employed (in your own terms or using tactics and/or patterns).
    }

    \begin{itemize}
    	\item \texttt{DeviceManager} monitors connected/operational devices on a gateway.
    	\item \texttt{DeviceManager} stores the requirements for pluggable devices set by applications
    \end{itemize}
    \emph{Employed tactics and patterns:} heartbeat, ping/echo

    \subsection*{Rationale}
        \paragraph{Failure detection}
            Gateway need to be able to autonomously detect failure of one of its
            connected motes and pluggable devices. This is achieved by making motes
            send heartbeats to their connected gateways. The gateways can
            then monitor their connected devices. The heartbeats contain a list
            of devices that are connected/operational at the moment the mote sends
            the heartbeat. Each gateway makes use of a \texttt{DeviceManager}
            component to monitor the devices. This component uses timers to keep track
            of how long it has been since a device has sent a heartbeat or occured in
            a list of connected devices. Once a timer expires, this is treated as
            a failure. \\

            A mote has failed when 3 consecutive heartbeats do not arrive within 1
            second of their expected arrival time. \\
            A pluggable device has failed when it does not occur in a heartbeat of the
            mote in which it is expected to be in. This is is detected within 2
            seconds after the arrival of the heartbeat.

        \paragraph{Automatic application deactivation and redundancy settings}
            Applications should be automatically suspended when they can no longer
            operate due to failure of a pluggable device or mote and reactivated
            once the failure is resolved. Application providers can design their
            applications such that they explicitly require redundancy in
            the available pluggable devices. \\
            This problem is tackled by the \texttt{DeviceManager}. It
            stores the requirements for pluggable devices set by applications for all
            applications that use the gateway that the \texttt{DeviceManager}
            runs on. When it detects that an application can no longer operate
            due to failures, it will send a command to the \texttt{ApplicationManager}
            (via the \texttt{GatewayFacade})
            to suspend that application. When the required devices are operational
            again, the \texttt{DeviceManager} detects this and sends a
            command to reactivate the application. \\

            Applications are suspended within 1 minute after detecting
            the failure of an essential pluggable device. \\
            Application are reactivated within 1 minute after the failure is resolved.

        \paragraph{Notifications}
            The infrastructure owner should be notified of any persistent
            pluggable device or mote failures. Customer organisations should be
            notified if one or more of their applications is suspended or
            reactivated. Applications using a failed pluggable device or any device
            on a failed mote should be notified. \\
            The \texttt{NotificationHandler} was put in place to deal with
            notifications. Other components can use it to generate notifications for
            certain users in the system. The \texttt{NotificationHandler} will then
            insert information relevant to the notification in the database (message,
            status, date and time, source, ...), and use an external delivery
            service to deliver the notification to users. The used delivery medium
            is based on the user's preferences. The \texttt{DeviceManager} sends request
            to the external delivery service to notifies the
            infrastructure owner, once mote or pluggable device failure occurs.
            Since they are stored in the database, users can always view
            their notifications via their dashboard. However, this funcionality is not
            expanded on in this decomposition yet. \\

            Infrastructure owners are notified within 1 minute after detecting a mote outage lasting at
            least 10 seconds. \\
            Infrastructure owners are notified within 1 minute after the detection of the unavailability of
            a pluggable device for 30 seconds. \\
            Applications are notified of the failure of relevant pluggable devices within 10 seconds.


    \subsection*{Considered Alternatives}
        \paragraph{Alternative for failure detection}
            An alternative would have been to move the \texttt{DeviceManager}
            component from gateways to the Online Service. This solution would make the
            gateways do less work, but would be very unscalable. The reason is
            that as the customer base (and thus the amount of devices) increases,
            the Online Service would need to keep track of huge amounts of devices.
            This would also flood the network to the Online Service with heartbeats.

        \paragraph{Alternative for Failure detection}
            Another alternative for failure detection could have been the use of
            a Ping/Echo mechanism instead of Heartbeats. Pings could then be used
            to check if a device is currently operational. However, as a device could
            not be operational for a moment because of e.g. interference, timers
            would still be necessary to keep track of operational devices. We opted
            to use heartbeats, as this would reduce the amount of data sent over
            the network used by the motes, and as motes would have to do slightly
            more work to process each Ping request in order to generate a reply.

        \paragraph{Av3: Notifications}
            Reliable and quick delivery of notifications is crucial to the
            system in order to solve problems should things go wrong. Currently,
            the solution is to use a third party service for delivery of
            notifications. In the case that no external services are found
            satisfactory, or if this dependency on an external service is
            unwanted, it is possible to build an internal solution for this.
            For example, a \texttt{NotificationSender} component could make use
            of the \texttt{Factory pattern} for different message channels for
            different delivery methods (each with their own sendNotification method).
            This solution allows us to easily add new message channels in the
            future with little effort. The disadvantage of this is that an
            internal solution takes a lot more time to implement.


    \subsection*{Deployment Decisions}
        For Av3 is important that the \texttt{DeviceManager} is deployed on another node as the
        \texttt{Mote} and the \texttt{PluggableDevice}. The \texttt{DeviceManager}
        checks the \texttt{Mote} and the \texttt{PluggableDevice} availability and in case of failure, the \texttt{DeviceManager}
        can notify the infrastructure owner of any persistent pluggable device or mote failures.\\
