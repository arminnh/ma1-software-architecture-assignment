\section{Decomposition 3: U2, UC4, UC6, UC9, UC10, UC17, UC19.7-11}

\subsection{Elements/Subsystem to decompose/expand}
    In this run we decompose/expand ...


\subsection{Selected architectural drivers}
    The non-functional drivers for this decomposition are:
    \begin{itemize}
    	\item \emph{U2}: Easy installation
    \end{itemize}

    The related functional drivers are:
    \begin{itemize}
        \item \emph{UC4}: Install mote \\
              Short description
        \item \emph{UC6}: Inset a pluggable device into a mote \\
              Short description
        \item \emph{UC9}: Configure pluggable device access rights \\
              Short description
        \item \emph{UC10}: Consult and configure topology \\
              Short description
        \item \emph{UC17}: Activate an application \\
              Short description
        \item \emph{UC19}: Subscribe to application \\
              Short description
    \end{itemize}


\subsection{Architectural design}
    This section describes what needs to be done to satisfy the requirements for
    this decomposition and how involved problems/obstacles are solved.

    \paragraph{U2: Gateway installation}
        The gateway should not require any configuration, other than being connected
        to the local wired or WiFi network, after it is plugged into an electrical
        socket. An infrastructure owner should be able get the SIoTIP gateway
        up-and-running (connected) within 10 minutes given that the information
        (e.g. WiFi SSID and passphrase) is available to the person responsible for
        the installation. \\
        We need something that registers the gateway automatically with the
        online service after bootup. A connection to the internet is a constraint
        of the GatewayFacade.

    \paragraph{U2: Mote installation}
        Installing a new mote should not require more configuration than adding it
        to the topology. Adding new motes, sensors or actuators should not involve
        more than just starting motes, and plugging devices into motes – plug-and-play! \\
        What the actual fuck? \\
        Reintroducing a previously known mote, with the same pluggable devices attached to it,
        should not require any configuration. It is automatically re-added on
        its last known location on the topology. The attached pluggable devices
        are automatically initialised and configured with their last known
        configuration and access rights. \\
        Thing that need to happen automatically:
        *) mote should find the gateway (mote sends a broadcast message->ReceiveBroadcast)
        *) gateway should register the mote (PluggableDeviceManager update, store entry in DB)
        *) on reintroduction of motes: PluggableDeviceManager notices this, makes the gateway send a message to online service to reuse some old topology

    \paragraph{U2: Pluggable device installation}
        Adding new sensors or actuators should require no further customer
        actions besides plugging it into the mote. Configurable sensors and
        actuators should have a working default configuration.
        Pluggable devices added to an already known mote are automatically
        added in the right location on the topology.
        Making (initialised) sensors and actuators available to customer
        organisations and applications should not require more effort than
        configuring access rights (cf. UC9). \\
        Solution for the problem.

    \paragraph{U2: Easy applications}
        Applications should work out of the box if the required sensors and
        actuators are available. Only when mandatory end-user roles must be
        assigned, additional explicit configuration actions are required
        from a customer organisation (cf. UC17, UC19). \\
        Solution for the problem.


\subsection{Instantiation and allocation of functionality}
    This section describes the new components which instantiate our solutions described
    in the section above and how components are deployed on physical nodes. \\
    Unless stated otherwise the responsibilities assigned in the first decomposition are unchanged.

    \paragraph{Decomposition}
        Figure \ref{fig:FIGURELABEL} shows the components resulting from the
        decomposition in this run.

        \begin{figure}[!h]
        	\centering
            %\includegraphics[width=1\textwidth]{IMAGE FILE NAME}
        	\missingfigure[figwidth=0.8\textwidth]{Component-and-connector diagram of this decomposition}
        	\caption{Component-and-connector diagram of this decomposition.}
            \label{fig:FIGURELABEL}
        \end{figure}

        The responsibilities of the components are as follows:

    \subparagraph{Component}
        Short description of its responsibilities. (Relevant QA or UC)


    % \paragraph{Behaviour}
        % USEFUL SEQUENCE DIAGRAMS FOR CHOSEN USE CASES


    \paragraph{Deployment}
        Figure \ref{fig:FIGURELABEL} shows the allocation of components
        to physical nodes.

        \begin{figure}[!h]
        	\centering
        	%\includegraphics[width=0.8\textwidth]{IMAGE FILE NAME}
        	\missingfigure[figwidth=0.8\textwidth]{Deployment diagram}
        	\caption{Deployment diagram of this decomposition.}
            \label{fig:FIGURELABEL}
        \end{figure}


\subsection{Interfaces for child modules}\label{add2-interfaces}
    This section describes the interfaces assigned to the components defined
    in the section above. Per interface, we list its methods by means of its
    syntax. The data types used in these interfaces are defined in the following section. \\

    Each method shows which (part of a) quality attribute or use case caused
    a need for the method. However, this does not mean that a method is
    only to be used to satisfy that quality  attribute or use case, it could
    be used for other causes not yet mentioned here.

    The interfaces and methods defined here are to be seen as an
    extension of the interfaces defined in previous sections, unless
    explicitly stated otherwise.

    \subsubsection{Component}
        \begin{itemize}
            \item InterfaceName
            \begin{itemize}
                \item \texttt{void methodName(.. parameters ..)}
                \begin{itemize}
                    \item Effect: Short description of the method
                    \item Created for: Reason for the addition of this method.
                \end{itemize}
            \end{itemize}
        \end{itemize}

\subsection{Data type definitions}
    This section defines new data types that are used in the interface descriptions above.

    \paragraph{DataType}
        Description of data type
