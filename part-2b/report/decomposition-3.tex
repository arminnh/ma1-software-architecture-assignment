\section{Decomposition X: DRIVERS (Elements/Subsystem to decompose/expand)}

\subsection{Elements/Subsystem to decompose/expand}
    In this run we decompose/expand ...


\subsection{Selected architectural drivers}
    The non-functional drivers for this decomposition are:
    \begin{itemize}
    	\item \emph{QA}: Name of QA
    \end{itemize}

    \noindent The related functional drivers are:
    \begin{itemize}
        \item \emph{UCX}: Name of UC \\
              Short description of the UC.
    \end{itemize}


\subsection{Architectural design}
    This section describes what needs to be done to satisfy the requirements for
    this decomposition and how involved problems/obstacles are solved.

    \paragraph{QA: Problem title}
        Short description of the problem.\\
        Solution for the problem.


\subsection{Instantiation and allocation of functionality}
    This section describes the new components which instantiate our solutions described
    in the section above and how components are deployed on physical nodes. \\
    Unless stated otherwise the responsibilities assigned in the first decomposition are unchanged.

    \paragraph{Decomposition}
        Figure \ref{fig:FIGURELABEL} shows the components resulting from the
        decomposition in this run.

        \begin{figure}[!h]
        	\centering
            %\includegraphics[width=1\textwidth]{IMAGE FILE NAME}
        	\missingfigure[figwidth=0.8\textwidth]{Component-and-connector diagram of this decomposition}
        	\caption{Component-and-connector diagram of this decomposition.}
            \label{fig:FIGURELABEL}
        \end{figure}

        \noindent The responsibilities of the components are as follows:

    \subparagraph{Component}
        Short description of its responsibilities. (Relevant QA or UC)


    % \paragraph{Behaviour}
        % USEFUL SEQUENCE DIAGRAMS FOR CHOSEN USE CASES


    \paragraph{Deployment}
        Figure \ref{fig:FIGURELABEL} shows the allocation of components
        to physical nodes.

        \begin{figure}[!h]
        	\centering
        	%\includegraphics[width=0.8\textwidth]{IMAGE FILE NAME}
        	\missingfigure[figwidth=0.8\textwidth]{Deployment diagram}
        	\caption{Deployment diagram of this decomposition.}
            \label{fig:FIGURELABEL}
        \end{figure}


\subsection{Interfaces for child modules}\label{add2-interfaces}
    This section describes the interfaces assigned to the components defined
    in the section above. Per interface, we list its methods by means of its
    syntax. The data types used in these interfaces are defined in the following section. \\

    \noindent Each method shows which (part of a) quality attribute or use case caused
    a need for the method. However, this does not mean that a method is
    only to be used to satisfy that quality  attribute or use case, it could
    be used for other causes not yet mentioned here.

    \noindent The interfaces and methods defined here are to be seen as an
    extension of the interfaces defined in previous sections, unless
    explicitly stated otherwise.

    \subsubsection{Component}
        \begin{itemize}
            \item InterfaceName
            \begin{itemize}
                \item \texttt{void methodName(.. parameters ..)}
                \begin{itemize}
                    \item Effect: Short description of the method
                    \item Created for: Reason for the addition of this method.
                \end{itemize}
            \end{itemize}
        \end{itemize}

\subsection{Data type definitions}
    This section defines new data types that are used in the interface descriptions above.

    \paragraph{DataType}
        Description of data type
