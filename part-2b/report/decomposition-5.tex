\section{Decomposition 5: Av1 (Gateway - Online Service communication subsystem)}


\subsection{Selected architectural drivers}
    The non-functional drivers for this decomposition are:
    \begin{itemize}
    	\item \emph{Av1}: Communication between SIoTIP gateway and Online Service
    \end{itemize}


\subsection{Architectural design}
    This section describes what needs to be done to satisfy the requirements for
    this decomposition and how involved problems/obstacles are solved.

    \paragraph{Av1: New Gateway responsibilities}
        The SIoTIP gateway is able to autonomously detect failures of its individual internal communication components.\\
        The Online Service should acknowledge each message sent by the SIoTIP gateway so that the gateway can detect failures.\\
        If an internal SIoTIP gateway component fails, the gateway first tries to restart the affected component.
        If the failure persists, the SIoTIP gateway reboots itself entirely. Note that the SIoTIP gateway,
        due to the occurred failure, cannot contact a system administrator itself.\\
        If (an internal communication component of) the Online Service or the communication
        channel has failed, the SIoTIP gateway will temporarily store all incoming pluggable data
        and any issued application commands internally.\\
        If the Online Service becomes unreachable, application parts running locally on the SIoTIP
        gateway continue to operate normally.\\
        The SIoTIP gateway will start synchronising with the Online service within 1 minute after the
        communication channel becomes available.\\
        The SIoTIP gateway can store at least 3 days of pluggable data before old data has to be overwritten. \\

        OnlineServiceBroker: Isolates communication-related concerns between Gateways and the Online Service along with GatewayBroker on the Online Service.
                             Forwards requests from one party to the other and transmits results and possible exceptions.

        OnlineServiceBrokerMonitor: Monitors the communication component on Gateways. If the communication component fails, the monitor
                                    tries to restart it. If the failure persists, the gateway reboots itself entirely.

        In OnlineServiceBroker:
            BrokerLogic: Handles all functionality related to communication.
            RequestStore: Temporarily stores all pluggable data and issued application commands until they can be deleted (= until an acknowledgement has been received for the request by the Online Service). Can store at least 3 days of pluggable data before old data has to be overwritten.
            OnlineServiceMonitor: Monitors the Gateway's connectivity to the Online Service. If the Online Service or the communication channel has failed, all requests to the Online Service will be stopped and stored in the RequestStore. An explicit command for this is not necessary,
                                  because the requests in the RequestStore will not be deleted, since no acknowledgements are received anymore from the Online Service. After the monitor detects that a connection to the Online Service is possible again, it makes the gateway start
                                  synchronising again. When the Online Service is unreachable, application parts running locally on the SIoTIP gateway continue to operate normally.

        INTERFACES:
            OnlineServiceBroker:
                HAS ALL INTERFACES PROVIDED BY COMPONENTS INSIDE OF IT

                interface CommunicationComponentMonitoring used by OnlineServiceBrokerMonitor:
                    boolean check()
                    void restart()

            BrokerLogic:
                interface OSCommunication used by Online Service:
                    void acknowledgement(int requestID)

                    void send(...)
                    void receive(...)

                interface OSMonitoring used by OnlineServiceMonitor:
                    Echo pingOnlineService()
                    void synchroniseWithOnlineService()

            RequestStore:
                All interfaces that are going to the online service now go to the broker
                All requests come in here and are stored
                The requests are then forwarded to BrokerLogic containing a unique requestID

                Interface Communication used by BrokerLogic:
                    void deleteRequest(int requestID)
                    List<Object> getNewRequestsForSynchronisation()

            OnlineServiceMonitor:
                Online service keeps track of connection to Online Service.
                If the Online Service becomes unreachable (= does not send ACKs anymore), then start pinging the Online Service

                Interface OSUpdates used by BrokerLogic:
                    void onlineServiceUpdate()


    \paragraph{Av1: New Online Service responsibilities}
        The Online Service is able to autonomously detect failures of its individual internal communication components.\\
        The Online Service is able to detect that a SIoTIP gateway is not sending data anymore based on the expected synchronisation interval.\\
        The Online Service notifies the infrastructure manager and a SIoTIP system administrator when the outage of a SIoTIP gateway is detected.\\
        The failure of an internal SIoTIP Online Service component is detected within 30 seconds.\\
        The detection time for a failed SIoTIP gateway or channel depends on the transmission rate
        of the gateway. An outage is dened as 3 consecutive expected synchronisations that do not
        arrive within 1 minute of their expected arrival time.\\
        The infrastructure owner is notied within 5 minutes after the detection of an outage of their gateway.\\
        A SIoTIP system administrator should be notied within 1 minute after the detection
        of a simultaneous outage of more than 1\% of the registered gateways.\\

        GatewayBroker: Isolates communication-related concerns between the Online Service Gateways and along with OnlineServiceBroker on Gateways.
                       Forwards requests from one party to the other and transmits results and possible exceptions. \\
                       Sends acknowledgements for all messages sent by Gateways so that they can detect failures.
        GatewayBrokerMonitor: Monitor the communication component for communication with gateways on the Online Service.

        In GatewayBroker:
            BrokerLogic: Handles all functionality related to communication.
            GatewayMonitor: Monitors the connectivity status all gateways. Can detect that a gateway is not sending data anymore based on the expected synchronisation interval. If 3 consecutive expected synchronisations do not arrive within 1 minute of their expected arrival time,
                            this is detected as a gateway outage. When outages of gateways are detected, the infrastructure owners that own the gateways and a SIoTIP system administrator are notified.
                            When the connectivity status change of a Gateway is detected, this is saved in the DeviceDB.

        INTERFACES:
            GatewayBroker:
                HAS ALL INTERFACES PROVIDED BY COMPONENTS INSIDE OF IT

                interface CommunicationComponentMonitoring used by GatewayBrokerMonitor:
                    boolean check()

            BrokerLogic:
                ALL INTERFACES FROM OS TO GATEWAY ARE NOW TO THIS COMPONENT

                interface GWCommunication used by Gateway/BrokerLogic:

            GatewayMonitor:
                % table(int id, int countSynchronisationsMissed, DateTime nextSyncPeriod)
                %       32 bit, 4 bit, 64 bit
                %     5000 gateways
                %     5000*(32+4+64)/8/1024 kilobyte ~~ 64 kb

                interface GatewayUpdates used by BrokerLogic:
                    void gatewayUpdate(int gatewayID, DateTime time)

            NotificationHandler:
                Interface Notify used by GatewayMonitor

            DeviceDB:
                Interface DeviceMgmt used by GatewayMonitor:
                    void setGatewayUnreachable()
                    void getPercentageOfUnreachableGateways()


    In BROKER, many clients can make remote method invocations on
    specific remote component objects hosted by a server. Clients thus
    communicate with the server objects in a many-to-one fashion, and
    their functional interfaces are often statically typed. The remote
    method invocation style of communication provided by B ROKER is
    best suited for systems that try to hide the presence of the network.

    MESSAGING relaxes this coupling and typing: clients send dynamically
    typed messages to specific remote services that reside at commu-
    nication endpoints, not (necessarily) to specific methods. M ESSAGING
    thus enables many-to-one communication without statically pre-
    defining the interface dependencies of clients to services.Distribution Infrastructure

    PUBLISHER-SUBSCRIBER decouples an application’s components even
    more: they can exchange events in a one-to-many manner with-
    out knowing one another’s identity explicitly, and without having
    to make a request each time new events are available. P UBLISHER -
    S UBSCRIBER middleware is therefore responsible for tracking which
    subscribers receive specific events sent asynchronously by pub-
    lishers. Subscribers react when receiving an event by performing
    some action, but publishers do not directly initiate the execution of
    a specific method on the subscribers.

    BROKER makes invocations
    on remote component objects look and act as much as possible
    like invocations on component objects in the same address space
    as their clients. MESSAGING and PUBLISHER-SUBSCRIBER are most appropriate
    for integration scenarios in which multiple, independently
    developed and self-contained services or applications must collaborate
    and form a coherent software system.

    A message channel does not come without cost, however, since it
    needs memory, networking resources, and persistent storage to support
    GUARANTEED DELIVERY [HoWo03]. Developers must therefore plan
    and configure the number and types of message channels explicitly
    and thoughtfully to ensure the desired quality of service in a given
    system deployment. A well-designed set of message channels forms a
    MESSAGE BUS [HoWo03] that acts like a messaging API for the clients
    and services in the distributed system.

    Employed tactics and patterns: broker, heartbeat, ping/echo

    DEPLOYMENT RATIONALE:
        OnlineServiceBroker
        OnlineServiceBrokerMonitor

        GatewayBroker
        GatewayBrokerMonitor

        For both the Online Service and Gateways, the components used for communication
        (\texttt{OnlineServiceBroker}, \texttt{GatewayBroker}) are to be deployed on different nodes than
        their monitoring components (\texttt{OnlineServiceBrokerMonitor}, \texttt{GatewayBrokerMonitor}).
        Otherwise, if the node of a communication
        component fails, its monitoring component would also fail and thus
        nothing would be detected.

    Alternative for monitoring of gateways:
        Gateway updated come to GatewayMonitor
        We could make GatewayMonitor ping all the gateways
        However, this would increase traffic on the network

    Alternative for communication:
        Messaging, Publishher Subscriber


\subsection{Instantiation and allocation of functionality}
    This section lists the new components which instantiate our solutions
    described in the section above. For each component we note the quality
    attribute or use case that prompted us to create it. Descriptions about
    the components can be found under chapter \ref{ch:elements-datatypes}. \\

    \begin{itemize}
        \item BrokerLogic: Av1
        \item OnlineServiceBroker: Av1
        \item OnlineServiceBrokerMonitor: Av1
        \item GatewayBroker: Av1
        \item GatewayBrokerMonitor: Av1
        \item RequestStore: Av1
        \item GatewayMonitor: Av1
        \item OnlineServiceMonitor: Av1
    \end{itemize}


\subsection{Interfaces for child modules}
    This section lists new interfaces assigned to the components defined
    in the section above. Detailed information about each interface and
    its methods can be found under chapter \ref{ch:elements-datatypes}. \\

    \subsubsection{BrokerLogic}
        \begin{itemize}
            \item Communication
            \item OSCommunication
            \item GWCommunication
            \item OSMonitoring
        \end{itemize}

    \subsubsection{OnlineServiceBroker}
        \begin{itemize}
            \item CommunicationComponentMonitoring
        \end{itemize}

    \subsubsection{GatewayBroker}
        \begin{itemize}
            \item CommunicationComponentMonitoring
        \end{itemize}

    \subsubsection{RequestStore}
        \begin{itemize}
            \item Communication
        \end{itemize}

    \subsubsection{GatewayMonitor}
        \begin{itemize}
            \item GatewayUpdates
        \end{itemize}

    \subsubsection{OnlineServiceMonitor}
        \begin{itemize}
            \item OSMonitoring
        \end{itemize}

\subsection{Data type definitions}
    This section lists the new data types introduced during this decomposition.

    \begin{itemize}
        \item None
    \end{itemize}
