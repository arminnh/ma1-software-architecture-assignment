\section{U2: Easy installation}

    \subsection*{Key Decisions}

        \begin{itemize}
        	\item Gateways automatically register themselves with the Online Service when a connection is possible.
        	\item Motes and pluggable devices are automatically added in topologies and have their statuses updated
                  in the database when they are plugged in and detected.
            \item When a status change is detected for pluggable devices, the \texttt{DeviceManager} automatically
                  checks or makes the \texttt{ApplicationManagementLogic} check and activate\\deactivate
                  specific applications.
        \end{itemize}

    \subsection*{Rationale}
        One of U2's constraints is that Gateway installation should be straightforward.
        It is a given that the SIoTIP Gateways will automatically connect to the internet
        after an infrastructure owner quickly set it up. After that, when a Gateway
        is not registered with the Online Service, if will automatically register.
        Then, this will be reflected in the SIoTIP system, the Gateway will show up
        as 'unplaced' in the infrastructure owner's Gateway, and the infrastructure
        owner will receive a notification of this. \\
        When Gateways disconnect from the Online Service (due to e.g. communication channel failure),
        the \texttt{GatewayMonitor} on the Online Service notices this and updates the Gateway's
        status in the system. When a Gateway notices it can connect to the Online Service again
        (\texttt{OnlineServiceMonitor}), it will automatically connect again and be reactivated
        in the system (no topology update by infrastructure owner required).\\

        Next, U2 says that mote installation, should not require more configuration than
        adding it in the topology. When the \texttt{DeviceManager} notices a new mote
        (heartbeat coming from a new device),
        it registers it in the system, links it to the gateway, and adds it as 'unplaced' in the topology of
        its infrastructure owner. If the mote disconnects and reconnects on the same
        gateway, it will be automatically reactivated and take it's last status/position in the topology.\\

        For pluggable devices, the same holds as for motes, except that the new pluggable device is then
        linked to it's mote in the system. Also, pluggable devices get the location of their
        motes in the topology by default, but still start with status 'unplaced'.\\

        For both motes and pluggable devices, if they are connected to a new device (gateway or mote)
        then the data concerning their last location and status is deleted.\\

        Lastly, U2 states that applications should work out of the box if the required sensors and
        actuators are available. For this, each time the \texttt{DeviceManager} recognizes a new
        pluggable device, it starts a check for application instances
        (related to customer organisations associated with the gateway's infrastructure owner)
        that can be activated.

    \subsection*{Considered Alternatives}
        \paragraph{Alternative for choice easy Gateway installation}
            To avoid having the infrastructure owner to place a new Gateway in the topology (to be more in line with U2),
            we could let the Gateway be linked to a location in the topology at the moment infrastructure owner buys the
            Gateway. Then, after installing the Gateway, it will automatically register and show up as 'placed' in the topology.
            However, this would just move the problem of placing the Gateway in the topology instead of solving the problem.
            It is then better to make the infrastructure owner place the Gateway after physically installing it. This way
            the infrastructure owner kan keep some spare Gateways that are not linked to a location and install them when
            one of the Gateways break.
