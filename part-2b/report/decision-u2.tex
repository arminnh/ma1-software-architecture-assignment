\section{U2: Easy installation}

    \subsection*{Key Decisions}

        \begin{itemize}
        	\item Gateways automatically register themselves with the Online Service when a connection is possible.
        	\item Motes and pluggable devices are automatically added in topologies and have their statuses updated
                  in the database when they are plugged in and detected.
            \item When a status change is detected for pluggable devices, the \texttt{DeviceManager} automatically
                  checks or makes the \texttt{ApplicationManagementLogic} check and activate\\deactivate
                  specific applications.
        \end{itemize}
        \emph{Employed tactics and patterns:} \ldots

    \subsection*{Rationale}
        \paragraph{U2: Gateway installation}
            The gateway should not require any configuration, other than being connected
            to the local wired or WiFi network, after it is plugged into an electrical
            socket. An infrastructure owner should be able get the SIoTIP gateway
            up-and-running (connected) within 10 minutes given that the information
            (e.g. WiFi SSID and passphrase) is available to the person responsible for
            the installation. \\

            A connection to the internet is a constraint of the GatewayFacade.
            After the gateway is connected to the internet (we don't model this),
            it connects to the gateway (we don't model this?) and registers itself (we model this). \\
            When an infrastructure owner orders a gateway, that gateway is linked to the IO.
            Gateway was already in the DeviceDB, but it was not linked to anyone. It has a gatewayID => unique identifier gatewayID same like motes.

            Important info related to gateways: GatewayID (new class), infrastructureOwnerID, IPAddress, status (active/inactive), location (in topology table)
                GatewayInfo(int gatewayID, int manufacturerID, int productID, int infrastructureOwnerID, IPAddress ip, int status)

            gateway registers with online service:
                % GatewayFacade -> DeviceDB: interface DeviceMgmt: registerGateway(int gatewayID, IPAddress address)
                %     Effect: Sets a gateway's status to 'active' and updates its IP address
                %     \item Created for: U2 - gateway installation

                We cannot link the gateway to an exact location for the infrastructure owner, because he might be managing multiple buildings
                an IP addresses can be dynamic. If we let the IO choose for which building the gateway is, then this is bad for usability and
                he has to configure anyways. Also in that case he cannot buy spare gateways, unless he buys spare gateways for every gateway in the building

        \paragraph{U2: Mote installation}
            Installing a new mote should not require more configuration than adding it
            to the topology. Adding new motes, sensors or actuators should not involve
            more than just starting motes, and plugging devices into motes – plug-and-play!
            Reintroducing a previously known mote, with the same pluggable devices attached to it,
            should not require any configuration. It is automatically re-added on
            its last known location on the topology. The attached pluggable devices
            are automatically initialised and configured with their last known
            configuration and access rights. \\
            Thing that need to happen automatically:
            *) mote should find the gateway (mote sends a broadcast message->ReceiveBroadcast) => this is done automatically? see remarks of the use case
            *) gateway should register the mote (DeviceManager update, store entry in DB)
            *) on reintroduction of motes: DeviceManager notices this, makes the gateway send a message to online service to reuse some old topology

        \paragraph{U2: Pluggable device installation}
            Adding new sensors or actuators should require no further customer
            actions besides plugging it into the mote. Configurable sensors and
            actuators should have a working default configuration.
            Pluggable devices added to an already known mote are automatically
            added in the right location on the topology.
            Making (initialised) sensors and actuators available to customer
            organisations and applications should not require more effort than
            configuring access rights (cf. UC9). \\
            *) After devices are plugged in: connect to mote, set up default configurations
            *) if the mote is already known, the device is added to the right location on the topology
            *) need something for configuration of access rights, can only happen for initialised devices

            *) for reactivating last configurations: just set status to active and don't change configuration field, it will still be the same as in the past
                alternative: current\_configuration and last\_configuration in DB
                alternative: store all configurations on Gateway -> but it has bad resources
                alternative: store all versions on DeviceDB -> but lots of useless data then = extra work for db

            *) Pluggable devices added to an already known mote are automatically added in "the right location" on the topology.
                what exactly is a location?
                => when a pluggable device is connected to a new mote, the pluggable device gets the location of the mote by default

        \paragraph{U2: Easy applications}
            Applications should work out of the box if the required sensors and
            actuators are available. Only when mandatory end-user roles must be
            assigned, additional explicit configuration actions are required
            from a customer organisation (cf. UC17, UC19). \\
            *) if there is a subsription and new hardware is plugged in: need something to check
               if some application can be activated now => see UC6: checkApplicationsForActivationForInfrastructureOwner

    \subsection*{Considered Alternatives}
        \paragraph{Alternative(s) for choice 1} Explain what alternative(s) you
        considered for this design choice and why they where not selected.

    \subsection*{Deployment Decisions}
        \ldots

    \subsection*{Considered Deployment Alternatives}
        \ldots
