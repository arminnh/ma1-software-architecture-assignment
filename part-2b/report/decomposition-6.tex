\section{Decomposition 6: P1, UC1, UC2, UC3, UC5, UC7, UC8, UC16, UC20 (Elements/Subsystem to decompose/expand)}


\subsection*{Selected architectural drivers}
    The non-functional drivers for this decomposition are:
    \begin{itemize}
    	\item \emph{P1}: Large number of users
    \end{itemize}

    The related functional drivers are:
    \begin{itemize}
        \item \emph{UC1}: Register a customer organisation
        \item \emph{UC2}: Register an end-user
        \item \emph{UC3}: Unregister an end-user
        \item \emph{UC5}: Uninstall mote
        \item \emph{UC7}: Remove a pluggable device from its mote
        \item \emph{UC8}: Initialise a pluggable device
        \item \emph{UC10}: Consult and configure the topology
        \item \emph{UC16}: Consult notification message
        \item \emph{UC20}: Unsubscribe from application
    \end{itemize}


\subsection*{Architectural design}
    This section describes what needs to be done to satisfy the requirements for
    this decomposition and how involved problems/obstacles are solved.

    https://www.alertra.com/blog/2010/improve-availability-performance-using-database-replication
    https://serverfault.com/questions/10781/what-are-the-performance-implications-for-using-sql-server-replication
    "
    Ik dacht aan misschien de meeste van de componenten gewoon duplicaten maar dan is de DB wel u bottleneck en dan moet ge zo'n systeem van distributed systems gebruiken om ook de databases juist te kunnen replicaten.
    Maar daarnaast echt geen idee ivm performance. Ge kunt wel bullshit van de "tactics" in u rationale zetten zoals the developers have to "increase computation efficiency" and "reduce computational overhead".
    Manage event rate en scheduling policy kunt ge misschien wel gebruiken om er voor te zorgen dat bepaalde taken gebeuren op een moment dat de load op de online service wat lager is maar ik weet Ni echt welke taken
    "

    *) Most components can be duplicated for load balancing
    *) Related DB's should use a DBMS that can be scaled horizontally.
    *) DB calls caused by IO vs CO have been mostly split up in DeviceDB and OtherDataDB, so those 2 groups of
       users don't influence each other too much


    \paragraph{P1: Problem title}
        The SIoTIP Online Service replies to the service requests of the
        infrastructure owner and customer organisations.

    \paragraph{P1: Problem title}
        The Online Service processes the data received from the gateways.

    \paragraph{P1: Problem title}
        The application execution subsystem should be able to execute an increasing
        number of active applications.

    \paragraph{P1: Problem title}
        The initial deployment of SIoTIP should be able to deal with at least 5000
        gateways in total, and should be provisioned to service at least 3000
        registered users simultaneously connected to SIoTIP.
        -> 5000 gateways * 4 motes per gateway * 3 devices per mote = 60000 devices
        -> Keep communication with gateways at a minimum
           e.g. gateway messages are of type "send data", "new device connected", ...
        -> LOAD BALANCING

    \paragraph{P1: Problem title}
        Scaling up to service an increasing amount of infrastructure owners,
        customers organisations and applications should (in worst case) be linear ;
        i.e. it should not require proportionally more resources (machines, etc.)
        than the initial amount of resources provisioned per customer
        organisation/infrastructure owner and per gateway.


    % UC1:
    %     1. The primary actor requests the (infrastructure owner-specific) registration form.
    %         UnregisteredUserClient -> UnregisteredUserFacade: interface Registration: boolean canCustomerOrganisationRegister(int infrastructureOwnerID, string hash)
    %         UnregisteredUserFacade -> UserManager: interface UserMgmt: boolean canCustomerOrganisationRegister(int infrastructureOwnerID, string hash)
    %         UserManager -> OtherDataDB: interface UserMgmt: boolean canCustomerOrganisationRegister(int infrastructureOwnerID, string hash)
    %
    %     2. The system asks the primary actor for an e-mail address.
    %
    %     3. The primary actor enters a company e-mail address.
    %
    %         UnregisteredUserClient -> UnregisteredUserFacade: interface Registration: boolean createEmailAddressForRegistration(string emailAddress, int infrastructureOwnerID)
    %
    %     4. The system verifies the provided e-mail address, and if the e-mail address is valid, the system
    %        informs the primary actor that a confirmation e-mail was sent to the entered e-mail address
    %        with further instructions.
    %
    %         UnregisteredUserFacade -> UserManager: interface UserMgmt: boolean createEmailAddressForRegistration(string emailAddress, int infrastructureOwnerID)
    %         UserManager -> OtherDataDB: interface UserMgmt: boolean createEmailAddressForRegistration(string emailAddress, int infrastructureOwnerID)
    %
    %     5. The primary actor confirms his or her e-mail address by following the link in the received e-mail.
    %
    %         UnregisteredUserClient -> UnregisteredUserFacade: interface Registration: boolean confirmEmail(string emailAddress, string randomString)
    %         UnregisteredUserFacade -> UserManager: interface UserMgmt: boolean confirmEmail(string emailAddress, string randomString)
    %         UserManager -> OtherDataDB: interface UserMgmt: boolean confirmEmail(string emailAddress, string randomString)
    %
    %     6. The system asks the primary actor for the remaining information:
    %        (a) Credentials (e.g., unique username and password)
    %        (b) Full name of the company and primary address
    %        (c) Billing and payment information (e.g. credit card information or bank account number,
    %            and, optionally, invoicing address)
    %        (d) Contact phone number
    %        (e) Notification preferences: SMS and/or e-mail.
    %
    %     7. The primary actor provides the requested information.
    %         UnregisteredUserClient -> UnregisteredUserFacade: interface Registration: int createCustomerOrganisation(Map<string, string> data)
    %
    %     8. The system verifies the provided information, and, if the provided information is correct, the
    %        system creates a new customer organisation profile.
    %
    %         UnregisteredUserFacade -> UserManager: interface UserMgmt: int createCustomerOrganisation(Map<string, string> data)
    %         UserManager -> OtherDataDB: interface UserMgmt: int createCustomerOrganisation(Map<string, string> data)
    %
    %     9. The system notifies the primary actor that the registration is successful.
    %         => return value of step 7


    % UC2:
    %     1. The primary actor indicates to the system that he or she wants to register an end-user.
    %         IN CustomerOrganisationClient
    %
    %     2. The system asks the primary actor for details about the end-user to be registered:
    %        Name, Phone number (for SMS) and/or e-mail address
    %        IN CustomerOrganisationClient
    %
    %     3. The primary actor supplies the requested information.
    %         CustomerOrganisationClient -> CustomerOrganisationFacade: interface UserMgmt: boolean registerEndUser(int customerOrganisationID, Map<string, string> data)
    %
    %     4. The system verifies the provided e-mail address or phone number, and checks that the end-user
    %         does not exist in the system already. If the provided information is valid, the system stores
    %         the end-user information and associates it with the primary actor.
    %
    %         CustomerOrganisationFacade -> UserManager: interface UserMgmt: boolean createEndUser(int customerOrganisationID, Map<string, string> data)
    %         UserManager -> OtherDataDB: interface UserMgmt: boolean createEndUser(int customerOrganisationID, Map<string, string> data)
    %
    %     5. The system informs the primary actor that the end-user was registered successfully.
    %         => return value of createEndUser


    % UC3:
    %     1. The primary actor indicates to the system that they want to unregister an end-user.
    %         CustomerOrganisationClient -> CustomerOrganisationFacade: interface UserMgmt: List<User> getEndUsers(int customerOrganisationID)
    %
    %     2. The system retrieves all end-users associated with the primary actor and presents this list to the primary actor.
    %
    %         CustomerOrganisationFacade -> UserManager: interface UserMgmt: List<User> getEndUsers(int customerOrganisationID) ALREADY HAVE THIS?
    %         UserManager -> OtherDataDB: interface UserMgmt: List<User> getEndUsers(int customerOrganisationID) ALREADY HAVE THIS?
    %
    %     3. The primary actor indicates which end-user she wants to unregister.
    %         CustomerOrganisationClient -> CustomerOrganisationFacade: interface UserMgmt: void unregisterEndUser(int customerOrganisationID, int userID)
    %
    %     4. The system marks the profile of the end-user as 'inactive'
    %
    %         CustomerOrganisationFacade -> UserManager: interface UserMgmt: void softDeleteEndUser(int customerOrganisationID, int userID)
    %         UserManager -> OtherDataDb: interface UserUserMgmt: void softDeleteEndUser(int customerOrganisationID, int userID)
    %
    %     5. The system checks if any applications should be deactivated because of end-user assignment to mandatory roles (Include: UC18: Check and deactivate applications).
    %         CustomerOrganisationFacade -> ApplicationManagementLogic: interface FrontEndAppReuqestss: checkApplicationsForDeactivationForCustomerOrganisations(List<int> IDs)
    %
    %     6. The system informs the primary actor that the end-user is now unregistered from SIoTIP.
    %         => return value of unregisterEndUser


    % UC5:
    %     1. The primary actor physically stops the mote (by unplugging, powering off, or taking it out of the vicinity of the network).
    %
    %     2. When the system no longer receives heartbeat messages,
    %         Already done: Mote -> DeviceManager: interface Heartbeat
    %         Use case starts with DeviceManager noticing that a mote is not sending heatbeats anymore.
    %
    %         -it marks the mote as 'inactive' in the topology.
    %             DeviceManager -> TopologyManager: interface TopologyMgmt: void deactivateMote(int moteID)
    %             TopologyManager -> DeviceDB: interface DBTopologyMgmt: void deactivateMote(int moteID)
    %             DeviceManager -> DeviceDB: interface DBDeviceMgmt: void deactivateMote(int moteID)
    %
    %         -it looks up the pluggable devices that were attached to the affected mote and marks these as 'inactive' in the topology.
    %             DeviceManager -> TopologyManager: interface TopologyMgmt: void deactivatePluggableDevices(List<PluggableDeviceID> pIDs)
    %             TopologyManager -> DeviceDB: interface DBTopologyMgmt: void deactivatePluggableDevices(List<PluggableDeviceID> pIDs)
    %             DeviceManager -> DeviceDB: interface DBDeviceMgmt: void deactivatePluggableDevices(List<PluggableDeviceID> pIDs)
    %
    %         -it checks for and deactivates applications that are affected by the unavailability of the pluggable devices (Include: UC18: Check and deactivate applications).
    %             DeviceManager -> ApplicationManagementLogic: interface GWAppInstanceMgmt: void checkApplicationsForDeactivationForPluggableDevices(List<PluggableDeviceID> pIDs)
    %
    %         -It sends a notification to the primary actor (Include: UC15: Send notification).
    %             DeviceManager -> NotificationHandler: interface Notify: notifyMoteUninstalled()


    % UC7:
    %     1. The primary actor unplugs the pluggable device from its mote.
    %
    %     2. A message is sent from the mote from which the pluggable device was removed to the system, containing the identifier of the removed pluggable device.
    %         Mote -> DeviceManager: interface DeviceManagement: void pluggableDeviceRemoved(PluggableDeviceID pID)
    %         (SAME FOR UC6: pluggableDevicePluggedIn(moteInfo, pID))
    %
    %     3. The system
    %        { marks the pluggable device as 'inactive',
    %         DeviceManager -> DeviceDB: interface DBDeviceMgmt: void deactivatePluggableDevice(PluggableDeviceID pID)
    %
    %        { updates the topology,
    %         DeviceManager -> TopologyManager: interface TopologyMgmt: void deactivatePluggableDevice(PluggableDeviceID pID)
    %         TopologyManager -> DeviceDB: interface DBTopologyMgmt: void deactivatePluggableDevice(PluggableDeviceID pID)
    %
    %        { it checks and deactivates applications for which the pluggable device was essential (Include: UC18: Check and deactivate applications).
    %         DeviceManager -> ApplicationManagementLogic: interface GWAppInstanceMgmt: void checkApplicationsForDeactivationForPluggableDevice(PluggableDeviceID pID)


    % UC8:
    %     1. The primary actor indicates that they want to initialise the newly attached pluggable for use.
    %         Client goes to UC10
    %
    %     2. The system allows the primary actor to configure the pluggable device in the topology (Include: UC10), e.g. to set its physical location.
    %
    %         See and create UC10
    %
    %     3. The system marks the pluggable device as 'active', and informs the primary actor that the pluggable is ready for use.
    %
    %         InfrastructureOwnerFacade -> InfrastructureOwnerManager: interface IOMgmt: void initialisePluggableDevice(PluggableDeviceID pID)
    %         InfrastructureOwnerManager -> DeviceDB: interface DBIODeviceMgmt: void initialisePluggableDevice(PluggableDeviceID pID)
    %         InfrastructureOwnerManager -> DeviceManager: interface DBIODeviceMgmt: void initialisePluggableDevice(PluggableDeviceID pID)
    %         DeviceManager -> DeviceManager: interface GWAppInstanceMgmt: void checkApplicationsForActivationForInfrastructureOwner(int infrastructureOwnerID)
    %
    %     4. The system allows the primary actor to configure the access rights for the pluggable device (Include: UC9: Configure pluggable device access rights).
    %
    %         See UC9
    %         InfrastructureOwnerClient -> InfrastructureOwnerFacade: interface AccessRights


    % UC10: Consult and configure the topology
    %     1. The primary actor indicates that they want to manage the topology.
    %
    %         InfrastructureOwnerClient -> InfrastructureOwnerFacade: interface TopologyMgmt: List<RoomTopology> getTopologyOfInfrastructureOwner(int infrastructureOwnerID)
    %
    %     2. The system presents the topology for the primary actor.
    %        { The topology shows a representation of the gateways, and the motes and pluggable devices associated with each gateway,
    %        and specifies key relationships between these devices, for example physical location information,
    %        or whether pluggable device X and pluggable device Y are interchangeable.
    %
    %         InfrastructureOwnerFacade -> TopologyManager: interface TopologyMgmt: List<RoomTopology> getTopologyOfInfrastructureOwner(int infrastructureOwnerID)
    %         TopologyManager -> DeviceDB: interface DBTopologyMgmt: List<RoomTopology> getTopologyOfInfrastructureOwner(int infrastructureOwnerID)
    %
    %        { New motes and pluggable devices that are not yet included are presented on the side.
    %
    %         InfrastructureOwnerFacade -> TopologyManager: interface TopologyMgmt: List<MoteInfo> getUnplacedMotes(int infrastructureOwnerID)
    %         TopologyManager -> DeviceDB: interface DBTopologyMgmt: List<MoteInfo> getUnplacedMotes(int infrastructureOwnerID)
    %
    %         InfrastructureOwnerFacade -> TopologyManager: interface TopologyMgmt: List<PluggableDeviceInfo> getUnplacedPluggableDevices(int infrastructureOwnerID)
    %         TopologyManager -> DeviceDB: interface DBTopologyMgmt: List<PluggableDeviceInfo> getUnplacedPluggableDevices(int infrastructureOwnerID)
    %
    %     3. The primary actor updates the topology.
    %        { This could involve changing the location of or relations between motes and pluggable devices,
    %        or extending the topology with motes and pluggable devices that were not yet included.
    %
    %         FOR INSIDE CLIENT
    %
    %     4. The primary actor indicates that they are done updating the topology.
    %
    %         InfrastructureOwnerClient -> InfrastructureOwnerFacade: interface TopologyMgmt: void updateTopology(int infrastructureOwnerID, List<RoomTopology> updatedTopology)
    %
    %     5. The system
    %        { stores the reconfigured topology and informs the primary actor that the reconfigured topology is now being used.
    %
    %         InfrastructureOwnerFacade -> TopologyManager: interface TopologyMgmt: void updateTopology(int infrastructureOwnerID, List<RoomTopology> updatedTopology)
    %         TopologyManager -> DeviceDB: interface DBTopologyMgmt: void updateTopology(int infrastructureOwnerID, List<RoomTopology> updatedTopology)
    %
    %        { activates 'inactive' applications (Include: UC17 : Activate an application).
    %         InfrastructureOwnerFacade -> ApplicationManagementLogic: interface FrontEndAppRequests: void checkApplicationsForActivationForInfrastructureOwner(int infrastructureOwnerID)
    %
    %        { checks for applications that require deactivation (Include: UC18 : Check and deactivate applications).
    %         InfrastructureOwnerFacade -> ApplicationManagementLogic: interface FrontEndAppRequests: void checkApplicationsForDectivationForInfrastructureOwner(int infrastructureOwnerID)
    %
    %     3a. If the primary actor wants to view status details for a specific gateway/mote/pluggable device:
    %         1. The system composes a detailed overview of the status of the selected gateway/mote/pluggable device (e.g., indicating that the device was unavailable between 2 and 3 a.m.) and presents it.
    %             InfrastructureOwnerClient -> InfrastructureOwnerFacade: interface TopologyMgmt: List<DeviceStatus> getStatusDetailsForGateway(int gatewayID)
    %             InfrastructureOwnerFacade -> TopologyManager: interface TopologyMgmt: List<DeviceStatus> getStatusDetailsForGateway(int gatewayID)
    %             TopologyManager -> DeviceDB: interface DBTopologyMgmt: List<DeviceStatus> getStatusDetailsForGateway(int gatewayID)
    %
    %             InfrastructureOwnerClient -> InfrastructureOwnerFacade: interface TopologyMgmt: List<DeviceStatus> getStatusDetailsForMote(int moteID)
    %             InfrastructureOwnerFacade -> TopologyManager: interface TopologyMgmt: List<DeviceStatus> getStatusDetailsForMote(int moteID)
    %             TopologyManager -> DeviceDB: interface DBTopologyMgmt: List<DeviceStatus> getStatusDetailsForMote(int moteID)
    %
    %             InfrastructureOwnerClient -> InfrastructureOwnerFacade: interface TopologyMgmt: List<DeviceStatus> getStatusDetailsForPluggableDevice(PluggableDeviceID pID)
    %             InfrastructureOwnerFacade -> TopologyManager: interface TopologyMgmt: List<DeviceStatus> getStatusDetailsForPluggableDevice(PluggableDeviceID pID)
    %             TopologyManager -> DeviceDB: interface DBTopologyMgmt: List<DeviceStatus> getStatusDetailsForPluggableDevice(PluggableDeviceID pID)


    % UC16:
    %     1. The primary actor indicates that he/she/they wants to consult a notification.
    %
    %         ApplicationProviderClient/CustomerOrganisationClient/InfrastructureOwnerClient/SysAdminClient -> RegisteredUserFacade: interface Notifications: List<Notification> consultNotifications(int userID)
    %
    %     2. The system retrieves all notifications for the primary actor and presents them to the primary actor, e.g. as a table or list.
    %
    %         RegisteredUserFacade -> NotificationHandler: interface Notifications: List<Notification> getNotifications(int userID)
    %         NotificationHandler -> OtherDataDB: interface DBNotificationMgmt: List<Notification> getNotifications(int userID)
    %
    %     3. The primary actor selects a notification.
    %
    %         ApplicationProviderClient/CustomerOrganisationClient/InfrastructureOwnerClient/SysAdminClient -> RegisteredUserFacade: interface Notifications: Notification consultNotification(int userID, int notificationID)
    %
    %     4. The system presents the details for the selected notification to the primary actor. This includes:
    %         { A description of the event that occurred.
    %         { Date and time the event occurred.
    %         { What triggered the event (e.g., the cause of an application that became inactive)
    %         { The status of the notification (e.g. 'sent', 'delivered', 'read')
    %         { The communication channel by which the notification was sent
    %
    %         RegisteredUserFacade -> NotificationHandler: interface Notifications: Notification getDetailedNotification(int userID, int notificationID)
    %         NotificationHandler -> OtherDataDB: interface DBNotificationMgmt: Notification getDetailedNotification(int userID, int notificationID)
    %
    %     5. The system marks the notification as 'read'.
    %         NotificationHandler -> OtherDataDB: interface DBNotificationMgmt: void setNotificationRead(int notificationID)


    % UC20:
    %     1. The primary actor indicates they want to unsubscribe from an application.
    %         CustomOrganisationClient -> CustomerOrganisationFacade: interface SubscriptionMgmt: List<Subscription> getSubscriptions(int customerOrganisationID)
    %
    %     2. The system looks up a list of applications the primary actor is subscribed to, and presents this overview to the primary actor, e.g. as a list or table.
    %         CustomerOrganisationFacade -> SubscriptionManager: interface SubscriptionMgmt: List<Subscription> getSubscriptions(int customerOrganisationID)
    %         SubscriptionManager -> OtherDataDB: interface DBSubscriptionMgmt: List<Subscription> getSubscriptions(int customerOrganisationID)
    %
    %     3. The primary actor indicates which application they want to unsubscribe from.
    %         CustomOrganisationClient -> CustomerOrganisationFacade: interface SubscriptionMgmt: void unsubscribeFromApplication(int customerOrganisationID, int subscriptionID, int applicationInstanceID)
    %
    %     4. The system
    %        (i) ends the subscription for the selected application,
    %
    %             CustomerOrganisationFacade -> SubscriptionManager: interface SubscriptionMgmt: void softDeleteSubscription(int subscriptionID)
    %             SubscriptionManager -> OtherDataDB: interface DBSubscriptionMgmt: void softDeleteSubscription(int subscriptionID)
    %
    %         (ii) deactivates the application, and
    %
    %             SubscriptionManager -> ApplicationManagementLogic: interface FrontEndAppRequests: void softDeleteApplicationInstance(int applicationInstanceID)
    %             ApplicationManagementLogic -> OtherDataDB: interface DBAppMgmt: void softDeleteApplicationInstance(int applicationInstanceID)
    %
    %             CustomerOrganisationFacade -> ApplicationManagementLogic: interface FrontEndAppRequests: void destroyApplicationInstance(int applicationInstanceID)
    %             ApplicationManagementLogic -> ApplicationContainerManager: interface AppInstanceMgmt: void destroyApplicationInstance(int applicationInstanceID)
    %
    %         (iii) constructs an invoice for outstanding payments for that application (cf. UC21: Send invoice).
    %
    %             SubscriptionManager -> InvoiceManager: interface InvoiceMgmt: void constructInvoiceForApplicationInstance(int applicationInstanceID, int customerOrganisationID)
    %             InvoiceManager -> OtherDataDB: interface InvoiceMgmt: void createInvoice(Invoice invoice)
    %             InvoiceManager -> ThirdPartyInvoicingService: interface InvoiceMgmt: void sendInvoice(Map<string, string> invoiceData)
    %
    %     5. The system informs the primary actor that the unsubscription was successful.
    %         => return value of unsubscribeFromApplication


\subsection*{Instantiation and allocation of functionality}
    This section lists the new components which instantiate our solutions
    described in the section above. For each component we note the quality
    attribute or use case that prompted us to create it. Descriptions about
    the components can be found under chapter \ref{ch:elements-datatypes}. \\

    \begin{itemize}
        \item ApplicationProviderClient: UC16
        \item ApplicationProviderFacade: UC16
        \item RegisteredUserFacade: UC16
        \item SystemAdministratorFacade: UC16
        \item UnregisteredUserClient: UC1
        \item UnregisteredUserFacade: UC1
    \end{itemize}


\subsection*{Interfaces for child modules}
    This section lists new interfaces assigned to the components defined
    in the section above. Detailed information about each interface and
    its methods can be found under chapter \ref{ch:elements-datatypes}.

    \subsubsection{ApplicationProviderFacade}
        \begin{itemize}
            \item Notifications
        \end{itemize}

    \subsubsection{CustomerOrganisationFacade}
        \begin{itemize}
            \item Notifications
            \item UserMgmt
        \end{itemize}

    \subsubsection{InfrastructureOwnerFacade}
        \begin{itemize}
            \item Notifications
            \item TopologyMgmt
        \end{itemize}

    \subsubsection{NotificationHandler}
        \begin{itemize}
            \item Notifications
        \end{itemize}

    \subsubsection{RegisteredUserFacade}
        \begin{itemize}
            \item Notifications
        \end{itemize}

    \subsubsection{SystemAdministratorFacade}
        \begin{itemize}
            \item Notifications
        \end{itemize}

    \subsubsection{UnregisteredUserFacade}
        \begin{itemize}
            \item Registration
        \end{itemize}

    \subsubsection{UserManager}
        \begin{itemize}
            \item UserMgmt
        \end{itemize}

\subsection*{New data types}
    This section lists the new data types introduced during this decomposition.

    \begin{itemize}
        \item Invoice
        \item EmailAddressAlreadyExistsException
        \item EmailAddressInvalidException
        \item EndUserAlreadyExistsException
        \item PaymentInformationInvalidException
        \item PhoneNumberInvalidException
        \item UserNameAlreadyExistsException
    \end{itemize}
