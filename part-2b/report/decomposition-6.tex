\section{Decomposition 6: P1, UC1, UC2, UC3, UC5, UC7, UC8, UC16, UC20 (Elements/Subsystem to decompose/expand)}


\subsection{Selected architectural drivers}
    The non-functional drivers for this decomposition are:
    \begin{itemize}
    	\item \emph{P1}: Large number of users
    \end{itemize}

    The related functional drivers are:
    \begin{itemize}
        \item \emph{UC1}: Register a customer organisation \\
        \item \emph{UC2}: Register an end-user \\
        \item \emph{UC3}: Unregister an end-user \\
        \item \emph{UC5}: Uninstall mote \\
        \item \emph{UC7}: Remove a pluggable device from its mote \\
        \item \emph{UC8}: Initialise a pluggable device \\
        \item \emph{UC16}: Consult notification message \\
        \item \emph{UC20}: Unsubscribe from application \\
    \end{itemize}


\subsection{Architectural design}
    This section describes what needs to be done to satisfy the requirements for
    this decomposition and how involved problems/obstacles are solved.

    https://www.alertra.com/blog/2010/improve-availability-performance-using-database-replication
    https://serverfault.com/questions/10781/what-are-the-performance-implications-for-using-sql-server-replication
    "
    Ik dacht aan misschien de meeste van de componenten gewoon duplicaten maar dan is de DB wel u bottleneck en dan moet ge zo'n systeem van distributed systems gebruiken om ook de databases juist te kunnen replicaten.
    Maar daarnaast echt geen idee ivm performance. Ge kunt wel bullshit van de "tactics" in u rationale zetten zoals the developers have to "increase computation efficiency" and "reduce computational overhead".
    Manage event rate en scheduling policy kunt ge misschien wel gebruiken om er voor te zorgen dat bepaalde taken gebeuren op een moment dat de load op de online service wat lager is maar ik weet Ni echt welke taken
    "


    \paragraph{P1: Problem title}
        The SIoTIP Online Service replies to the service requests of the
        infrastructure owner and customer organisations.

    \paragraph{P1: Problem title}
        The Online Service processes the data received from the gateways.

    \paragraph{P1: Problem title}
        The application execution subsystem should be able to execute an increasing
        number of active applications.

    \paragraph{P1: Problem title}
        The initial deployment of SIoTIP should be able to deal with at least 5000
        gateways in total, and should be provisioned to service at least 3000
        registered users simultaneously connected to SIoTIP.
        -> 5000 gateways * 4 motes per gateway * 3 devices per mote = 60000 devices
        -> Keep communication with gateways at a minimum
           e.g. gateway messages are of type "send data", "new device connected", ...
        -> LOAD BALANCING

    \paragraph{P1: Problem title}
        Scaling up to service an increasing amount of infrastructure owners,
        customers organisations and applications should (in worst case) be linear ;
        i.e. it should not require proportionally more resources (machines, etc.)
        than the initial amount of resources provisioned per customer
        organisation/infrastructure owner and per gateway.


\subsection{Instantiation and allocation of functionality}
    This section lists the new components which instantiate our solutions
    described in the section above. For each component we note the quality
    attribute or use case that prompted us to create it. Descriptions about
    the components can be found under chapter \ref{ch:elements-datatypes}. \\

    \begin{itemize}
        \item Component: (P1 or UC)
    \end{itemize}


\subsection{Interfaces for child modules}
    This section lists new interfaces assigned to the components defined
    in the section above. Detailed information about each interface and
    its methods can be found under chapter \ref{ch:elements-datatypes}. \\

    \subsubsection{Component}
        \begin{itemize}
            \item Interface
        \end{itemize}

\subsection{Data type definitions}
    This section lists the new data types introduced during this decomposition.

    \begin{itemize}
        \item DateTime: Represents an instant in time, typically expressed as a date and time of day.
    \end{itemize}
