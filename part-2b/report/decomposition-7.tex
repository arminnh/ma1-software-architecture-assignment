\section{Decomposition 7: UC28, UC29 (Elements/Subsystem to decompose/expand)}
    At this point, all quality attributes have been handled. The remaining
    decompositions handle all of the use cases that are left. The order
    is based on the priority of the use cases.


\subsection{Selected architectural drivers}
    The functional drivers are:
    \begin{itemize}
        \item \emph{UC28}: Log in \\
        \item \emph{UC29}: Log out \\
    \end{itemize}


\subsection{Architectural design}
    This section describes what needs to be done to satisfy the requirements for
    this decomposition and how involved problems/obstacles are solved.

    \paragraph{UC: Problem title}
        Short description of the problem.\\
        Solution for the problem.

        WHen you check credentials, return a string sessionID to the client.
        The client sends this sessionID with every request.
        The Online Service checks the sessionID to determine if the user is logged in.
        Monika: store sessionID in otherDataDB
        PMS: store sessionID in sessionDB
        Check what is better for us
        http://shiflett.org/articles/storing-sessions-in-a-database
        
        UC28 
        1. The primary actor indicates he or she want to log in to the system and provides authentication
            credentials, for example a user name and password.
            Clients -> UserFacade: interface Authentification:
                            login()
            UserFacade -> AuthentificationManager: inteface  Authentification:
                            login(string name, string passwod)               
        2. The system veries the provided authentication credentials.
            AuthentificationManager -> SessionDB: interface Authentification:
                                    controlCredentials(string name, string passwod)
        3. If the provided credentials are correct, the system conrms successful login to the primary
            actor and logs him or her in.
            
         3a. If the provided credentials are incorrect, the system noties the primary actor of this. Continue
            from step 1.
            
            { This use case abstracts the used method of authentication. For example, authentication can
               be done using techniques such as username-password, security tokens, cryptographic keys, etc.
               The employed method depends, among others, on the type of primary actor.
            { Registration of customer organisation and end-users is described in UC1. Registration of
               infrastructure owners, SysAdmins, and application providers is done out-of-band, and hence
               not described in use cases.
               
               
         UC29:
            1. The primary actor indicates he or she wants to log out of the system.
            2. The system logs the primary actor out and indicates success to the primary actor.


\subsection{Instantiation and allocation of functionality}
    This section lists the new components which instantiate our solutions
    described in the section above. For each component we note the quality
    attribute or use case that prompted us to create it. Descriptions about
    the components can be found under chapter \ref{ch:elements-datatypes}. \\

    \begin{itemize}
        \item Component: (Relevant UC)
    \end{itemize}


\subsection{Interfaces for child modules}
    This section lists new interfaces assigned to the components defined
    in the section above. Detailed information about each interface and
    its methods can be found under chapter \ref{ch:elements-datatypes}. \\

    \subsubsection{Component}
        \begin{itemize}
            \item Interface
        \end{itemize}

\subsection{Data type definitions}
    This section lists the new data types introduced during this decomposition.

    \begin{itemize}
        \item DateTime: Represents an instant in time, typically expressed as a date and time of day.
    \end{itemize}
