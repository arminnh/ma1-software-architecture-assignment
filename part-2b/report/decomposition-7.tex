\section{Decomposition 7: UC28, UC29 (Elements/Subsystem to decompose/expand)}
    At this point, all quality attributes have been handled. The remaining
    decompositions handle all of the use cases that are left. The order
    is based on the priority of the use cases.


\subsection{Selected architectural drivers}
    The functional drivers are:
    \begin{itemize}
        \item \emph{UC28}: Log in
        \item \emph{UC29}: Log out
    \end{itemize}

    Sessions are used to check whether or not a user is currently logged in. \\
    On successful login, a new session is created and stored in the SessionDB with
    a unqiue sessionID which is returned to the front end. The client then sends this
    sessionID in future requests to represent that the user is logged in. \\
    A database is used for checking the sessions, so that replication can be done
    on components that handle front end requests. If the session was stored in files,
    this would add extra overhead to keep all replica's that store session files consistent.
    It is easier to do this using a database. Also, a separate database is used
    to not add a lot of extra requests for sessions to another database.

    % UC28:
    %     1. The primary actor indicates he or she want to log in to the system and provides authentication credentials, for example a user name and password.
    %
    %         ApplicationProviderClient/CustomerOrganisationClient/InfrastructureOwnerClient/SysAdminClient -> RegisteredUserFacade: interface Authentication: string login(Map<string, string> credentials)
    %             Effect: Verifies the provided authentication credentials. If everything is correct, logs the user in, creates a unique session, and returns the session's ID. \\
    %                     Returns an empty string if the provided credentials are incorrect.
    %             \item Created for: UC28.1
    %
    %     2. The system veries the provided authentication credentials.
    %
    %         RegisteredUserFacade -> AuthenticationManager: interface Authentication: string login(Map<string, string> credentials)
    %             Effect: Verifies the provided authentication credentials. If everything is correct, logs the user in, creates a unique session, and returns the session's ID. \\
    %                     Returns an empty string if the provided credentials are incorrect.
    %             \item Created for: UC28.2
    %
    %         AuthenticationManager -> OtherDataDB: interface Authentication: boolean verifyAuthenticationCredentials(Map<string, string> credentials)
    %             Effect: Returns true if the provided authentication credentials are correct.
    %             \item Created for: UC28.2
    %
    %         AuthenticationManager -> SessionDB: interface Authentication: string createNewSession(int useID)
    %             Effect: Returns the ID of a new login session for a certain user.
    %             \item Created for: UC28.2
    %
    %     3. If the provided credentials are correct, the system confirms successful login to the primary actor and logs him or her in.
    %         => return value of login
    %
    %     3a. If the provided credentials are incorrect, the system noties the primary actor of this. Continue from step 1.
    %         => return value of login

    % UC29:
    %     1. The primary actor indicates he or she wants to log out of the system.
    %
    %         ApplicationProviderClient/CustomerOrganisationClient/InfrastructureOwnerClient/SysAdminClient -> RegisteredUserFacade: interface Authentication: boolean logout(string sessionID)
    %             Effect: If a session with the given sessionID exists, logs the user out and returns true.
    %             \item Created for: UC29.1
    %
    %     2. The system logs the primary actor out and indicates success to the primary actor.
    %
    %         RegisteredUserFacade -> AuthenticationManager: interface Authentication: boolean logout(string sessionID)
    %             Effect: If a session with the given sessionID exists, logs the user out and returns true.
    %             \item Created for: UC29.1
    %
    %         AuthenticationManager -> SessionDB: interface Authentication: boolean deleteSession(string sessionID)
    %             Effect: If a session with the given sessionID exists, deletes the session and returns true.
    %             \item Created for: UC29.1

    % ApplicationProviderClient/CustomerOrganisationClient/InfrastructureOwnerClient/SysAdminClient -> RegisteredUserFacade: interface Authentication: boolean doesSessionExist(string sessionID)
    %     Effect: Returns true if a session with the given sessionID exists.
    %     \item Created for: UC28
    %
    % RegisteredUserFacade -> AuthenticationManager: interface Authentication: boolean doesSessionExist(string sessionID)
    %     Effect: Returns true if a session with the given sessionID exists.
    %     \item Created for: UC28
    %
    % AuthenticationManager -> SessionDB: interface Authentication: boolean doesSessionExist(string sessionID)
    %     Effect: Returns true if a session with the given sessionID exists.
    %     \item Created for: UC28

\subsection{Instantiation and allocation of functionality}
    This section lists the new components which instantiate our solutions
    described in the section above. For each component we note the quality
    attribute or use case that prompted us to create it. Descriptions about
    the components can be found under chapter \ref{ch:elements-datatypes}. \\

    \begin{itemize}
        \item SessionDB: UC28, UC29
        \item AuthenticationManager: UC28, UC29
    \end{itemize}


\subsection{Interfaces for child modules}
    This section lists new interfaces assigned to the components defined
    in the section above. Detailed information about each interface and
    its methods can be found under chapter \ref{ch:elements-datatypes}.

    \subsubsection{ApplicationProviderFacade}
        \begin{itemize}
            \item Authentication
        \end{itemize}

    \subsubsection{AuthenticationManager}
        \begin{itemize}
            \item Authentication
        \end{itemize}

    \subsubsection{CustomerOrganisationFacade}
        \begin{itemize}
            \item Authentication
        \end{itemize}

    \subsubsection{InfrastructureOwnerFacade}
        \begin{itemize}
            \item Authentication
        \end{itemize}

    \subsubsection{NotificationHandler}
        \begin{itemize}
            \item Authentication
        \end{itemize}

    \subsubsection{OtherDataDB}
        \begin{itemize}
            \item Authentication
        \end{itemize}

    \subsubsection{RegisteredUserFacade}
        \begin{itemize}
            \item Authentication
        \end{itemize}

    \subsubsection{SessionDB}
        \begin{itemize}
            \item Authentication
        \end{itemize}

    \subsubsection{SystemAdministratorFacade}
        \begin{itemize}
            \item Authentication
        \end{itemize}
