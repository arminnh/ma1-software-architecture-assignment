\section{Decomposition 7: UC28, UC29 (Elements/Subsystem to decompose/expand)}
    At this point, all quality attributes have been handled. The remaining
    decompositions handle all of the use cases that are left. The order
    is based on the priority of the use cases.


\subsection{Selected architectural drivers}
    The functional drivers are:
    \begin{itemize}
        \item \emph{UC28}: Log in
        \item \emph{UC29}: Log out
    \end{itemize}

    WHen you check credentials, return a string sessionID to the client.
    The client sends this sessionID with every request.
    The Online Service checks the sessionID to determine if the user is logged in.
    Monika: store sessionID in otherDataDB
    PMS: store sessionID in sessionDB
    Check what is better for us
    http://shiflett.org/articles/storing-sessions-in-a-database

    UC28:
        1. The primary actor indicates he or she want to log in to the system and provides authentication credentials, for example a user name and password.

            ApplicationProviderClient/CustomerOrganisationClient/InfrastructureOwnerClient/SysAdminClient -> RegisteredUserFacade: interface Authentication: string login(Map<string, string> credentials)
                Effect: Verifies the provided authentication credentials. If everything is correct, logs the user in, creates a unique session, and returns the session's id. \\
                        Returns an empty string if the provided credentials are incorrect.
                \item Created for: UC28.1

        2. The system veries the provided authentication credentials.

            RegisteredUserFacade -> AuthenticationManager: interface Authentication: string login(Map<string, string> credentials)
                Effect: Verifies the provided authentication credentials. If everything is correct, logs the user in, creates a unique session, and returns the session's id. \\
                        Returns an empty string if the provided credentials are incorrect.
                \item Created for: UC28.2

            AuthenticationManager -> OtherDataDB: interface Authentication: boolean verifyAuthenticationCredentials(Map<string, string> credentials)
                Effect: Returns true if the provided authentication credentials are correct.
                \item Created for: UC28.2

            AuthenticationManager -> SessionDB: interface Authentication: string createNewSession(int useID)
                Effect: Returns the id of a new login session for a certain user.
                \item Created for: UC28.2

        3. If the provided credentials are correct, the system confirms successful login to the primary actor and logs him or her in.
            => return value of login

        3a. If the provided credentials are incorrect, the system noties the primary actor of this. Continue from step 1.
            => return value of login

    UC29:
        1. The primary actor indicates he or she wants to log out of the system.

            ApplicationProviderClient/CustomerOrganisationClient/InfrastructureOwnerClient/SysAdminClient -> RegisteredUserFacade: interface Authentication: boolean logout(string sessionID)
                Effect: Logs the user out and returns true if the given sessionID exists.
                \item Created for: UC29.1

        2. The system logs the primary actor out and indicates success to the primary actor.

            RegisteredUserFacade -> AuthenticationManager: interface Authentication: boolean logout(string sessionID)
                Effect: Logs the user out and returns true if the given sessionID exists.
                \item Created for: UC29.1

            AuthenticationManager -> SessionDB: interface Authentication: boolean deleteSession(string sessionID)
                Effect: If the given sessionID exists, deletes the session and returns true.
                \item Created for: UC29.1


\subsection{Instantiation and allocation of functionality}
    This section lists the new components which instantiate our solutions
    described in the section above. For each component we note the quality
    attribute or use case that prompted us to create it. Descriptions about
    the components can be found under chapter \ref{ch:elements-datatypes}. \\

    \begin{itemize}
        \item Component: (Relevant UC)
    \end{itemize}


\subsection{Interfaces for child modules}
    This section lists new interfaces assigned to the components defined
    in the section above. Detailed information about each interface and
    its methods can be found under chapter \ref{ch:elements-datatypes}.

    \subsubsection{Component}
        \begin{itemize}
            \item Interface
        \end{itemize}

\subsection{New data types}
    This section lists the new data types introduced during this decomposition.

    \begin{itemize}
        \item DateTime: Represents an instant in time, typically expressed as a date and time of day.
    \end{itemize}
