\section{Av2: Application failure}

    \subsection*{Key Decisions}

        \begin{itemize}
        	\item \texttt{ApplicationInstance}s are executed within \texttt{ApplicationContainer}s.
            \item \texttt{ApplicationContainerManager} creates/destroys/handles communication for \texttt{ApplicationContainer}s.
            \item \texttt{ApplicationContainerMonitor} monitors \texttt{ApplicationContainer}s and \texttt{ApplicationInstance}s.
            \item \texttt{ApplicationExecutionSubsystemMonitor} monitors the application execution subsystem.
        \end{itemize}
        \emph{Employed tactics and patterns:} container

    \subsection*{Rationale}
        To handle Av2, we developed the whole application execution subsystem in a decomposition
        with Av2 and important application instance related use cases. The application execution subsystem
        is composed of the following components:
        \begin{itemize}
            \item \texttt{ApplicationContainer}
            \item \texttt{ApplicationContainerMonitor}
            \item \texttt{ApplicationContainerManager}
            \item \texttt{ApplicationExecutionSubsystemMonitor}
            \item \texttt{DeviceDataConverter}
            \item \texttt{DeviceCommandConstructor}
        \end{itemize}

        Av2 states "The system is able to autonomously detect failures of its individual
        application execution components, failing applications, and failing application containers.".\\
        These responsibilities are handled by the \texttt{ApplicationContainer}, \texttt{ApplicationContainerMonitor}, and \texttt{ApplicationExecutionSubsystemMonitor}.
        The \texttt{ApplicationContainer} provides a sandbox environment for an \texttt{ApplicationInstance} to run in and has the ability
        to monitor the instance. When an application instance fails, the container notifies the \texttt{ApplicationContainerMonitor} of this.\\
        Next to this, the \texttt{ApplicationContainerMonitor} pings \texttt{ApplicationContainer}s regularly to check whether or not the container
        has failed. If a failure of an application instance/container is detected, the \texttt{ApplicationContainerMonitor} sends a command to the
        \texttt{ApplicationContainerManager} to restart the instance/container or to create a new one in case the first two restarts did not work.
        If the application instance then keeps failing, it is suspended and the application provider and affected customer organisation are notified.
        When an application fails, a message of this is sent to the other parts of the application, so that they can possible run in a degraded mode. \\
        The \texttt{ApplicationContainerMonitor} also keeps track of how many times an \texttt{ApplicationInstance} has failed, so that when the container
        fails, there also is data about the instance.\\
        Lastly, the \texttt{ApplicationExecutionSubsystemMonitor} monitors other parts of the application execution subsystem, and restarts them
        in case failure is detected. \\

        Because of how different \texttt{ApplicationInstance}s run each in their own \texttt{ApplicationContainer}, no other applications or availability
        of other functionality of the system is affected.

    \subsection*{Deployment Decisions}
        For Av2, is important that \texttt{ApplicationInstance}s are deployed on different node from the \texttt{ApplicationContainerMonitor} that is responsible for
        monitoring. If the node the \texttt{ApplicationInstance} fails, then the \texttt{ApplicationContainerMonitor} won't also automatically fail with the \texttt{ApplicationInstance}
        and interested parties can be informed. \\
        In order to make sure that no applications or functionality of the system is affected when an application instance/container fails, each \texttt{ApplicationContainer}
        runs on its own node. The \texttt{ApplicationContainerManager} keeps track of all \texttt{ApplicationContainer}.
