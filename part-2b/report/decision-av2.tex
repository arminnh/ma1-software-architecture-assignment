\section{Av2: Application failure}

    \subsection*{Key Decisions}

        \begin{itemize}
        	\item \texttt{ApplicationInstance}s are executed within \texttt{ApplicationContainer}s.
            \item \texttt{ApplicationContainerManager} creates/destroys/handles communication for \texttt{ApplicationContainer}s.
            \item \texttt{ApplicationContainerMonitor} monitors \texttt{ApplicationContainer}s and \texttt{ApplicationInstance}s.
            \item \texttt{ApplicationExecutionSubsystemMonitor} monitors the application execution subsystem.
        \end{itemize}
        \emph{Employed tactics and patterns:} container

    \subsection*{Rationale}
        RATIONALE: ApplicationContainerManager and DeviceCommandConstructor need to be
        reconfigured ivm some used interfaces

        \paragraph{Av2: Detection of failures}
            The system is able to autonomously detect failures of its individual
            application execution components, failing applications, and failing application containers. \\
            Upon detection, a SIoTIP system administrator is notified. \\
            The failure of an internal application execution component is detected within 30 seconds.
            Detection of failed hardware or crashed software happens within 5 seconds.
            SIoTIP system administrators are notified within 1 minute.\\

            The \texttt{ApplicationContainer} is a container/sandbox that has one running application instance.
            Failures of individual application execution components means that one of \texttt{ApplicationContainer}, \texttt{ApplicationContainerMonitor}, \texttt{ApplicationContainerManager} crashed.
            The \texttt{ApplicationContainerMonitor} monitors the \texttt{ApplicationContainer} instances
            to detect failing applications and the \texttt{ApplicationContainer} itself.
            If one of the other two components failed, the \texttt{ApplicationExecutionSubsystemMonitor} is put in place to detect this.
            These components will be deployed on the Online Service and on gateways.
            Since gateways are weaker machines than the ones on the Online Service,
            the ApplicationContainer can be confgured differently for gateways.
            The ApplicationContainers will then have stricter limits on
            resources used of the node they are working on.
            DeviceCommandConstructor and ApplicationContainerManager interfaces need to begin
            re-routed depening on whether they are in a Gateway or on the Online Service.

            To detect failures, we made use of the \texttt{Container} pattern.
            The application execution subsystem is composed of:
            \begin{itemize}
                \item \texttt{ApplicationContainer}
                \item \texttt{ApplicationContainerMonitor}
                \item \texttt{ApplicationContainerManager}
                \item \texttt{ApplicationExecutionSubsystemMonitor}
            \end{itemize}

            The \texttt{ApplicationContainer}s are deployed in groups on different nodes.

            \texttt{ApplicationContainer}: is a container/sandbox that has 1 running application instance
            \texttt{ApplicationContainerMonitor}: monitors the \texttt{ApplicationContainer} instances

            To detect failing applications, \texttt{ApplicationContainer} and \texttt{ApplicationContainerMonitor}
                ApplicationContainer -> ApplicationContainerMonitor: void applicationCrashed(id applicationInstanceID)

            To detect failling application containers, \texttt{ApplicationContainerMonitor}
                ApplicationContainerMonitor -> ApplicationContainer: Echo ping()
                -> we say container has crashed/failed when the following has no response:

            To detect failures of individual application execution components,
            This means that one of \texttt{ApplicationContainer}, \texttt{ApplicationContainerMonitor}, \texttt{ApplicationContainerManager} crashed.
            If the \texttt{ApplicationContainer} failed, then the \texttt{ApplicationContainerMonitor} would detect this.
            If one of the other two components failed, the \texttt{ApplicationExecutionSubsystemMonitor} is put in place to detect this.


        \paragraph{Av2: Resolution of application failures and application execution component failures}
            In case of application crash, the system autonomously restarts failed applications.
            If part of an application fails, the remaining parts remain operational,
            possibly in a degraded mode (graceful degradation). \\
            After 3 failed restarts the application is suspended, and the
            application developer and customer organisation are notified within 5 minutes.\\

            In case of failure of application execution components or an application
            container, a system administrator is notified. \\
            SIoTIP system administrators are notified within 1 minute.\\

            When an application instance fails, the ApplicationContainerMonitor detects this and
            sends a command to the ApplicationContainerManager to restart the application instance.
            The ApplicationContainerMonitor keeps track of how many times the
            application instance has been restarted after a failure. After 3 failed restarts, the monitor
            send a command to the ApplicationContainerManager to suspend the application instance and send
            a notification to the application developers of the application and to the
            affected customer organisaiton. Also, to achieve graceful degradation,
            the ApplicationContainerManager notifies other parts of the application instance
            of its suspension.

            If one of the components of the application execution subsystem fails,
            a SIoTIP system administrator is notified.

        \paragraph{Av2: Failures do not impact other applications or other functionality of the system}
            This does not affect other applications that are executing on the Online
            service or SIoTIP gateway. This does not affect the availability of
            other functionality of the system, such as the dashboards. \\
            Applications fail independently: they are executed within their own
            container to avoid application crashes to affect other applications.\\

            Each ApplicationContainer contains one application instance. If an application fails,
            then this will be handled by the application execution subsystem so this
            does not affect any other application or other functionality of the system.
            The ApplicationContainers are constructed such that failures of applications
            do not affect the containers. The ApplicationContainers are to be deployed
            on different nodes alone or grouped with other containers. Write something here.

    \subsection*{Considered Alternatives}
         The \texttt{ApplicationContainer}s are deployed in groups on different nodes.

    \subsection*{Deployment Decisions}
        For Av2 is important that the \texttt{ApplicationInstance} is deployed on another node the \texttt{ApplicationContainerMonitor} that is responsible for
        monitoring. Once the \texttt{ApplicationInstance} fails, the \texttt{ApplicationContainerMonitor} won't fail with the the \texttt{ApplicationInstance}
        and interested parties can be informed.

        These components will be deployed on the Online Service and on gateways.
        Since gateways are weaker machines than the ones on the Online Service,
        the ApplicationContainer can be confgured differently for gateways.
        The ApplicationContainers will then have stricter limits on
        resources used of the node they are working on.
        DeviceCommandConstructor and ApplicationContainerManager interfaces need to begin
        re-routed depening on whether they are in a Gateway or on the Online Service.


    \subsection*{Considered Deployment Alternatives}
        \ldots
