\section{Av1: Communication between SIoTIP gateway and Online Service}

    \subsection*{Key Decisions}

        \begin{itemize}
            \item \texttt{OSCommunicationLogic} handles communication from Gateways to the Online Service by remote method invocation.
        	\item \texttt{OnlineServiceMonitor} monitors the Gateway's connectivity to the Online Service.
            \item \texttt{RequestStore} stores all pluggable device data and application commands on Gateways.
        	\item \texttt{OnlineServiceCommunicationMonitor} monitors the internal communication components of Gateways.

            \item \texttt{GWCommunicationLogic} handles communication from the Online Service to Gateways by remote method invocation.
        	\item \texttt{GatewayMonitor} monitors the connectivity of the Online Service to gateways.
        	\item \texttt{GatewayCommunicationMonitor} monitors the internal communication components of Gateways.
        \end{itemize}
        \emph{Employed tactics and patterns:} heartbeats, ping/echo, buffering


    \subsection*{Rationale}
        To handle all communication going from Gateways to the Online Service, the \texttt{OnlineServiceCommunicationHandler}
        was introduced. It isolates all communication-related concerns along with \texttt{GatewayCommunicationHandler} on
        the Online Service. It forwards requests from one party to the other in a Remote Procedure Call-like manner and transmits results and
        possible exceptions. \\
        Av1 requres autonomous detection of failures of individual internal communication components in Gateways. This is why we added the
        \texttt{OnlineServiceCommunicationMonitor} which monitors the communication handler. If the component is to fail, the monitor will
        detect it using a ping/echo mechanism and try to restart the component. If the failure persists, the monitor makes the gateway
        reboot itself entirely. \\
        To handle more of Av1's constrains, the communication handler is decomposed into:
        \begin{itemize}
            \item \texttt{OSCommunicationLogic}
            \item \texttt{OnlineServiceMonitor}
            \item \texttt{RequestStore}
        \end{itemize}
        The \texttt{OnlineServiceMonitor} monitors the Gateway's connectivity to the Online Service. Messages sent by the Online Service
        are used as heartbeats. An update is sent to the monitor for each message. If no message has been received by the Online Service
        in somewhat less a minute, the monitor makes the gateway ping the Online Service in order to check if it is reachable. \\
        The \texttt{RequestStore} is used to satisfy the constraint that all incoming pluggable data and issued application commands
        are stored internally when the Online Service is unreachable. This is done by sending those kinds of requests through the \texttt{RequestStore}
        to store them there. Then a unique requestID is added to the request before it is passed to the \texttt{OSCommunicationLogic}.
        When the Online Service receives a request, it sends an acknowledgement for the requestID back to the gateway it came from.
        Then, the request is deleted from the \texttt{RequestStore}. This way, if the Online Service becomes unreachable at any moment,
        all data that should be stored in this situration already is stored in the \texttt{RequestStore}.\\
        After the monitor detects that a connection to the Online Service is possible again, it makes the Gateway start synchronising again.
        In this case, the \texttt{OSCommunicationLogic} requests the data from the \texttt{RequestStore} as new requests and tries again. \\
        When the Online Service is unreachable, application parts running locally on the SIoTIP Gateway continue to operate normally. Nothing is done
        to influence the way the application parts work in that situation.\\

        On the Online Service, a very similar approach is taken. There, the \texttt{GatewayCommunicationHandler} exists out of:
        \begin{itemize}
            \item \texttt{GWCommunicationLogic}
            \item \texttt{GatewayMonitor}
        \end{itemize}
        which have responsibilities analogue to the Gateway's version. Also, the \texttt{GatewayCommunicationMonitor}
        here monitors the Online Service's internal communication components, but does not restart them, as this was not necessary for the quality attribute.

        A difference between the \texttt{GatewayMonitor} and \texttt{OnlineServiceMonitor} is that the \texttt{GatewayMonitor}
        has to keep track of the connectivity of \emph{all} SIoTIP Gateways. The \texttt{GatewayMonitor} detects failures of
        gateways when 3 consecutive expected synchronisations do not arrive within 1 minute of their expected arrival time,
        as Av1 states. When a status change of a Gateway is detected, this is reflected in the \texttt{DeviceDB}. This is
        done to be able to notify a SIoTIP system administrator after the detection
        of a simultaneous outage of more than 1\% of the registered gateways.

    \subsection*{Considered Alternatives}
        \paragraph{Alternative for monitoring of Gateways}
            Instead of making the GatewayMonitor check how many synchronization periods Gateways have missed,
            we could make it ping all of them and detect outages faster. However, this would massively increase
            the amount of messages sent by the \texttt{GatewayCommunicationHandler}, and thus is not a good option.

        \paragraph{Alternative for communication}
            We have chosen that communication between Gateways and the Online Service is to be done in a "Remote Procedure Call-like manner".
            Another option would have been to use some pattern like Message Channel or Publisher-Subscriber. We believe these other ways
            to be too slow because of the way the messages would need to be stored somewhere and then some time later consumed by the Gateways. This would thus
            also take more resources than Remote Procedure Calls. Also, it might be more intuitive to develop this system using Remote Method
            Invocation, where for example a part of the \texttt{ApplicationManager} could just invoke the "sendActuationCommand" method on a Gateway object,
            rather than send a message through some channel, and parse that message when developing the Gateway communication component.

    \subsection*{Deployment Decisions}
        For both the Online Service and Gateways, the components used for communication
        (\texttt{OnlineServiceCommunicationHandler}, \texttt{GatewayCommunicationHandler}) are to be deployed on different nodes than
        their monitoring components (\texttt{OnlineServiceCommunicationMonitor}, \texttt{GatewayCommunicationMonitor}). \\
        This is done so that if the node of a communication component fails,
        its monitoring component would not also fail and thus detect nothing because it was running on the same node.
