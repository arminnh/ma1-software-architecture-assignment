\section{Av1: Communication between SIoTIP gateway and Online Service}

    \subsection*{Key Decisions}

        \begin{itemize}
            \item \texttt{OSCommunicationLogic} handles communication from Gateways to the Online Service by remote method invocation.
        	\item \texttt{OnlineServiceMonitor} monitors the Gateway's connectivity to the Online Service.
            \item \texttt{RequestStore} stores all pluggable device data and application commands on Gateways.
        	\item \texttt{OnlineServiceCommunicationMonitor} monitors the internal communication components of Gateways.

            \item \texttt{GWCommunicationLogic} handles communication from the Online Service to Gateways by remote method invocation.
        	\item \texttt{GatewayMonitor} monitors the connectivity of the Online Service to gateways.
        	\item \texttt{GatewayCommunicationMonitor} monitors the internal communication components of Gateways.
        \end{itemize}
        \emph{Employed tactics and patterns:} heartbeats, ping/echo


    \subsection*{Rationale}
        \paragraph{Av1: New Gateway responsibilities}
            The SIoTIP gateway is able to autonomously detect failures of its individual internal communication components.\\
            The Online Service should acknowledge each message sent by the SIoTIP gateway so that the gateway can detect failures.\\
            If an internal SIoTIP gateway component fails, the gateway first tries to restart the affected component.
            If the failure persists, the SIoTIP gateway reboots itself entirely. Note that the SIoTIP gateway,
            due to the occurred failure, cannot contact a system administrator itself.\\
            If (an internal communication component of) the Online Service or the communication
            channel has failed, the SIoTIP gateway will temporarily store all incoming pluggable data
            and any issued application commands internally.\\
            If the Online Service becomes unreachable, application parts running locally on the SIoTIP
            gateway continue to operate normally.\\
            The SIoTIP gateway will start synchronising with the Online service within 1 minute after the
            communication channel becomes available.\\

            The \texttt{OnlineServiceBroker} isolates communication-related concerns between Gateways and the Online Service
            along with GatewayBroker on the Online Service.Forwards requests from one party to the other and transmits results and
            possible exceptions.
            The SIoTIP gateway is able to autonomously detect failures of its individual internal communication components.
            The  \texttt{OnlineServiceBrokerMonitor} monitors the communication component on Gateways.
            If the communication component fails, the monitor tries to restart it. If the failure persists,
            the gateway reboots itself entirely.\\
            The \texttt{BrokerLogic} handles all functionality related to communication. In the \texttt{OnlineServiceBroker}
            are \texttt{RequestStore} and \texttt{OnlineServiceMonitor}.

            The \texttt{OnlineServiceMonitor} monitors the Gateway's connectivity to the Online Service. It checks acknowledgement of each
            message sent by the SIoTIP gateway. If (an internal communication component of) the Online Service or the communication
            channel has failed, all requests to the Online Service will be stopped and stored in the \texttt{RequestStore}. An explicit command for this is not necessary,
            because the requests in the \texttt{RequestStore} will not be deleted, since no acknowledgements are received
            anymore from the Online Service. It can store at least 3 days of pluggable data before old data has to be overwritten.
            After the monitor detects that a connection to the Online Service is possible
            again, it makes the gateway start synchronising again. When the Online Service is unreachable, application
            parts running locally on the SIoTIP gateway continue to operate normally.


        \paragraph{Av1: New Online Service responsibilities}
            The Online Service is able to autonomously detect failures of its individual internal communication components.\\
            The Online Service is able to detect that a SIoTIP gateway is not sending data anymore based on the expected synchronisation interval.\\
            The Online Service notifies the infrastructure manager and a SIoTIP system administrator when the outage of a SIoTIP gateway is detected.\\
            The failure of an internal SIoTIP Online Service component is detected within 30 seconds.\\
            The detection time for a failed SIoTIP gateway or channel depends on the transmission rate
            of the gateway. An outage is defined as 3 consecutive expected synchronisations that do not
            arrive within 1 minute of their expected arrival time.\\
            The infrastructure owner is notified within 5 minutes after the detection of an outage of their gateway.\\
            A SIoTIP system administrator should be notified within 1 minute after the detection
            of a simultaneous outage of more than 1\% of the registered gateways.\\

            GatewayBroker: Isolates communication-related concerns between the Online Service Gateways and along with OnlineServiceBroker on Gateways.
                       Forwards requests from one party to the other and transmits results and possible exceptions. \\
                       Sends acknowledgements for all messages sent by Gateways so that they can detect failures.
            GatewayBrokerMonitor: Monitor the communication component for communication with gateways on the Online Service.

            In GatewayBroker:
                BrokerLogic: Handles all functionality related to communication.
                GatewayMonitor: Monitors the connectivity status all gateways. Can detect that a gateway is not sending data anymore based on the expected synchronisation interval. If 3 consecutive expected synchronisations do not arrive within 1 minute of their expected arrival time,
                            this is detected as a gateway outage. When outages of gateways are detected, the infrastructure owners that own the gateways and a SIoTIP system administrator are notified.
                            When the connectivity status change of a Gateway is detected, this is saved in the DeviceDB.



    \subsection*{Considered Alternatives}
        Alternative for monitoring of gateways:
            Gateway updated come to GatewayMonitor
            We could make GatewayMonitor ping all the gateways
            However, this would increase traffic on the network

        Alternative for communication:
            Messaging, Publishher Subscriber

    \subsection*{Deployment Decisions}
        OnlineServiceBroker
        OnlineServiceBrokerMonitor

        GatewayBroker
        GatewayBrokerMonitor

        For both the Online Service and Gateways, the components used for communication
        (\texttt{OnlineServiceBroker}, \texttt{GatewayBroker}) are to be deployed on different nodes than
        their monitoring components (\texttt{OnlineServiceBrokerMonitor}, \texttt{GatewayBrokerMonitor}).
        Otherwise, if the node of a communication
        component fails, its monitoring component would also fail and thus
        nothing would be detected.

    \subsection*{Considered Deployment Alternatives}
        \ldots
