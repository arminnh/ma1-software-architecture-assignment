\section{Decomposition 1: SIoTIP System (X1, Y3, UCa, UCb, UCc)}
\subsection{Module to decompose}
In this run we decompose the \texttt{SIoTIP System}.

\subsection{Selected architectural drivers}
The non-functional drivers for this decomposition are:

\begin{itemize}
	\item \emph{Av3}: Pluggable device or mote failure
    \item \emph{M1}: Integrate new sensor or actuator manufacturer
	\item \emph{U2}: Easy installation
\end{itemize}

The related functional drivers are:

\begin{itemize}
	\item \emph{UC14}: Send heartbeat
	\item \emph{UCb}: name
	\item \emph{UCc}: name
\end{itemize}

\paragraph{Rationale}
A short discussion of why these drivers were selected for this decomposition.

\subsection{Architectural design}
\paragraph{Topic}
Discussion of the solution selected for (a part of) one of the architectural
drivers.

\subsubsection*{Alternatives considered}
\paragraph{Alternatives for solution}
A discussion of the alternative solutions and why that were not selected.

\subsection{Instantiation and allocation of functionality}
\paragraph{Decomposition}
Main aspects of the resulting decomposition.

\subparagraph{ModuleB}
Per introduced component a paragraph describing its responsibilities.

\subparagraph{ModuleC}
Per introduced component a paragraph describing its responsibilities.

\begin{figure}[!htp]
	\centering
	%\includegraphics[width=0.8\textwidth]{}
	\missingfigure[figwidth=0.8\textwidth]{Component-and-connector diagram}
	\caption{Component-and-connector diagram of this decomposition.
	}\label{fig:it1-cc_main}
\end{figure}

\paragraph{Behaviour}
If needed and explanation of the behaviour of certain aspects of the design so
far.

\begin{figure}[!htp]
	\centering
	%\includegraphics[width=0.8\textwidth]{}
	\missingfigure[figwidth=0.8\textwidth]{Sequence diagram}
	\caption{Sequence diagram illustrating a key behavioural aspect.
	}\label{fig:it1-seq_aspect1}
\end{figure}

\paragraph{Deployment}
Rationale of the allocation of components to physical nodes.

\begin{figure}[!htp]
	\centering
	%\includegraphics[width=0.8\textwidth]{}
	\missingfigure[figwidth=0.8\textwidth]{Deployment diagram}
	\caption{Deployment diagram of this decomposition.
	}\label{fig:it1-depl_main}
\end{figure}

\subsection{Interfaces for child modules}
\subsubsection*{ModuleB}
\begin{itemize}
	\item InterfaceA
	\begin{itemize}
		\item \texttt{returnType operation1(ParamType param1)} throws TypeOfException
		\begin{itemize}
			\item Effect: Describe the effect of calling this operation.
			\item Exceptions:
			\begin{itemize}
				\item TypeOfException: Describe when this exception is thrown.
			\end{itemize}
		\end{itemize}

		\item \texttt{returnType operation2()}
		\begin{itemize}
			\item Effect: Describe the effect of calling this operation.
			\item Exceptions: None
		\end{itemize}
	\end{itemize}
\end{itemize}

\subsection{Data type definitions}
Describe per complex data type used in the interfaces what it represents.

\paragraph{returnType} This data element represents X.

\paragraph{ParamType} This data element represents Y.

\subsection{Verify and refine}
This section describes per component which (parts of) the remaining
requirements it is responsible for.

\paragraph{ModuleB}
\begin{itemize}
	\item \emph{Z1}: name
	\item \emph{UCd}: name
\end{itemize}

\paragraph{ModuleC}
\begin{itemize}
	\item \emph{UCba}: name\\Description which part of the original use case is
	the responsibility of this component.
\end{itemize}
