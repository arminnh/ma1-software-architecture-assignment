\chapter{Non-functional requirements}\label{sec:non-functional}
In this section, we model the non-functional requirements for the system in the
form of \emph{quality attribute scenarios}. We provide for each type
(availability, performance and modifiability) one requirement.

\section{Availability}
\subsection{\emph{Av1}: Unusable database}
A database in the system cannot be used anymore. This could happen because of
corrupted disks, power outages, too many requests coming into the database
server, network failures, natural physical disasters, etc. A different database
will be used while this one is unusable.

\begin{itemize}
    \item \textbf{Source:} Database server
    \item \textbf{Stimulus:}
        \begin{itemize}
            \item The database crashed.
            \item The database does not respond to requests.
            \item The database does not respond to requests within reasonable time.
            \item The database returns invalid data or responses.
        \end{itemize}

    \item \textbf{Artifact:} Persistent storage
    \item \textbf{Environment:} Normal operation
    \item \textbf{Response:}
        \begin{itemize}
            \item Use a working replica until the database can be used again.
            \item Log the fault.
            \item If the problem with the database cannot be fixed automatically
                  (e.g. by an automatic restart and resynchronisation), send
                  a technician if it is possible for them to fix the problem
                  (e.g. replace corrupted disks). Otherwise (e.g. power outages,
                  natural physical disasters), send a notification
                  of high priority to SIoTIP system administrators.
        \end{itemize}

    \item \textbf{Response measure:}
        \begin{itemize}
            \item If a database becomes unusable, this should be detected
                  within 3s of the system trying to use the database.
            \item When the system detects that a database is unusable,
                  a working replica should be used within 5s.
            \item If a database is unusable because of a system crash or user error,
                  the database should restart and start its boot procedure within 5s.
            \item If a database is unusable because of corrupted disks, a technician
                  should replace the disk within 30 min.
        \end{itemize}
\end{itemize}

\subsection{\emph{Av2}: Broken sensor}
A sensor breaks. The infrastructure owner is notified and if possible, another
sensor is used for the responsibility that the broken one was fulfilling.

\begin{itemize}
    \item \textbf{Source:} Sensor
    \item \textbf{Stimulus:}
        \begin{itemize}
            \item No data can be received anymore from the sensor.
            \item The sensor stopped sending regular updates.
            \item The sensor sends invalid/corrupted data.
            \item The sensor disappeared from the heartbeats of the mote it
                  is connected to.
        \end{itemize}

    \item \textbf{Artifact:} Communication channel between sensor and gateway
    \item \textbf{Environment:} Normal operation
    \item \textbf{Response:}
        \begin{itemize}
            \item The gateway uses another sensor to be used for
                  the same responsibility as the broken one.
            \item Log the fault and notify the infrastructure owner.
        \end{itemize}

    \item \textbf{Response measure:}
        \begin{itemize}
            \item When a sensor breaks, this is detected within 2s.
            \item If there is another sensor that can fulfill the same responsibility
                  as the broken one, it should be chosen within 2s.
        \end{itemize}
\end{itemize}

\section{Performance}
\subsection{\emph{P1}: Application requests under peak load}
An application sends requests to the Online Service while the system is under
peak load. These requests could be for getting data from a specific
sensor, for configuring application settings, etc.
These request should be processed in a timely manner.

\begin{itemize}
    \item \textbf{Source:} Application
    \item \textbf{Stimulus:}
        \begin{itemize}
            \item An application sends a request to the Online Service.
        \end{itemize}

    \item \textbf{Artifact:} The whole system
    \item \textbf{Environment:} Under peak load
    \item \textbf{Response:}
        \begin{itemize}
            \item When a request comes into the system, core work necessary
                  to generate a reply is done first. Other work of less importance
                  (e.g. sending emails, generating low-priority notifications, etc.)
                  is scheduled to be done later.
            \item Load balancing is used to divide the requests over
                  available servers.
            \item Servers that are closest to source of the request are chosen
                  over servers that are farther away to minimize network delay.
        \end{itemize}

    \item \textbf{Response measure:}
        \begin{itemize}
            \item Applications get a response within 3s.
        \end{itemize}
\end{itemize}

\subsection{\emph{P2}: Mote data to application delay}
When a mote sends data to an application, that data reaches its
destination in a bounded time. This bound is determined by the priority
of the data. The possible priorities are low, medium, and high.
These priorities can be set by an infrastructure owner.

\begin{itemize}
    \item \textbf{Source:} Mote
    \item \textbf{Stimulus:}
        \begin{itemize}
            \item A mote sends data to an application.
        \end{itemize}

    \item \textbf{Artifact:} The whole system
    \item \textbf{Environment:} Normal operation
    \item \textbf{Response:}
        \begin{itemize}
            \item The data is always sent to the next node before other extra
                  work is done. For example, other work could be logging
                  or sending notifications.
        \end{itemize}

    \item \textbf{Response measure:}
        \begin{itemize}
            \item If the data priority is low, the data reaches the application
                  within 60s.
            \item If the data priority is medium, the data reaches the application
                  within 10s.
            \item If the data priority is high, the data reaches the application
                  within 1s.
        \end{itemize}
\end{itemize}

\section{Modifiability}
\subsection{\emph{M1}: Add a new type of sensor}
The SIoTIP corporation wishes to provide a new type of sensor to customers.

\begin{itemize}
    \item \textbf{Source:} SIoTIP developer
    \item \textbf{Stimulus:}
        \begin{itemize}
            \item Developers add a new type of sensor to the system
        \end{itemize}

    \item \textbf{Artifact:} System codebase and database.
    \item \textbf{Environment:} Normal operation, during a development iteration
    \item \textbf{Response:}
        \begin{itemize}
            \item The software on motes and gateways must be updated so that
                  the new sensor can be read, calibrated and configured.
            \item Components in the system responsible for storage of sensor readings
                  must be updated to save the readings of the new sensor correctly.
            \item UI components must be updated so that the readings of the
                  new sensor can be displayed correctly.
            \item This modification does not affect the way data is communicated
                  whithin components of the system.
            \item Infrastructure owners can choose when the update will happen.
                  The motes and gateways must be updated before the new
                  typ of sensor can be ordered.
            \item The updates to the UI components and storage components can
                  be deployed at run time.
                  There is no need to destroy any login sessions for this, since the
                  sensor will not be in use at that time yet.
        \end{itemize}

    \item \textbf{Response measure:}
        \begin{itemize}
            \item This modification is finished within 5 man months of development time.
            \item The deployment of the modifications to the UI components and storage
                  components is finished within 10 minutes.
        \end{itemize}
\end{itemize}

\subsection{\emph{M2}: Changes to UI components or new UI components}
Developers want to add or edit components that make up the UI.
For example, they want to improve components that are reused in dashboards.

\begin{itemize}
    \item \textbf{Source:} SIoTIP developers
    \item \textbf{Stimulus:}
        \begin{itemize}
            \item Developers add a UI component.
            \item Developers edit a UI component.
        \end{itemize}

    \item \textbf{Artifact:} User interface and platform.
    \item \textbf{Environment:} Normal operation, during a development iteration
    \item \textbf{Response:}
        \begin{itemize}
            \item Users are not able to use the system during the deployment
                  of the changes. All users that are logged in at the time
                  the deployment starts will be logged out first.
            \item All users are notified in advance of incoming downtime.
            \item The update does not affect currently ongoing activity in
                  the system.
            \item The update only affects the subsystem responsible for the UI.
        \end{itemize}

    \item \textbf{Response measure:}
        \begin{itemize}
            \item The deployment of the new UI components is finished within 30 minutes.
                  Users can log in again immediately after the deployment is finished.
        \end{itemize}
\end{itemize}

\section{Usability}
\subsection{\emph{U1}: System administrator reviews application}
Before an application can be published or when its automated tests
fail, it needs to be reviewed by a SIoTIP system administrator.

\begin{itemize}
    \item \textbf{Source:} SIoTIP system administrator
    \item \textbf{Stimulus:}
        \begin{itemize}
            \item The system administrator wants to review the application
                  and/or test logs quickly.
            \item The system administrator wants to contact the application
                  provider.
        \end{itemize}
    \item \textbf{Artifact:} System administrator dashboard
    \item \textbf{Environment:} At normal operation
    \item \textbf{Response:}
        \begin{itemize}
            \item The system administrator dashboard is built up using clear
                  components which contain informations about different
                  elements of the system. One such component displays applications
                  that are waiting for a review.
            \item The application review page contains a component that
                  describes the application's functionality. There are clear
                  buttons to accept or decline the application. There is a
                  button to open the application.
            \item Optionally, the application review contains a component which
                  displays images  of the application from an infrastructure
                  owner's point of view.
            \item There is a component which displays the log history of the application.
            \item There is a component for communication with the application provider.
        \end{itemize}

    \item \textbf{Response measure:}
        \begin{itemize}
            \item All actions for reviewing the application, accepting or
                  declining the application, reading the application's log history,
                  and contacting the application provider are possible within 5 clicks.
        \end{itemize}
\end{itemize}

\subsection{\emph{U2}: Infrastructure owner manages application}
An infrastructure owner wants to manage an application. For example, they
might want to edit the topology of some hardware, subscribe customer
organisations to applications, order new hardware, etc.

\begin{itemize}
    \item \textbf{Source:} Infrastructure owner
    \item \textbf{Stimulus:}
        \begin{itemize}
            \item The infrastructure owner wants to use the infrastructure owner
                  dashboard efficiently
            \item The infrastructure owner wants to feel comfortable with the
                  infrastructure owner dashboard
        \end{itemize}

    \item \textbf{Artifact:} Infrastructure owner dashboard
    \item \textbf{Environment:} At normal operation
    \item \textbf{Response:}
        \begin{itemize}
            \item The infrastructure owner dashboard consists of clear UI
                  components that provide information about different elements of the system.
            \item Topologies of hardware are displayed clearly in diagrams.
            \item There is a help system the user can use to learn about the functionality
                  of their dashboard. When this help system is used, it displays
                  key information about the page the user is currently
                  on. It also contains a search form so the user can learn about
                  anything they want that is relevant to their dashboard.
        \end{itemize}

    \item \textbf{Response measure:}
        \begin{itemize}
            \item All actions an infrastructure owner can take in the system can be
                  done within 5 clicks.
        \end{itemize}
\end{itemize}
