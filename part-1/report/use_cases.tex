\section{Use case overview}\label{sec:uc_overview}

\paragraph{UCXXX: send data to gateway}
A mote sends data to a gateway.
\paragraph{UCXXX: upload application}
An appplication developers uploads an application to the Online Service.
An application is automatically tested. When the application passes all tests, then the application
is available in system.
\paragraph{UCXXX: get data from sensor}
An application requests to get data from a specific sensor.
\paragraph{UCXXX: search for applications}
The customer organisation representative wants to add a new application. He has 
to send a request to the infrastructure owner.
\paragraph{UCXXX: review application}
A SIoTIP system administrator reviews and approves or declines it.
\paragraph{UC01: Manage application}
SIotIP system administrator can ma
\paragraph{UC03: Create account}
SIotIP system administrator is responsible for creating an user accounts, then
the users have acces to a dashboard.
\paragraph{UC05: set data}
The end-user that uses application can interact with devices. For example, the user
can turn off the lights. The application send an user input to Online service.
\paragraph{UC07: Update application}
An application provider can make changes in the system and update the application. 
For example, the application provider wants deploy a new functionality or fix
some bug.
\paragraph{UC08: Debug application}
The Online Service provides a debugging environment for the application provider.
He can debbug uploaded applications easily, when an error occurs.
\paragraph{UC09: get upload progress}
The application provider can follow the progress and results of the tests, that
are running after loading an application to the system. 
\paragraph{UC10: Browse applications}
The customer organisation has in customer organisation dashboard a list of available
application. He can browse those applications. 
\paragraph{UC12: Get invoices for applications }
The customer organisation can use many application. The customer organisation can get an 
overview of the invoices for their application subscriptions.
\paragraph{UC13: Send sensor data}
The mote send data from the connected sensors to the Gateway.
\paragraph{UC14: Log in/out }
When the users want to use their dashboard they have to be authorized.
\paragraph{UC15: Edit account settings}
The user can 
\paragraph{UC16: Buy new hardware }
The infrastructure owner needs to buy a hardware. The infrastructure 
owner adds the new hardware to the shopping basket. The infrastracture owner accepts 
the basket and specifying the payment information and delivery address.
\paragraph{UC17: Contact hardware manufacturer}
I AM NOT SURE ABOUT THIS DESC. The infrastructure owner contacts the hardware
manufacturer to request integration the application with sensor devices. 
\paragraph{UC18: Manage topology of hardware}
The infrastrucutre owner can add, edit or remove an information about topology 
of sensors and actuators.
\paragraph{UC19: Provide customer organisation data for registration}
The infrastructure owner write an customer organisation data to the SIoTIP system.
The SIoTIP use this data in proccess of registration.
\paragraph{UC20: Allocate hardware to customer organisation}
The infrastructure owner allocate installed sensors and actuators to specific customer
organisations.He also assign access rights to the customer organisation on each floor.
\paragraph{UC22: Manage user roles}
The customer organisation representative can manage the roles of the end-user of the applications.
He can set up a responsibility and rights of users.
\paragraph{UC23: Notify}
The users of the SIoTIP can receive notification of an action or when something get
wrong.
\paragraph{UC24: Log in}
The user has to logs in in the system.
\paragraph{UC25: Log out}
The user log out the system, when he wants to finish his activity.
\paragraph{UC26: Unsuscrite from application}
The customer organisation unsuscribe from application, when he does not need application.
\paragraph{UC27: send heartbeat}
The mote  periodically sends a heartbeats to its gateway.
\paragraph{UC28: }
\paragraph{UC29: }
\paragraph{UC30: }
\paragraph{UC31: }

\section{Detailed use cases}

\subsection{\emph{UC17}: Send sensor data}
\begin{itemize}
    \item \textbf{Name:} Send sensor data
    \item \textbf{Primary actor:} Mote
    \item \textbf{Secondary actors:} Sensor, Gateway
    \item \textbf{Interested parties:}
        \begin{itemize}
            \item \textit{Customer organisation:} pays for an application that uses this sensor.
            \item \textit{Infrastructure owner:} needs the sensor data for an application they set up.
            \item \textit{End-users:} use the sensor data in an application.
        \end{itemize}

    \item \textbf{Preconditions:}
        \begin{itemize}
            \item The mote has received data from a sensor connected to it.
            \item The mote is connected to a gateway.
        \end{itemize}

    \item \textbf{Postconditions:}
        \begin{itemize}
            \item The gateway has received and processed the sensor data.
            \item The Online Service has received the data and can process it.
        \end{itemize}

    \item \textbf{Main scenario:}
        \begin{enumerate}
           \item The mote sends the sensor data to the connected gateway.
           \item The gateway receives the data and if applicable, runs some application logic.
           \item The gateway collects data until a synchronisation point is reached.
                 At that point, the gateway sends the data to the Online Service.
        \end{enumerate}

    \item \textbf{Alternative scenarios:}
        \begin{enumerate}
            \item [3b.] The gateway determined that the data was important
                  (e.g. cause for alarm, notification, etc.) and sent the data
                  to the Online Service immediately instead of waiting for the
                  sycnhronisation point.
        \end{enumerate}

    \item \textbf{Remarks:}
        \begin{itemize}
            \item It is essential that the synchronisation protocol works
                  correctly in the presence of non-reliable network communication
                  so that there is no loss of data.
        \end{itemize}
\end{itemize}

\subsection{\emph{UC6}: Upload application}
\begin{itemize}
    \item \textbf{Name:} Upload application
    \item \textbf{Primary actor:} Application Developer
    \item \textbf{Secondary actor(s)}: SIotIP system
    \item \textbf{Interested parties:}
        \begin{itemize}
            \item \textit{Customers organisations:} want to subscribe the applications.
        \end{itemize}

    \item \textbf{Preconditions:}
        \begin{itemize}
            \item The application developer has access to his dashboard.
        \end{itemize}

    \item \textbf{Postconditions:}
        \begin{itemize}
            \item The application is uploaded into the Online Service.
            \item The application is available to customer organisations for subscription.
        \end{itemize}

    \item \textbf{Main scenario:}
    \begin{enumerate}
       \item The application developer logs in and opens his dashboard.
       \item The system provides the ability to upload new application.
       \item The application developer uploads application.
       \item The system check application and  initiates a number of automated tests.
       \item The application developer follows  the  progress  and  results
             of these tests via application provider dashboard.
       \item The application successfully passes all tests.
       \item The system makes the application available for the customers organisations.
       \item The system send a notification to the application developer.
    \end{enumerate}

    \item \textbf{Alternative scenarios:}
    \begin{enumerate}
        \item [4b.] The system can not load application and send error
                    to the application developer.
        \item [7b.] The system interrupt loading of the application,
                    because of potential memory leak.
        \item [8b.] The SIoTIP administrator performs a secondary review and decides whether to accept
                    or reject the application.
    \end{enumerate}

    \item \textbf{Remarks:}
        \begin{itemize}
            \item First remark
        \end{itemize}
\end{itemize}

\subsection{\emph{UC4}: Get data from sensors/actuators}
\begin{itemize}
    \item \textbf{Name:} Get data from sensor
    \item \textbf{Primary actor:} Application
    \item \textbf{Secondary actor(s)}: Online Service, Gateway, Mote
    \item \textbf{Interested parties:}
        \begin{itemize}
            \item \textit{End-user:} wants to get information from sensors
        \end{itemize}

    \item \textbf{Preconditions:}
        \begin{itemize}
            \item The application is uploaded in Online Service.
            \item A connection between Online Service and Getway is established.
        \end{itemize}

    \item \textbf{Postconditions:}
        \begin{itemize}
            \item The application received data from sensor.
        \end{itemize}

    \item \textbf{Main scenario:}
    \begin{enumerate}
       \item The application send request to Online Service.
       \item The Online service communicate with the Gateway.
       \item The Gateway relay information to the mote by sending request.
       \item The mote send data to Online Service (include:send sensor data (UCXX))
       \item The Online Service send data to the application.
    \end{enumerate}

    \item \textbf{Alternative scenarios:}
    \begin{enumerate}
        \item [5b.] 5b. The application does not receive any data within 2s.
                    It will retry the request 2 more times.
    \end{enumerate}

    \item \textbf{Remarks:}
        \begin{itemize}
            \item First remark
        \end{itemize}
\end{itemize}

\subsection{\emph{UC31}: Subscribe to application}
\begin{itemize}
    \item \textbf{Name:} Subscribe to application
    \item \textbf{Primary actor:} Customer orgranisation
    \item \textbf{Secondary actor(s)}: secondary actor(s)
    \item \textbf{Interested parties:}
        \begin{itemize}
            \item \textit{Name of interested party:} reason why party is interested
        \end{itemize}

    \item \textbf{Preconditions:}
        \begin{itemize}
            \item The primary actor is authenticated.
            \item The customer organisation wants to install new application.
        \end{itemize}

    \item \textbf{Postconditions:}
        \begin{itemize}
            \item First postcondition.
            \item Second postcondition.
        \end{itemize}

    \item \textbf{Main scenario:}
    \begin{enumerate}
       \item The manager of customer organisation open his dashboard.
       \item The system display him new applications.
       \item The manager choose one of the application and subbscribe for it.
       \item
    \end{enumerate}

    \item \textbf{Alternative scenarios:}
    \begin{enumerate}
        \item [3b.] Alternative at step 3
    \end{enumerate}

    \item \textbf{Remarks:}
        \begin{itemize}
            \item First remark
        \end{itemize}
\end{itemize}

\subsection{\emph{UC02}: Review application}
\begin{itemize}
    \item \textbf{Name:} Review application
    \item \textbf{Primary actor:} SIoTIP System Administrator
    \item \textbf{Secondary actor(s)}:
    \item \textbf{Interested parties:}
        \begin{itemize}
            \item \textit{Application developers:} want to add their application to the system
        \end{itemize}

    \item \textbf{Preconditions:}
        \begin{itemize}
            \item The system administrator is authenticated.
            \item The application needs to be reviewed by a system administrator.
        \end{itemize}

    \item \textbf{Postconditions:}
        \begin{itemize}
            \item The system adminstrator accepts or declines the application.
            \item The application developers have been notified of this event.
        \end{itemize}

    \item \textbf{Main scenario:}
        \begin{enumerate}
            \item The system administrator navigates to the applications component on their
                  dashboard and selects the application that needs to be reviewed.
            \item The system administrator reviews the application's functionality and
                  log history, and the message history with the developers.
            \item The system administrator indicates he wants to accept the application.
            \item The system logs this event and makes the application available to
                  customer ogranisations.
            \item The system notifies the application developers.
        \end{enumerate}

    \item \textbf{Alternative scenarios:}
        \begin{enumerate}
            \item [3b.] The system administrator indicates he wants decline to the application
                  and is prompted by the system to fill in the reason for this. The
                  application will thus not become available to customer organisations.
            \item [3c.] The system administrator indicates he wants communicate
                  with the developers before making a decision. Afterwards, return to
                  step 2.
        \end{enumerate}

    \item \textbf{Remarks:}
        \begin{itemize}
            \item When an application first gets uploaded to the system, it needs to be
                  reviewed and accepted by a system administrator. This will prevent
                  the available apps to be flooded with apps of very poor quality
                  or apps that are copies of other apps.
            \item If at some point in the future, automated testing for an app fails,
                  the app will need to be reviewed again before it can be made
                  available again.
        \end{itemize}
\end{itemize}
