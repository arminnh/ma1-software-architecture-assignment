\documentclass[english]{sareport}
% use the option peerreview for creating an anonymized version of your report
% E.g., \documentclass[english,peerreview]{sareport}

\usepackage[colorlinks, linkcolor=black, citecolor=black, urlcolor=black]{hyperref}


% Set all authors, if your group counts 2, set third author empty \authorthree{}
% Set the groupname as well
\authorone{Monika Filipcikova (r)}
\authortwo{Armin Halilovic(r)}
\groupname{Filipcikova-Halilovic}

\academicyear{2016--2017}

\casename{Shared Internet Of Things Infrastructure Platform}
\phasenumber{1}
\phasename{Domain Analysis}


\begin{document}
\maketitle

\tableofcontents

\chapter{Domain analysis}\label{sec:domain}
\section{Domain models}
This section shows the domain model(s).

\begin{figure}[!htp]
    \centering
    %\includegraphics[width=0.8\textwidth]{}
    \missingfigure[figwidth=0.8\textwidth]{Domain model}
    \caption{The domain model for the system.}\label{fig:domain_model}
\end{figure}

\section{Domain constraints}
In this section we provide additional domain constraints.

\begin{itemize}
    \item This is a first constraint.
    \item This is a second constraint.
\end{itemize}

\section{Glossary}
In this section, we provide a glossary of the most important terminology used
in this analysis.

\begin{itemize}
    \item \textbf{Term1}: definition
    \item \textbf{Term2}: definition
\end{itemize}

\chapter{Functional requirements}\label{sec:functional}
\section*{Use case model}

\begin{figure}[!htp]
    \centering
    %\includegraphics[width=0.8\textwidth]{}
    \missingfigure[figwidth=0.8\textwidth]{Use case model}
    \caption{Use case diagram for the system.}\label{fig:use_case_model}
\end{figure}

\section{Use case overview}\label{sec:uc_overview}
\paragraph{UC1: Name}
Short summary of this use case scenario

\section{Detailed use cases}
\subsection{\emph{UC1}: Name}
\begin{itemize}
    \item \textbf{Name:} Name of use case 1
    \item \textbf{Primary actor:} primary actor
    \item \textbf{Secondary actor(s)}: secondary actor(s)
    \item \textbf{Interested parties:}
        \begin{itemize}
            \item \textit{Name of interested party:} reason why party is interested
        \end{itemize}

    \item \textbf{Preconditions:}
        \begin{itemize}
            \item First precondition.
            \item Second precondition.
        \end{itemize}

    \item \textbf{Postconditions:}
        \begin{itemize}
            \item First postcondition.
            \item Second postcondition.
        \end{itemize}

    \item \textbf{Main scenario:}
    \begin{enumerate}
       \item Step 1
       \item Step 2
       \item Step 3
       \item \ldots
    \end{enumerate}

    \item \textbf{Alternative scenarios:}
    \begin{enumerate}
        \item [3b.] Alternative at step 3
    \end{enumerate}

    \item \textbf{Remarks:}
        \begin{itemize}
            \item First remark
        \end{itemize}
\end{itemize}

\chapter{Non-functional requirements}\label{sec:non-functional}
In this section, we model the non-functional requirements for the system in the
form of \emph{quality attribute scenarios}. We provide for each type
(availability, performance and modifiability) one requirement.

\section{Availability}
\subsection{\emph{Av1}: Database is down}
A database in the system does not send any data.

\begin{itemize}
    \item \textbf{Source:} External: Database server
    \item \textbf{Stimulus:}
        \begin{itemize}
            \item The database crashed / does not send any response.
            \item The database returns invalid data or response.
        \end{itemize}

    \item \textbf{Artifact:} Persistent storage
    \item \textbf{Environment:} Normal operation
    \item \textbf{Response:}
        \begin{itemize}
            \item Use a working replica until the server can be used again.
            \item If the server cannot fix the by itself, send a technician to
                  fix the problem with the database.
        \end{itemize}

    \item \textbf{Response measure:}
        \begin{itemize}
            \item If database is down because of System Crash or User Error,
                  the database should restart within 1s to 5s.
            \item If database is down because of corrupted disks, a technician
                  should replace the disk within 30 min.
            \item If restarting the database cannot fix the problem
                  (e.g. Network Failure or Natural Physical Disaster),
                  a working replica should be used within a range of 1s to 5s.
        \end{itemize}
\end{itemize}

\subsection{\emph{Av2}: Sensor breaks}
A sensor breaks. Another sensor is used for the responsibility of the broken one.

\begin{itemize}
    \item \textbf{Source:} External: sensor
    \item \textbf{Stimulus:}
        \begin{itemize}
            \item No data received anymore from sensor
            \item Sensor is missing in heartbeat of mote
        \end{itemize}

    \item \textbf{Artifact:} Communication channel between sensor and gateway
    \item \textbf{Environment:} Any state of operation, At run-time, ...
    \item \textbf{Response:}
        \begin{itemize}
            \item The gateway uses another sensor to be used for the same responsibility as the broken one.
            \item Report the failure to the infrastructure owner.
        \end{itemize}

    \item \textbf{Response measure:}
        \begin{itemize}
            \item A new sensor should be chosen within the range of 1ms to 10s % TODO: ask
        \end{itemize}
\end{itemize}

\section{Performance}
\subsection{\emph{P1}: Many applications make many requests}
Many applications make many requests simultaneously to the Online Service.
These requests should be processed in a timely manner.

\begin{itemize}
    \item \textbf{Source:} Applications
    \item \textbf{Stimulus:}
        \begin{itemize}
            \item Many applications generate requests at approximately the same time.
                  This causes the system to be under load.
        \end{itemize}

    \item \textbf{Artifact:} The whole system
    \item \textbf{Environment:} Under load
    \item \textbf{Response:}
        \begin{itemize}
            \item Load balancing should be used to divide the requests over
                  available servers.
            \item The servers that are closest to applications should be preferred
                  over servers that are farther away to minimize network delay.
        \end{itemize}

    \item \textbf{Response measure:}
        \begin{itemize}
            \item Applications get a response within [50ms, 5s].
        \end{itemize}
\end{itemize}

\subsection{\emph{P2}: Name of the quality attribute scenario}
Shortly describe the context of the scenario.

\begin{itemize}
    \item \textbf{Source:} source
    \item \textbf{Stimulus:}
        \begin{itemize}
            \item Description of a first stimulus.
            \item Description of a second stimulus.
        \end{itemize}

    \item \textbf{Artifact:} the stimulated artifact
    \item \textbf{Environment:} the condition under which the stimulus occurs
    \item \textbf{Response:}
        \begin{itemize}
            \item Describe how the system should respond to the stimulus.
        \end{itemize}

    \item \textbf{Response measure:}
        \begin{itemize}
            \item Describe how the satisfaction of a response is measured.
        \end{itemize}
\end{itemize}

\section{Modifiability}
\subsection{\emph{M1}: Name of the quality attribute scenario}
Shortly describe the context of the scenario.

\begin{itemize}
    \item \textbf{Source:} source
    \item \textbf{Stimulus:}
        \begin{itemize}
            \item Description of a first stimulus.
            \item Description of a second stimulus.
        \end{itemize}

    \item \textbf{Artifact:} the stimulated artifact
    \item \textbf{Environment:} the condition under which the stimulus occurs
    \item \textbf{Response:}
        \begin{itemize}
            \item Describe how the system should respond to the stimulus.
        \end{itemize}

    \item \textbf{Response measure:}
        \begin{itemize}
            \item Describe how the satisfaction of a response is measured.
        \end{itemize}
\end{itemize}

\subsection{\emph{M2}: Name of the quality attribute scenario}
Shortly describe the context of the scenario.

\begin{itemize}
    \item \textbf{Source:} source
    \item \textbf{Stimulus:}
        \begin{itemize}
            \item Description of a first stimulus.
            \item Description of a second stimulus.
        \end{itemize}

    \item \textbf{Artifact:} the stimulated artifact
    \item \textbf{Environment:} the condition under which the stimulus occurs
    \item \textbf{Response:}
        \begin{itemize}
            \item Describe how the system should respond to the stimulus.
        \end{itemize}

    \item \textbf{Response measure:}
        \begin{itemize}
            \item Describe how the satisfaction of a response is measured.
        \end{itemize}
\end{itemize}

\section{Usability}
\subsection{\emph{U1}: Application devs upload their app}
Application developers have built an application and wish to upload it to
the Online Service. This should go smoothly.

\begin{itemize}
    \item \textbf{Source:} Application developers
    \item \textbf{Stimulus:}
        \begin{itemize}
            \item Application developers want to use the Application provider
                  dashboard efficiently.
        \end{itemize}

    \item \textbf{Artifact:} Application provider dashboard
    \item \textbf{Environment:} At normal operation
    \item \textbf{Response:}
        \begin{itemize}
            \item Applications and their statuses should be displayed in a clear table.
            \item The UPLOAD APP button should be pretty!
        \end{itemize}

    \item \textbf{Response measure:}
        \begin{itemize}
            \item The application can be uploaded with a minimal amount of clicks needed

        \end{itemize}
\end{itemize}

\subsection{\emph{U2}: Infrastructure owner changes topology}
The infrastructure owner wants to change the topology of sensors or actuators
in the system.

\begin{itemize}
    \item \textbf{Source:} Infrastructure owner
    \item \textbf{Stimulus:}
        \begin{itemize}
            \item Infrastructure owner wants to use the infrastructure owner
                  dashboard efficiently
            \item Infrastructure owner wants to feel comfortable with the
                  infrastructure owner dashboard
        \end{itemize}

    \item \textbf{Artifact:} Infrastructure owner dashboard
    \item \textbf{Environment:} At normal operation
    \item \textbf{Response:}
        \begin{itemize}
            \item Text is aggregated into consistent paragraphs.
            \item Links and buttons have a distinct styling to make them stand out.
            \item The topology is displayed clearly in a diagram.
            \item There is a help system the infrastructure owner can use to learn
        \end{itemize}

    \item \textbf{Response measure:}
        \begin{itemize}
            \item The topology can be changed with a minimal amount of text/diagrams
                  displayed and clicks needed to do the changes. In other words,
                  the time do to this is bounded by X minutes.
        \end{itemize}
\end{itemize}

\end{document}
