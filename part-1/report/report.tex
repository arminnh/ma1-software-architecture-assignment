\documentclass[english]{sareport}
% use the option peerreview for creating an anonymized version of your report
% E.g., \documentclass[english,peerreview]{sareport}

\usepackage[colorlinks, linkcolor=black, citecolor=black, urlcolor=black]{hyperref}


% Set all authors, if your group counts 2, set third author empty \authorthree{}
% Set the groupname as well
\authorone{Monika Filipcikova (r)}
\authortwo{Armin Halilovic(r)}
\groupname{Filipcikova-Halilovic}

\academicyear{2016--2017}

\casename{Shared Internet Of Things Infrastructure Platform}
\phasenumber{1}
\phasename{Domain Analysis}


\begin{document}
\maketitle

\tableofcontents

\chapter{Domain analysis}\label{sec:domain}
\section{Domain models}
This section shows the domain model(s).

\begin{figure}[!htp]
    \centering
    %\includegraphics[width=0.8\textwidth]{}
    \missingfigure[figwidth=0.8\textwidth]{Domain model}
    \caption{The domain model for the system.}\label{fig:domain_model}
\end{figure}

\section{Domain constraints}
In this section we provide additional domain constraints.

\begin{itemize}
    \item This is a first constraint.
    \item This is a second constraint.
\end{itemize}

\section{Glossary}
In this section, we provide a glossary of the most important terminology used
in this analysis.

\begin{itemize}
    \item \textbf{Term1}: definition
    \item \textbf{Term2}: definition
\end{itemize}

\chapter{Functional requirements}\label{sec:functional}
\section*{Use case model}

\begin{figure}[!htp]
    \centering
    %\includegraphics[width=0.8\textwidth]{}
    \missingfigure[figwidth=0.8\textwidth]{Use case model}
    \caption{Use case diagram for the system.}\label{fig:use_case_model}
\end{figure}

\section{Use case overview}\label{sec:uc_overview}
\paragraph{UC1: Name}
Short summary of this use case scenario

\section{Detailed use cases}
\subsection{\emph{UC1}: Name}
\begin{itemize}
    \item \textbf{Name:} Name of use case 1
    \item \textbf{Primary actor:} primary actor
    \item \textbf{Secondary actor(s)}: secondary actor(s)
    \item \textbf{Interested parties:}
        \begin{itemize}
            \item \textit{Name of interested party:} reason why party is interested
        \end{itemize}

    \item \textbf{Preconditions:}
        \begin{itemize}
            \item First precondition.
            \item Second precondition.
        \end{itemize}

    \item \textbf{Postconditions:}
        \begin{itemize}
            \item First postcondition.
            \item Second postcondition.
        \end{itemize}

    \item \textbf{Main scenario:}
    \begin{enumerate}
       \item Step 1
       \item Step 2
       \item Step 3
       \item \ldots
    \end{enumerate}

    \item \textbf{Alternative scenarios:}
    \begin{enumerate}
        \item [3b.] Alternative at step 3
    \end{enumerate}

    \item \textbf{Remarks:}
        \begin{itemize}
            \item First remark
        \end{itemize}
\end{itemize}

\chapter{Non-functional requirements}\label{sec:non-functional}
In this section, we model the non-functional requirements for the system in the
form of \emph{quality attribute scenarios}. We provide for each type
(availability, performance and modifiability) one requirement.

\section{Availability}
\subsection{\emph{Av1}: Database is down}
A database in the system does not send any data.

\begin{itemize}
    \item \textbf{Source:} External: Database server
    \item \textbf{Stimulus:}
        \begin{itemize}
            \item The database crashed / does not send any response.
            \item The database returns invalid data or response.
        \end{itemize}

    \item \textbf{Artifact:} Persistent storage
    \item \textbf{Environment:} Normal operation
    \item \textbf{Response:}
        \begin{itemize}
            \item Use a working replica until the server can be used again.
            \item If the server cannot fix the by itself, send a technician to
                  fix the problem with the database.
        \end{itemize}

    \item \textbf{Response measure:}
        \begin{itemize}
            \item If database is down because of System Crash or User Error,
                  the database should restart within 5s.
            \item If database is down because of corrupted disks, a technician
                  should replace the disk within 30 min.
            \item If restarting the database cannot fix the problem
                  (e.g. Network Failure or Natural Physical Disaster),
                  a working replica should be used within 5s.
        \end{itemize}
\end{itemize}

\subsection{\emph{Av2}: Sensor breaks}
A sensor breaks. Another sensor is used for the responsibility of the broken one.

\begin{itemize}
    \item \textbf{Source:} External: sensor
    \item \textbf{Stimulus:}
        \begin{itemize}
            \item No data received anymore from sensor
            \item Sensor is missing in heartbeat of mote
        \end{itemize}

    \item \textbf{Artifact:} Communication channel between sensor and gateway
    \item \textbf{Environment:} Any state of operation, At run-time, ...
    \item \textbf{Response:}
        \begin{itemize}
            \item The gateway uses another sensor to be used for 
                  the same responsibility as the broken one.
            \item Report the failure to the infrastructure owner.
        \end{itemize}

    \item \textbf{Response measure:}
        \begin{itemize}
            \item A new sensor should be chosen within 10s.
        \end{itemize}
\end{itemize}

\section{Performance}
\subsection{\emph{P1}: Many applications make many requests}
Many applications make many requests simultaneously to the Online Service.
These requests should be processed in a timely manner.

\begin{itemize}
    \item \textbf{Source:} Applications
    \item \textbf{Stimulus:}
        \begin{itemize}
            \item Many applications generate requests at approximately the same time.
                  This causes the system to be under load.
        \end{itemize}

    \item \textbf{Artifact:} The whole system
    \item \textbf{Environment:} Under high load
    \item \textbf{Response:}
        \begin{itemize}
            \item Load balancing should be used to divide the requests over
                  available servers.
            \item The servers that are closest to applications should be preferred
                  over servers that are farther away to minimize network delay.
        \end{itemize}

    \item \textbf{Response measure:}
        \begin{itemize}
            \item Applications get a response within 5s.
        \end{itemize}
\end{itemize}

\subsection{\emph{P2}: Mote data to application delay}
When a mote sends data an application, that data reaches its 
destination in a bounded time. This bound is determined by the priority
of the data. The possible priorities are low, medium, and high. 
These priorities can be set by an infrastructure owner.

\begin{itemize}
    \item \textbf{Source:} Mote
    \item \textbf{Stimulus:} 
        \begin{itemize}
            \item Data was sent from a mote to an application. For example, 
                  this could be data from a sensor/a confirmation from an actuator.
        \end{itemize}

    \item \textbf{Artifact:} The whole system
    \item \textbf{Environment:} At runtime under normal operation
    \item \textbf{Response:}
        \begin{itemize}
            \item The data is always sent to the next node before other extra 
                  work is done. For example, this other work could be logging
                  or sending notifications.
        \end{itemize}

    \item \textbf{Response measure:}
        \begin{itemize}
            \item If the data priority is low, the data reaches the application 
            within 60s.
            \item If the data priority is medium, the data reaches the application 
            within 10s.
            \item If the data priority is high, the data reaches the application 
            within 1s.
        \end{itemize}
\end{itemize}

\section{Modifiability}
\subsection{\emph{M1}: Add a new type of sensor}
We wish to provide a new type of sensor to customers.

\begin{itemize}
    \item \textbf{Source:} SIOTIP developers
    \item \textbf{Stimulus:} 
        \begin{itemize}
            \item Developers add a new type of sensor to the system
        \end{itemize}

    \item \textbf{Artifact:} System codebase and database.
    \item \textbf{Environment:} During a development iteration
    \item \textbf{Response:}
        \begin{itemize}
            \item Describe how the system should respond to the stimulus.
        \end{itemize}

    \item \textbf{Response measure:}
        \begin{itemize}
            \item Describe how the satisfaction of a response is measured.
        \end{itemize}
\end{itemize}

\subsection{\emph{M2}: Developers changes UI components}
Developer wants to change add or edit components that make up the UI.
For example, those components can reused in dashboards.

\begin{itemize}
    \item \textbf{Source:} SIOTIP developers
    \item \textbf{Stimulus:} 
        \begin{itemize}
            \item Developers add a UI component.
            \item Developers edit a UI component.
        \end{itemize}

    \item \textbf{Artifact:} User interface and platform.
    \item \textbf{Environment:} During a development iteration
    \item \textbf{Response:}
        \begin{itemize}
            \item 
        \end{itemize}

    \item \textbf{Response measure:}
        \begin{itemize}
            \item Describe how the satisfaction of a response is measured.
        \end{itemize}
\end{itemize}

\section{Usability}
\subsection{\emph{U1}: System administrator reviews application}
Before an application can be published or when its automated tests 
fail, it needs to be reviewed by an SIoTIP system administrator.  

\begin{itemize}
    \item \textbf{Source:} SIoTIP system administrator
    \item \textbf{Stimulus:}
        \begin{itemize}
            \item The system administrator wants to review the application 
                  and/or test logs quickly.
            \item The system administrator wants to contact the application
        \end{itemize}
    \item \textbf{Artifact:} System administrator dashboard
    \item \textbf{Environment:} At normal operation
    \item \textbf{Response:}
        \begin{itemize}
            \item The dashboard contains a component that describes the application's 
                  functionality. There are clear buttons to accept or decline 
                  the application. There is a button to open the application.
            \item Optionally, this component displays images
                  of the application from a infrastructure owner's point of view.
            \item There is a component which displays the log history of the application.
            \item There is a component for communication with the application developers.      
        \end{itemize}

    \item \textbf{Response measure:}
        \begin{itemize}
            \item All actions for reviewing the application, accepting or
                  declining the application, reading the application's log history,
                  and contacting developers are possible within 3 clicks.
        \end{itemize}
\end{itemize}

\subsection{\emph{U2}: Infrastructure owner changes topology}
The infrastructure owner wants to change the topology of sensors or actuators
in the system.

\begin{itemize}
    \item \textbf{Source:} Infrastructure owner
    \item \textbf{Stimulus:}
        \begin{itemize}
            \item Infrastructure owner wants to use the infrastructure owner
                  dashboard efficiently
            \item Infrastructure owner wants to feel comfortable with the
                  infrastructure owner dashboard
        \end{itemize}

    \item \textbf{Artifact:} Infrastructure owner dashboard
    \item \textbf{Environment:} At normal operation
    \item \textbf{Response:}
        \begin{itemize}
            \item Text is aggregated into consistent paragraphs.
            \item Links and buttons have a distinct styling to make them stand out.
            \item The topology is displayed clearly in a diagram.
            \item There is a help system the infrastructure owner can use to learn
        \end{itemize}

    \item \textbf{Response measure:}
        \begin{itemize}
            \item The topology can be changed with a minimal amount of text/diagrams
                  displayed and clicks needed to do the changes. In other words,
                  the time do to this is bounded by X minutes.
        \end{itemize}
\end{itemize}

\end{document}
